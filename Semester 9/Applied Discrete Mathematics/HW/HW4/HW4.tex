\documentclass{article}

\usepackage{fancyhdr}
\usepackage{extramarks}
\usepackage{amsmath}
\usepackage{amsthm}
\usepackage{amsfonts}
\usepackage{tikz}
\usepackage[plain]{algorithm}
%\usepackage{algpseudocode}
\usepackage{amssymb}
\usepackage{enumitem}
\usepackage{relsize}
\usepackage{textcomp}
\usepackage{graphicx}
\usepackage{bm}
\usepackage{wasysym}
%\usepackage{textcomp}
%\usepackage{tabularx}
\usepackage{xcolor,colortbl}

%\usetikzlibrary{automata,positioning}
%\usepgfplotslibrary{external} 
%\tikzexternalize

%
% Basic Document Settings
%

\topmargin=-0.45in
\evensidemargin=0in
\oddsidemargin=0in
\textwidth=6.5in
\textheight=9.0in
\headsep=0.25in

\linespread{1.1}

\pagestyle{fancy}
\lhead{\hmwkAuthorName}
\chead{\hmwkClass\ (\hmwkClassInstructor\ \hmwkClassTime): \hmwkTitle}
\rhead{\hmwkDueDate}
\lfoot{\lastxmark}
\cfoot{\thepage}

\newcommand{\minus}{\scalebox{0.5}[1.0]{$-$}}
\newcommand\tab[1][0.5cm]{\hspace*{#1}}
\renewcommand\headrulewidth{0.4pt}
\renewcommand\footrulewidth{0.4pt}
\renewcommand{\theenumi}{\Alph{enumi}}

\setlength\parindent{0pt}

%
% Create Problem Sections
%

\newcommand{\enterProblemHeader}[1]{
	\nobreak\extramarks{}{{#1} continued on next page\ldots}\nobreak{}
	\nobreak\extramarks{{#1} (continued)}{{#1} continued on next page\ldots}\nobreak{}
}

\newcommand{\exitProblemHeader}[1]{
	\nobreak\extramarks{{#1} (continued)}{{#1} continued on next page\ldots}\nobreak{}
	\stepcounter{homeworkProblemCounter}
	\nobreak\extramarks{{#1}}{}\nobreak{}
}

\newcounter{homeworkProblemCounter}
\setcounter{secnumdepth}{0}
\newcounter{partCounter}
\nobreak\extramarks{Problem \arabic{homeworkProblemCounter}}{}\nobreak{}

\newenvironment{homeworkProblem}[1][-1]{
	\subsection*{{#1}:}
	\enterProblemHeader{{#1}}
	\exitProblemHeader{{#1}}
}

\newcommand{\hmwkTitle}{Homework 4}
\newcommand{\hmwkDueDate}{October 30, 2019}
\newcommand{\hmwkClass}{MTH 5051}
\newcommand{\hmwkClassTime}{Section 01}
\newcommand{\hmwkClassInstructor}{Dr. Jim Jones}
\newcommand{\hmwkAuthorName}{\textbf{Eric Pereira}}

%
% Title Page
%
\pagenumbering{gobble}
\title{
	\vspace{2in}
	\textmd{\textbf{\hmwkClass:\ \hmwkTitle}}\\
	\normalsize\vspace{0.1in}\small{Due\ on\ \hmwkDueDate\ at 11:59pm}\\
	\vspace{0.1in}\large{\textit{\hmwkClassInstructor\ \hmwkClassTime}}
	\vspace{3in}
}

\author{\hmwkAuthorName}
\date{}

\renewcommand{\part}[1]{\textbf{\large Part \Alph{partCounter}}\stepcounter{partCounter}\\}

%
% Various Helper Commands
%

% Useful for algorithms
\newcommand{\alg}[1]{\textsc{\bfseries \footnotesize #1}}

% For derivatives
\newcommand{\deriv}[1]{\frac{\mathrm{d}}{\mathrm{d}x} (#1)}

% For partial derivatives
\newcommand{\pderiv}[2]{\frac{\partial}{\partial #1} (#2)}

% Integral dx
\newcommand{\dx}{\mathrm{d}x}

% Alias for the Solution section header
\newcommand{\solution}{\textbf{\large Solution}}

% Probability commands: Expectation, Variance, Covariance, Bias
\newcommand{\E}{\mathrm{E}}
\newcommand{\Var}{\mathrm{Var}}
\newcommand{\Cov}{\mathrm{Cov}}
\newcommand{\Bias}{\mathrm{Bias}}

\begin{document}
	\maketitle
	
	\pagebreak
	\pagenumbering{arabic}
	
	
	%%%%%%%%%%%%%%%%%%%      
	%	NEW SECTION   %
	%%%%%%%%%%%%%%%%%%%
	\section{Chapter 9.1}
	
	%%%%%%%%%%%%%%%%%%%%%%%%%%%%%%%%%%%%%%%%%%%%%%%%%%%%%%%%%%%%%%%%%%%%%%%%%%%%%%%%%%
	%                                                                                %
	%                          Section 9.1 Problem 2                                 %
	%                                                                                %
	%%%%%%%%%%%%%%%%%%%%%%%%%%%%%%%%%%%%%%%%%%%%%%%%%%%%%%%%%%%%%%%%%%%%%%%%%%%%%%%%%%
	
	\begin{homeworkProblem}[Problem 5]
		\tab Determine the generating function for the number of ways
		to distribute 35 pennies (from an unlimited supply) among five
		children if 
		\begin{enumerate}[label=\alph*)]
			\item There are no restrictions;
			\item Each child gets at least 1\cent;
			\item Each child gets at least 2\cent;
			\item The oldest child gets at least 10\cent; and
			\item The two youngest children must each get at least 10\cent.
		\end{enumerate}
		
		
		\textbf{\\Solution:}
		\begin{enumerate}[label=\alph*)]
			\item For this problem we have 5 children, and 35 pennies so the generating function would
				be:
				\begin{align*}
					(1+x+x^2+x^3+..+x^{35})^5
				\end{align*}
			\item This would be very similar except it would not include the 1 in the addition of numbers
				because there is no way that a single child could not have all the coins, so it looks like:
				\begin{align*}
					(x+x^2+X^3+...+x^{35})^5
				\end{align*}
			\item Here you don't include any case where someone has 1 penny, which gives you:
				\begin{align*}
					(x^2+x^3+x^4+...+x^{35})^5
				\end{align*}
			\item If one child gets at least 10\cent the equation would look a bit different. You 
				would have one child that has 10\cent and at most the other children could have is 25\cent.
				This would look.
				\begin{align*}
					(x^{10}+x^{11}+x^{12}+...+x^{35})(1+x+x^2+x^3+...+x^25)^4
				\end{align*}
			\item Now if the 2 youngest children are receiving 10\cent, it would look very similar to the last
				program except:
				\begin{align*}
					(x^{10}+x^{11}+x^{12}+...+x^{25})^2(1+x+x^2+x^3+...+x^{15})^3
				\end{align*}
		\end{enumerate}
		
	\end{homeworkProblem} 


	%%%%%%%%%%%%%%%%%%%%%%%%%%%%%%%%%%%%%%%%%%%%%%%%%%%%%%%%%%%%%%%%%%%%%%%%%%%%%%%%%%
	%                                                                                %
	%                          Section 9.1 Problem 5                                 %
	%                                                                                %
	%%%%%%%%%%%%%%%%%%%%%%%%%%%%%%%%%%%%%%%%%%%%%%%%%%%%%%%%%%%%%%%%%%%%%%%%%%%%%%%%%%
	
	\begin{homeworkProblem}[Problem 5]
		\tab Find the generating function for the number of integer solutions to the equation
		$c_1+c_2+c_3+c_4=20$ where $-3\le{c_1}, -3\le{c_2}, -5\le{c_3\le{5}},$ and $ 0 \le{c_4}$.
		
		\textbf{\\Solution:}
		
		\tab The first thing that we should do is simplify the equation so that we can consider the worst case
		scenario, so this equation would look like:
		\begin{align*}
			(c_1+3)+(c_2+3)+(c_3+5)+c_4=31
		\end{align*}
		
		\tab This makes it much easier to turn into a generating function, no negative numbers to deal with.
		Let's now correlate this to some other equation to make it easier to describe.
		\begin{align*}
			x_1+x_2+x_3+x_4=31
		\end{align*}
		
		\tab In this case, when creating the generating function $x_1, x_2,$ and $x_4$ are all the same, they are
		some value greater than or equal to 0, up to 31. The problem with $x_3$ is that it has to be less than
		or equal to 10, as we added 5 in the original equation above $x_3$ can be no more than 10, and no less than
		0 which. This would make the generating function look like:
		\begin{align*}
			(1+x_1+x_2+...+x_{31})^3(1+x_1+x_2+...+x_10)
		\end{align*}
	\end{homeworkProblem} 

	%%%%%%%%%%%%%%%%%%%      
	%	NEW SECTION   %
	%%%%%%%%%%%%%%%%%%%
	\section{Chapter 9.2}
	
	%%%%%%%%%%%%%%%%%%%%%%%%%%%%%%%%%%%%%%%%%%%%%%%%%%%%%%%%%%%%%%%%%%%%%%%%%%%%%%%%%%
	%                                                                                %
	%                          Section 9.2 Problem 1                                 %
	%                                                                                %
	%%%%%%%%%%%%%%%%%%%%%%%%%%%%%%%%%%%%%%%%%%%%%%%%%%%%%%%%%%%%%%%%%%%%%%%%%%%%%%%%%%
	
	\begin{homeworkProblem}[Problem 1]
		\tab Find generating functions for the following sequences. [For example, in the case
		of the sequence 0,1,3,9,27,..., the answer required is $\dfrac{x}{1-3x}$, not $\sum_{i=0}^{\infty}3^ix^{i+1}$
		or simply $0+x+3x^2+9x^3+...$.]
		\begin{enumerate}[label=\alph*)]
			\item $\binom{8}{0}, \binom{8}{1}, \binom{8}{2},...,\binom{8}{8}$
			\item $\binom{8}{1}, 2\binom{8}{2},3\binom{8}{3},...,8\binom{8}{8}$
			\item $1,-1,1,-1,1,-1,...$
			\item $0,0,0,-6,6,-6,6,...$
			\item $1,0,1,0,1,0,1,...$
			\item $0,0,1,a^2,a^3,...,a\ne0$
		\end{enumerate}
		
		
		\textbf{\\Solution:}
		\begin{enumerate}[label=\alph*)]
			\item This would be:
				\begin{align*}
					(1+x)^8
				\end{align*}
			\item This would be:
				\begin{align*}
					8(1+x)^7
				\end{align*}
			\item This would be:
				\begin{align*}
					\dfrac{1}{(1+x)}
				\end{align*}
			\item This would be:
				\begin{align*}
					\frac{6x^3}{(1+x)}
				\end{align*}
			\item This would be:
				\begin{align*}
					\dfrac{1}{(1-x^2)}
				\end{align*}
			\item This would be:
				\begin{align*}
					\dfrac{x^2}{(1-ax)}
				\end{align*}
		\end{enumerate}
		
	\end{homeworkProblem} 

	%%%%%%%%%%%%%%%%%%%%%%%%%%%%%%%%%%%%%%%%%%%%%%%%%%%%%%%%%%%%%%%%%%%%%%%%%%%%%%%%%%
	%                                                                                %
	%                          Section 9.2 Problem 9                                 %
	%                                                                                %
	%%%%%%%%%%%%%%%%%%%%%%%%%%%%%%%%%%%%%%%%%%%%%%%%%%%%%%%%%%%%%%%%%%%%%%%%%%%%%%%%%%
	
	\begin{homeworkProblem}[Problem 9]
		\tab Find the coefficient of $x^{15}$ in each of the following.
		\begin{enumerate}[label=\alph*)]
			\item $x^3(1-2x)^{10}$
			\item $\dfrac{x^3-5x}{(1-x)^3}$
			\item $\dfrac{(1+x)^4}{(1-x)^4}$
		\end{enumerate}
		
		
		\textbf{\\Solution:}
		
		\begin{enumerate}[label=\alph*)]
			\item First is to expand on this problem. The expansion form would look like:
				\begin{align*}
					x^3\left(\sum_{i=0}^{10} (-1)^i\binom{10}{i}(2x)^i \right)
				\end{align*}
				\tab We find a small problem here. It seems that the largest possible $x$
				value is $x^13$ and not $x^15$, therefore the answer is \textbf{0}.
			\item For this problem we want to do what we did before and expand the equation.
				The easiest way to expand this is:
				\begin{align*}
					x^3-5x\left(\sum_{i=0}^{\infty}\right)
				\end{align*}
			\item 
		\end{enumerate}
		
		
	\end{homeworkProblem} 

	%%%%%%%%%%%%%%%%%%%%%%%%%%%%%%%%%%%%%%%%%%%%%%%%%%%%%%%%%%%%%%%%%%%%%%%%%%%%%%%%%%
	%                                                                                %
	%                          Section 9.1 Problem 11                                %
	%                                                                                %
	%%%%%%%%%%%%%%%%%%%%%%%%%%%%%%%%%%%%%%%%%%%%%%%%%%%%%%%%%%%%%%%%%%%%%%%%%%%%%%%%%%
	
	\begin{homeworkProblem}[Problem 11]
		\tab In how many ways can 3000 identical envelopes be divided, in packages of 25,
		among four student groups so that each group gets at least 150, but not more than 1000,
		of the envelopes?
		
		\textbf{\\Solution:}
		
		\tab We can start this problem by simplifying it a bit, if the envelopes have to be shipped in packages
		of 25 then we can calculate the amount of packages and use that as our main value. The amount of packages in 
		this case would be $\frac{3000}{25}=600$. Now, we have to calculate the minimum and maximum amount of envelopes
		a student can have into how many packages that would be. The minimum amount of packages a student group
		could have would be $\frac{150}{25}=6$, and the maximum packages a student group could have is 
		$\frac{1000}{25}=40$. This means that the answer would look something like:
		\begin{align*}
			(x^6+x^7+x^8+...+x^{40})^4
		\end{align*}
		
	\end{homeworkProblem}

	%%%%%%%%%%%%%%%%%%%      
	%	NEW SECTION   %
	%%%%%%%%%%%%%%%%%%%
	\section{Supplementary Exercises: Chapter 9} 
	
	%%%%%%%%%%%%%%%%%%%%%%%%%%%%%%%%%%%%%%%%%%%%%%%%%%%%%%%%%%%%%%%%%%%%%%%%%%%%%%%%%%
	%                                                                                %
	%                    	Supplementary Exercises Problem 6                        %
	%                                                                                %
	%%%%%%%%%%%%%%%%%%%%%%%%%%%%%%%%%%%%%%%%%%%%%%%%%%%%%%%%%%%%%%%%%%%%%%%%%%%%%%%%%%
	
	\begin{homeworkProblem}[Problem 6]
		\tab How many 10-digit telephone numbers use only the digits 1,3,5, and 7, with each
		digit appearing at least twice or not at all. 
			
		\textbf{\\Solution:\\}
		
		\tab So, Primarily in this problem is to recognize how many possible solutions exist for the 10 numbers
		in general. 
		
	\end{homeworkProblem}

	%%%%%%%%%%%%%%%%%%%      
	%	NEW SECTION   %
	%%%%%%%%%%%%%%%%%%%
	\section{Chapter 10.1} 
	
	%%%%%%%%%%%%%%%%%%%%%%%%%%%%%%%%%%%%%%%%%%%%%%%%%%%%%%%%%%%%%%%%%%%%%%%%%%%%%%%%%%
	%                                                                                %
	%                          Section 10.1 Problem 1                                %
	%                                                                                %
	%%%%%%%%%%%%%%%%%%%%%%%%%%%%%%%%%%%%%%%%%%%%%%%%%%%%%%%%%%%%%%%%%%%%%%%%%%%%%%%%%%
	
	\begin{homeworkProblem}[Problem 1]
		\tab Find a recurrence relation, with initial condition, that uniquely determines each 
		of the following geometric progressions.
		\begin{enumerate}[label=\alph*)]
			\item 2, 10, 50, 250, ...
			\item 6, -18, 54, -162, ...
			\item 7, $\frac{14}{5}$, $\frac{28}{25}$, $\frac{56}{125}$, ...
		\end{enumerate}
		\textbf{\\Solution:}
		
		\begin{enumerate}[label=\alph*)]
			\item It seems that this is just multiplying the previous number by 5, starting at 2. This
				would look like:
				\begin{align*}
					5a_{n-1},\ n\ge{1},\ a_0=2 
				\end{align*}
			\item This seems to start at 6, and be multiplied by a -3 each time. This would look like:
				\begin{align*}
					-3a_{n-1},\ n\ge{1},\ a_0=6
				\end{align*}
			\item This starts at 7, and seems to be multiplied by $\frac{2}{5}$ each time. This would look like:
				\begin{align*}
					\frac{2a_{n-1}}{5},\ n\ge{1},\ a_0=7
				\end{align*}
		\end{enumerate}
		
		
	\end{homeworkProblem}

	%%%%%%%%%%%%%%%%%%%%%%%%%%%%%%%%%%%%%%%%%%%%%%%%%%%%%%%%%%%%%%%%%%%%%%%%%%%%%%%%%%
	%                                                                                %
	%                          Section 10.1 Problem 3                                %
	%                                                                                %
	%%%%%%%%%%%%%%%%%%%%%%%%%%%%%%%%%%%%%%%%%%%%%%%%%%%%%%%%%%%%%%%%%%%%%%%%%%%%%%%%%%
	
	\begin{homeworkProblem}[Problem 3]
		\tab If $a_n$, $n\ge{0}$, is the unique solution of the recurrence relation 
		$a_{n+1}-da_n=0$, and $a_3=\frac{153}{49}$, $a_5=\frac{1377}{2401}$, what is $d$?
		
		\textbf{\\Solution:\\}
		
		\tab I think the first step to this problem is to solve for the case $a_4$. This would could be done
		by figuring out an equation for any generalized recurrence based on $a_0$. We could rearrange the
		original equation given and make:
		\begin{align*}
			a_n=d^na_0
		\end{align*}
		\tab From here we can plug in both $a_5$ and $a_3$ to get $d^2$ and then get the square root
		of that. This would look like:
		\begin{align*}
			& a_3=d^3a_0=\frac{153}{49} \\
			& a_5=d^5a_0=\frac{1377}{2401} \\
			& \frac{a_5}{a_3}=d^2=\frac{9}{49} \\
			& d=\sqrt{\frac{9}{49}}=\bm{\pm{\frac{3}{7}}}
		\end{align*}
		
	\end{homeworkProblem}

	%%%%%%%%%%%%%%%%%%%%%%%%%%%%%%%%%%%%%%%%%%%%%%%%%%%%%%%%%%%%%%%%%%%%%%%%%%%%%%%%%%
	%                                                                                %
	%                          Section 10.1 Problem 4                                %
	%                                                                                %
	%%%%%%%%%%%%%%%%%%%%%%%%%%%%%%%%%%%%%%%%%%%%%%%%%%%%%%%%%%%%%%%%%%%%%%%%%%%%%%%%%%
	
	
	\begin{homeworkProblem}[Problem 4]
		\tab The number of bacteria in a culture is 1000, (approximately), and this number increases
		by 250\% every two hours. Use a recurrence relation to determine the number of bacteria present
		after one day.
		
		\textbf{\\Solution:\\}
		
		\tab Well, to start initially we have to figure out approximately how much the culture increases in one
		day. This can be done by dividing 24 (the hours we have in a day) by the 2 (the time we know the bacteria
		increases by), and put 2.5 (the amount it increases by) to the power of that number. We know that
		$a_0=1000$ so the equation would look like:
		\begin{align*}
			a_{n}+2.5a_{n}, n/ge{1}, a_0=1000
		\end{align*}
		\tab Now to get more in depth we can solve this by:
		\begin{align*}
			&a_{n}=(3.5)a_0 \\
			&n=10 \\ \\
			&a_{12}=(3.5)^{12}\times{1000}=\bm{3,379,220,508.06}
		\end{align*}
		
	\end{homeworkProblem}


	%%%%%%%%%%%%%%%%%%%      
	%	NEW SECTION   %
	%%%%%%%%%%%%%%%%%%%
	\section{Chapter 10.2} 
	
	%%%%%%%%%%%%%%%%%%%%%%%%%%%%%%%%%%%%%%%%%%%%%%%%%%%%%%%%%%%%%%%%%%%%%%%%%%%%%%%%%%
	%                                                                                %
	%                          Section 10.2 Problem 1                                %
	%                                                                                %
	%%%%%%%%%%%%%%%%%%%%%%%%%%%%%%%%%%%%%%%%%%%%%%%%%%%%%%%%%%%%%%%%%%%%%%%%%%%%%%%%%%
	\begin{homeworkProblem}[Problem 1]
		\tab Solve the following recurrence relations. (No final answer should involve
		complex numbers)
		
		\begin{enumerate}[label=\alph*)]
			\item $a_n=5a_{n-1}+6a_{n-2}, n\ge{2}, a_0=1, a_1=3$
			\setcounter{enumi}{2}
			\item $a_{n+2}+a_n=0, n\ge{0}, a_0=0, a_1=3$
			\item $a_n-6a_{n-1}+9a_{n-2}=0, n\ge{2}, a_0=5, a_1=12$
		\end{enumerate}
		
		\textbf{\\Solution:\\}
		
		\begin{enumerate}[label=\alph*)]
			\item Let's start by saying $a_n=cr^n$. It is also important to mention that $c,r\neq{0}$
			If this is the case then we can rewrite the recurrence
				equation as:
				\begin{align*}
					r^2-5r-6=0
				\end{align*}
				\tab Now from here we want to find the roots of this in order to create a recurrence relation.
				the roots are:
				\begin{align*}
					&r^2-5r-6 \\
					&(r+1)(r-6)
				\end{align*}
				\tab and to continue on with solving for distinct real roots:
				\begin{align*}
					&a_n=(-1)^nA+(6)^nB \\
					&a_0=1=A+B \\
					&B=1-A \\
					&a_1=3=-A+6B \\
					&3=-A+6-6A  &A=\frac{3}{7} \\
					&1=\frac{3}{7}+B &B=\frac{4}{7} \\
					&\bm{a_n=\frac{3}{7}+\frac{24}{7}}
				\end{align*}
				
			\setcounter{enumi}{2}
			
			\item We are doing something similar to (a), where we say $a_n=cr^n$ and $c,r\neq{0}$
				\begin{align*}
					r^2+1=0
				\end{align*}
				\tab From here we are going to split it up to its roots, however the process for finding
				the answer works differently as we are going to have complex roots. This would look like:
				\begin{align*}
					&(r+\sqrt{-1})(r-\sqrt{-1}) &\pm{i} \\
					&a_n=A(i)^n+B(-i)^n \\
					&a_n=A\left(\cos \left(\frac{\pi}{2}\right)+i\sin{\left(\frac{\pi}{2}\right)}\right)^n
					+B\left(\cos \left(\frac{\pi}{2}\right)+i\sin{\left(\frac{-\pi}{2}\right)}\right)^n \\
					&\quad =C\cos{\left(\frac{n\pi}{2}\right)}+D\sin{\left(\frac{n\pi}{2}\right)} \\
					& a_0=0=C\cos{\left(\frac{0\pi}{2}\right)}+D\sin{\left(\frac{0\pi}{2}\right)}=C \\
					& a_1=3=C\cos{\left(\frac{3\pi}{2}\right)}+D\sin{\left(\frac{3\pi}{2}\right)}=D \\
					& \bm{a_n=3\ } \textbf{sin} \bm{\left( \frac{n\pi}{2}\right) ,\ n\ge{0}}
				\end{align*}				
				
			\item Similar to the previous questions, we will say $a_n=cr^n$ where $c,r\neq{0}$.
				The process is a bit different because of repeated real roots, this would look like:
				\begin{align*}
					&r^2-6r+9=0 \\
					&(r-3)(r-3) \\
					&a_n=3^nA+3^nBn \\
					&a_0=5=A &A=5 \\
					&a_1=12=3A+3B \\
					&12=15+3B & B=-1 \\
					&\bm{a_n=(3^n\times{5})-n3^n}
				\end{align*}
		\end{enumerate}
		
	\end{homeworkProblem}

	%%%%%%%%%%%%%%%%%%%%%%%%%%%%%%%%%%%%%%%%%%%%%%%%%%%%%%%%%%%%%%%%%%%%%%%%%%%%%%%%%%
	%                                                                                %
	%                          Section 10.2 Problem 11                               %
	%                                                                                %
	%%%%%%%%%%%%%%%%%%%%%%%%%%%%%%%%%%%%%%%%%%%%%%%%%%%%%%%%%%%%%%%%%%%%%%%%%%%%%%%%%%
	\begin{homeworkProblem}[Problem 11]
		\begin{enumerate}[label=\alph*)]
			\item For $n\ge{1}$, let $a_n$ count the number of binary strings of length $n$,
				where there are no consecutive 1's. Find and solve a recurrence relation for $a_n$.
			\item For $n\ge{1}$, let $b_n$ count the number of binary strings of length $n$, where
				there are no consecutive 1's and the first and last bit of the string are not both 1.
				Find and solve a recurrence relation for $b_n$.
		\end{enumerate}
		
		\textbf{\\Solution:}
		
		\begin{enumerate}[label=\alph*)]
			\item We find that there is a strange phenomenon with the case of 1 and 2 length strings. The one length 
				strings have 2 items that have 3 items without consecutive ones. If you have a string of length 3, it
				will have 5 items, and a string of 4 items will have a length of 8 without consecutive ones. A pattern 
				emerges from this. This pattern being:
				\begin{align*}
					&a_n=a_{n-1}+a_{n-2}, \quad n\ge3 \\
					&0=x^2-x+1
				\end{align*}
				\tab This is a lot like the Fibonacci sequence, similar to the diagram that is shown at 10.16 and 10.17 in the textbook, except we start the sequence at point $F_3$ because $F_3=a_1$ and
				so forth. So we can use this to help answer our question. This could look like:
				\begin{align*}
					&a_n=F_{n+2}=\frac{A^{n+2}-B^{n+2}}{A-B} \\
					&A=\frac{1+\sqrt{5}}{2}, \quad B=\frac{1-\sqrt{5}}{2}
				\end{align*}
				
			\item For problem b) have to consider all the cases where the first and last bits can not be 1. In this case
			$a_1=1$, $a_2=3$, $a_3=4$, $a_4=7$, $a_5=11$, which ends up looking exactly like the last problem:
			\begin{align*}
				a_n=a_{n-1}+a_{n-2}, \qquad n\ge 3
			\end{align*}
			\tab This problem however is that this does not correlate so easily with the Fibonacci sequence, however it still
			does correlate to the sequence in some way. Let's use our previous expression in answer a) and claim our new 
			equation as $b_n$
			\begin{align*}
				&b_n=a_{n-1}+a_{n-3}=F_{n+1}+F_{n-1}
			\end{align*}
			\tab From here we can note the A and B values from previous problem to find that 
			\begin{align*}
				b_n=A^n+B^n
			\end{align*}
		\end{enumerate}
		
	\end{homeworkProblem}

	%%%%%%%%%%%%%%%%%%%%%%%%%%%%%%%%%%%%%%%%%%%%%%%%%%%%%%%%%%%%%%%%%%%%%%%%%%%%%%%%%%
	%                                                                                %
	%                          Section 10.2 Problem 12                               %
	%                                                                                %
	%%%%%%%%%%%%%%%%%%%%%%%%%%%%%%%%%%%%%%%%%%%%%%%%%%%%%%%%%%%%%%%%%%%%%%%%%%%%%%%%%%
	\begin{homeworkProblem}[Problem 12]
		\tab Suppose that poker chips come in four colors —
		red, white, green, and blue. Find and solve a recurrence relation for the number 
		of ways to stack $n$ of these poker chips so that there are no consecutive blue chips.
		
		\textbf{\\Solution:}
		
		\tab To break this down, let's say that $a_n$ is the way to arrange the chips without consecutive blues.
		Now we have to create two separate variables for 2 separate case sceanrios, the first scenario being blue
		where a substack will have a choice of 3 items below it and if it is non-blue it can have a choice of 4 items
		on top of it. We can write these as $b_n$ and $c_n$ respectively and can figure out that the equation for this
		is:
		\begin{align*}
			&a_{n}=b_n+c_n \\
			&c_n=a_n-b_n, \qquad b_n = a_n-c_n \\
			&a_{n+1}=3b_n+4c_n=3(b_n+c_n) + c_n\\
			& \qquad \; =3a_n+3_{n-1}
		\end{align*}
		
		\tab We can then rewrite this equation and find its roots, as such:
		\begin{align*}
			&a_{n+1}-3a_n-3a_{n-1} \\
			&r^2-3r-3=0\\
			&r=\frac{-3\pm\sqrt{21}}{2}
		\end{align*}
		
		\tab Now we have to plug in these values into $A$ and $B$ to find a recurrence equation as so:
		\begin{align*}
			&a_n=A\left(\frac{-3+\sqrt{21}}{2}\right)^n+B\left(\frac{-3-\sqrt{21}}{2}\right)^n \\
			&a_0=1=A\left(\frac{-3+\sqrt{21}}{2}\right)^0+B\left(\frac{-3-\sqrt{21}}{2}\right)^0 \\
			&1=A+B, \qquad B=1-A \\
			&a_1=4=A\left(\frac{-3+\sqrt{21}}{2}\right)+B\left(\frac{-3-\sqrt{21}}{2}\right) \\
			&\quad=A\left(\frac{-3+\sqrt{21}}{2}\right)+(1-A)\left(\frac{-3-\sqrt{21}}{2}\right) \\
			&A=\frac{5+\sqrt{21}}{2\sqrt{21}} \qquad B=\frac{\sqrt{21}-5}{2\sqrt{21}} \\
			&\bm{a_n=\left(\frac{5+\sqrt{21}}{2\sqrt{21}}} \times \frac{-3+\sqrt{21}}{2}\right)+
			\left(\frac{\sqrt{21}-5}{2\sqrt{21}} \times \frac{-3-\sqrt{21}}{2}\right)
		\end{align*}
		
		
	\end{homeworkProblem}
	


\end{document}