\documentclass{article}

\usepackage{fancyhdr}
\usepackage{extramarks}
\usepackage{amsmath}
\usepackage{amsthm}
\usepackage{amsfonts}
\usepackage{tikz}
\usepackage[plain]{algorithm}
\usepackage{algpseudocode}
\usepackage{amssymb}
\usepackage{enumitem}
\usepackage{relsize}
\usepackage{textcomp}
\usepackage{graphicx}
\usepackage{bm}
\usepackage{textcomp}

\usetikzlibrary{automata,positioning}

%
% Basic Document Settings
%

\topmargin=-0.45in
\evensidemargin=0in
\oddsidemargin=0in
\textwidth=6.5in
\textheight=9.0in
\headsep=0.25in

\linespread{1.1}

\pagestyle{fancy}
\lhead{\hmwkAuthorName}
\chead{\hmwkClass\ (\hmwkClassInstructor\ \hmwkClassTime): \hmwkTitle}
\rhead{\hmwkDueDate}
\lfoot{\lastxmark}
\cfoot{\thepage}

\newcommand{\minus}{\scalebox{0.5}[1.0]{$-$}}
\newcommand\tab[1][0.5cm]{\hspace*{#1}}
\renewcommand\headrulewidth{0.4pt}
\renewcommand\footrulewidth{0.4pt}
\renewcommand{\theenumi}{\Alph{enumi}}

\setlength\parindent{0pt}

%
% Create Problem Sections
%

\newcommand{\enterProblemHeader}[1]{
	\nobreak\extramarks{}{{#1} continued on next page\ldots}\nobreak{}
	\nobreak\extramarks{{#1} (continued)}{{#1} continued on next page\ldots}\nobreak{}
}

\newcommand{\exitProblemHeader}[1]{
	\nobreak\extramarks{{#1} (continued)}{{#1} continued on next page\ldots}\nobreak{}
	\stepcounter{homeworkProblemCounter}
	\nobreak\extramarks{{#1}}{}\nobreak{}
}

\newcounter{homeworkProblemCounter}
\setcounter{secnumdepth}{0}
\newcounter{partCounter}
\nobreak\extramarks{Problem \arabic{homeworkProblemCounter}}{}\nobreak{}

%
% Homework Problem Environment
%
% This environment takes an optional argument. When given, it will adjust the
% problem counter. This is useful for when the problems given for your
% assignment aren't sequential. See the last 3 problems of this template for an
% example.
%
\newenvironment{homeworkProblem}[1][-1]{
	\subsection*{{#1}:}
	\enterProblemHeader{{#1}}
	\exitProblemHeader{{#1}}
}

%
% Homework Details
%   - Title
%   - Due date
%   - Class
%   - Section/Time
%   - Instructor
%   - Author
%

\newcommand{\hmwkTitle}{Homework \#2}
\newcommand{\hmwkDueDate}{September 23, 2019}
\newcommand{\hmwkClass}{MTH 5051}
\newcommand{\hmwkClassTime}{Section 01}
\newcommand{\hmwkClassInstructor}{Dr. Jim Jones}
\newcommand{\hmwkAuthorName}{\textbf{Eric Pereira}}

%
% Title Page
%
\pagenumbering{gobble}
\title{
	\vspace{2in}
	\textmd{\textbf{\hmwkClass:\ \hmwkTitle}}\\
	\normalsize\vspace{0.1in}\small{Due\ on\ \hmwkDueDate\ at 07:00pm}\\
	\vspace{0.1in}\large{\textit{\hmwkClassInstructor\ \hmwkClassTime}}
	\vspace{3in}
}

\author{\hmwkAuthorName}
\date{}

\renewcommand{\part}[1]{\textbf{\large Part \Alph{partCounter}}\stepcounter{partCounter}\\}

%
% Various Helper Commands
%

% Useful for algorithms
\newcommand{\alg}[1]{\textsc{\bfseries \footnotesize #1}}

% For derivatives
\newcommand{\deriv}[1]{\frac{\mathrm{d}}{\mathrm{d}x} (#1)}

% For partial derivatives
\newcommand{\pderiv}[2]{\frac{\partial}{\partial #1} (#2)}

% Integral dx
\newcommand{\dx}{\mathrm{d}x}

% Alias for the Solution section header
\newcommand{\solution}{\textbf{\large Solution}}

% Probability commands: Expectation, Variance, Covariance, Bias
\newcommand{\E}{\mathrm{E}}
\newcommand{\Var}{\mathrm{Var}}
\newcommand{\Cov}{\mathrm{Cov}}
\newcommand{\Bias}{\mathrm{Bias}}

\begin{document}
	\maketitle
	
	\pagebreak
	\pagenumbering{arabic}
	
	
	\section*{Pigeonhole Problems Sheet:}
	
	%%%%%%%%%%%%%%%%%%%%%%%%%%%%%%%%%%%%%%%%%%%%%%%%%%%%%%%%%%%%%%%%%%%%%%%%%%%%%%%%%%
	%                                                                                %
	%               Sections Pigenhole Problems Set Problem 1                        %
	%                                                                                %
	%%%%%%%%%%%%%%%%%%%%%%%%%%%%%%%%%%%%%%%%%%%%%%%%%%%%%%%%%%%%%%%%%%%%%%%%%%%%%%%%%%
	\begin{homeworkProblem}[Problem 1]
		A lottery will select 3 balls from a bin containing numbered balls 1,2,...36. Players buy a ticket with 3 of these numbers
		selected. To win the grand prize, the ticket must match the selected balls. A smaller prize is offered if the sum of numbers on the
		ticket match the sum of the selected balls.
		\begin{enumerate}[label=(\alph*)]
			\item How many players must play to guarantee that there is at least one duplicate ticket sold?
			\item How many players must play to guarantee there are at least two tickets sold with the same sum?
			\item How many players must play to guarantee that there are at least three tickets sold with the same sum?
		\end{enumerate}
	
		\textbf{Solution:}
		\begin{enumerate}[label=(\alph*)]
			\item So, to absolutely guarantee you have to sell every ticket possible, and
				add one to it. This means you have sold every 
				possible combination, and one more means that you have every combination,\
				and the next one is repeated from one of the previous combinations. 
				\begin{align*}
					\binom{36}{3} + 1 = \bm{7,141}
				\end{align*}
			\item To determine the total possible sums we can get we have to determine all
			 	possible sums. We can determine all possible sums
				by finding the range of numbers between the lowest and highest sum. We can
				do this by:
				\begin{align*}
					\text{Lowest sum: } 1+2+3=6 \\
					\text{Highest sum: } 36+35+34=105
				\end{align*}
				The range of numbers between 6 and 105 is 100 numbers. If this is every
				possible sum, if we add 1 to this value that means 
				that a pigeonhole will have a second value. So:
				\begin{align*}
					100+1=\bm{101}
				\end{align*}
			\item In order to determine whether 3 tickets are sold we can use some of the 
				information we got from (b). If there are 100 
				pigeonholes (possible sums), lets double it so that there are at least 2 
				in each pigeonhole. In order to get a third we just
				have to add one.
				\begin{align*}
					100 \text{ pigeonholes }\times 2 + 1 = \bm{201}
				\end{align*}
		\end{enumerate}
	\end{homeworkProblem}
	%%%%%%%%%%%%%%%%%%%%%%%%%%%%%%%%%%%%%%%%%%%%%%%%%%%%%%%%%%%%%%%%%%%%%%%%%%%%%%%%%%
	%                                                                                %
	%               Sections Pigenhole Problems Set Problem 2                        %
	%                                                                                %
	%%%%%%%%%%%%%%%%%%%%%%%%%%%%%%%%%%%%%%%%%%%%%%%%%%%%%%%%%%%%%%%%%%%%%%%%%%%%%%%%%%		
			
	
	%\newpage

	\begin{homeworkProblem}[Problem 2]
		(Exercise 7)
		\begin{enumerate}[label=(\alph*)]
			\item Show that if 14 numbers are selected from the set {1, 2, ..., 25}, there are at least two whose sum is 26.
			\item Generalize the results from part (a) and example 5.44
		\end{enumerate}
	
	
		\textbf{Solution:}
		\begin{enumerate}[label=(\alph*)]
			\item There are a total of 12 pairs of numbers within the range of 1-25 that add up to 26. They are:
				\begin{align*}
					[(1,25), (2,24),(3,23),(4,22),(5,21),(6,20),(7,19),(8,18),(9,17),(10,16),(11,15),(12,14),(13)]
				\end{align*}
			
				13 is not in this list, 13 doubled is 26 and we can not choose 13 twice. This means that the absolute maximum
				amount of numbers we could choose that don't add up to 26 is 13, because we can take 1 number from the list of
				pairs and the number 13. Any number number chosen after will correspond to another number that has already been
				picked and that will create a pair of numbers that add up to 26.
			
			\item To generalize, similar to the results in example 5.44 if you have an $n$ size
				list of numbers and I am trying to add up 2 numbers to a value that is $n+1$ 
				a list of numbers equal to $\dfrac{n}{2}$ can make up this list if $n$ is even, and 
				the ceiling of $\dfrac{n}{2}$ if it is odd. As a result you need a minimum of 
				$\dfrac{n}{2} + 1$. 
		\end{enumerate}
	
	
	\end{homeworkProblem}
	%\newpage
	
	
	%%%%%%%%%%%%%%%%%%%%%%%%%%%%%%%%%%%%%%%%%%%%%%%%%%%%%%%%%%%%%%%%%%%%%%%%%%%%%%%%%%
	%                                                                                %
	%               Sections Pigenhole Problems Set Problem 3                        %
	%                                                                                %
	%%%%%%%%%%%%%%%%%%%%%%%%%%%%%%%%%%%%%%%%%%%%%%%%%%%%%%%%%%%%%%%%%%%%%%%%%%%%%%%%%%
	\begin{homeworkProblem}[Problem 3]
		\tab Let ABCD be a square with AB=1. Show that if we select 5 points in the interior, then there are at least two points
		whose distance apart is less than $\dfrac{1}{\sqrt{2}}$\\
	
		\textbf{Solution:}\\
		
		\tab So if we have a square, let us select the 4 furthest possible points from each other, the corners.
		now, from here, the furthest possible point, from any of the other that we can choose within the square
		is the direct center of the square. If each side of the square is 1, then by using pythagorean theorom
		the distance to the center from any point is $\sqrt{\dfrac{1}{2}^2+\dfrac{1}{2}^2}=\dfrac{1}{\sqrt{2}}$.
		Now, because this is the absolute largest possible distance if we are at the exact corners, if we have to
		choose an arbitrary point on the interior of the square it will have to be less than$\dfrac{1}{\sqrt{2}}$, 
		Thus proving that there are at least two points whose distance is less than $\dfrac{1}{\sqrt{2}}$
		
	
	\end{homeworkProblem}
	%\newpage
	
	%%%%%%%%%%%%%%%%%%%%%%%%%%%%%%%%%%%%%%%%%%%%%%%%%%%%%%%%%%%%%%%%%%%%%%%%%%%%%%%%%%
	%                                                                                %
	%               Sections Pigeonhole Problems Set Problem 4                        %
	%                                                                                %
	%%%%%%%%%%%%%%%%%%%%%%%%%%%%%%%%%%%%%%%%%%%%%%%%%%%%%%%%%%%%%%%%%%%%%%%%%%%%%%%%%%
	
	\begin{homeworkProblem}[Problem 4]
		\tab A lossless compression algorithm takes an input file of bits and produces an output 
		file of bits. For the algorithm to be lossless, the input file must be able to be exactly 
		reconstructed from the output file. Show that if the algorithm produces an
		output file of size less than N bits for some specific input file of size N, 
		then there must be some other specific input file
		whose output file is actually larger than its input file.\\
		
		
		\textbf{Solution:}
		
		\tab lets say that the piece of data itself is a one to one pigeonhole situation, where
		n pieces of data in input (pigeons) map to n pieces of data in output (pigeonholes). 
		If there is a  compression algorithm the pigeons have to map to less pigeonholes. In order
		to have data be lossless you need an input file that is larger in order to map n pieces
		of data to n data in output in order to have truly lossless data.  
	\end{homeworkProblem}
	%\newpage
	
	%%%%%%%%%%%%%%%%%%%%%%%%%%%%%%%%%%%%%%%%%%%%%%%%%%%%%%%%%%%%%%%%%%%%%%%%%%%%%%%%%%
	%                                                                                %
	%               Sections Pigeonhole Problems Set Challenge                       %
	%                                                                                %
	%%%%%%%%%%%%%%%%%%%%%%%%%%%%%%%%%%%%%%%%%%%%%%%%%%%%%%%%%%%%%%%%%%%%%%%%%%%%%%%%%%
	
	\begin{homeworkProblem}[Classic Challenge Problem]
		\tab Each point in the x-y plane colored red, blue or green. Prove that there exists a
		 rectangle where the corners have the same color. \\ 
		
		\textbf{Solution:} 
		
		\tab So, let's say we have a single row. In this single row a color has to be repeated
		once at least every 4 cells. There are 6 possible combinations where the colors match
		in this case, lets say (1,2), (1,3), (1,4), (2,3), (2,4), (3,4). Because there are 6 possible
		repeat patterns, and 3 colors a pattern has to repeat every $(6\cdot 3)+1$ times, so every
		4 by 19 rows something has to repeat. Because the x-y plane is infinite there are no restrictions
		to size, so it has to repeat. 
		
	\end{homeworkProblem}
	
	%%%%%%%%%%%%%%%%%%%      
	%	NEW SECTION   %
	%%%%%%%%%%%%%%%%%%%
	\section*{Chapter 4.1}
		
	%%%%%%%%%%%%%%%%%%%%%%%%%%%%%%%%%%%%%%%%%%%%%%%%%%%%%%%%%%%%%%%%%%%%%%%%%%%%%%%%%%
	%                                                                                %
	%                           Section 4.1 Problem 1                                %
	%                                                                                %
	%%%%%%%%%%%%%%%%%%%%%%%%%%%%%%%%%%%%%%%%%%%%%%%%%%%%%%%%%%%%%%%%%%%%%%%%%%%%%%%%%%
	
	\begin{homeworkProblem}[Problem 1]
		\tab Prove each of the following for all $n \ge 1$ by the Principle of Mathematical Induction.
		\begin{enumerate}[label=(\alph*)]
			\item $1^2+3^2+5^2+...+(2n-1)^2=\dfrac{n(2n-1)(2n+1)}{3}$
			\item $1\cdot{3}+2\cdot{4}+3\cdot{5}+...+n(n+2)=\dfrac{n(n+1)(2n+7)}{6}$
		\end{enumerate}
	
		\textbf{Solution:}
		
		\begin{enumerate}[label=(\alph*)]
			
			\item The first step to induction is to provide an basis case, the example case can be 1,
				that can be seen here:
				\begin{align*}
					1^2 = \dfrac{1(1)(3)}{3}
				\end{align*}
				\tab This is true. So now we can move on. Now let\textquotesingle s make an assumption, 
				let\textquotesingle s assume:
				\begin{align*}
					S(k):1^2+3^2+5^2+...+(2k-1)^2=\dfrac{k(2k-1)(2k+1)}{3} \text{ for some } k \ge 1
				\end{align*}
				\tab From here lets do the inductive step and trying to prove this for $k+1$. We are trying to prove:
				\begin{align*}
					S(k+1):1^2+3^2+5^2+...+(2k-1)^2+(2k+1)^2&=\dfrac{(k+1)(2(k+1)-1)(2(k+1)+1)}{3} \\
															&=\dfrac{(k+1)(2k+1)(2k+3)}{3}
				\end{align*}
				\tab So from our assumption:
				\begin{align*}
					S(k+1):\qquad &[1^2+3^2+5^2+...+(2k-1)^2] + (2k+1)^2 \\
					&\dfrac{k(2k-1)(2k+1)}{3} + (2k+1)^2 \\
					&\dfrac{k(2k-1)(2k+1)}{3} + \dfrac{3(2k+1)^2}{3} \\
					&\dfrac{(2k+1)[k(2k-1)+3(2k+1)]}{3} \\
					&\dfrac{(2k+1)(2k^2+5k+3)}{3} \\
					&\dfrac{(k+1)(2k+1)(2k+3)}{3}
				\end{align*}
				\tab Since $k$ is an arbitrary value its true for any case $k\ge{1}$, proving
				\begin{align*}
					1^2+3^2+5^2+...+(2n-1)^2=\dfrac{n(2n-1)(2n+1)}{3}
				\end{align*}
			
			\item The first step to induction is to provide an basis case, lets try 1 in this case. 
				\begin{align*}
					1\cdot 3=\dfrac{1(2)(9)}{6}
				\end{align*}
				\tab This is true, so now we can move on. Now let\textquotesingle s make an assumption, 
				let\textquotesingle s assume:
				\begin{align*}
					S(k):1\cdot{3}+2\cdot{4}+3\cdot{5}+...+k(k+2)=\dfrac{k(k+1)(2k+7)}{6} \text{ for some } k\ge 1
				\end{align*}
				\tab From here lets do the inductive step and trying to prove this for $k+1$. We are trying to prove:
				\begin{align*}
					S(k+1): 1\cdot{3}+2\cdot{4}+3\cdot{5}+...+k(k+2)+(k+1)((k+1)+2)&=\dfrac{(k+1)((k+1)+1)(2(k+1)+7)}{6} \\
					                                                               &=\dfrac{(k+1)((k+2)(2k+9)}{6}
				\end{align*}
				\tab So from our assumption:
				\begin{align*}
					S(k+1):\qquad &[1\cdot{3}+2\cdot{4}+3\cdot{5}+...+k(k+2)]+(k+1)((k+1)+2) \\
					&\dfrac{k(k+1)(2k+7)}{6}+(k+1)(k+3) \\
					&\dfrac{k(k+1)(2k+7)}{6}+\dfrac{6(k+1)(k+3)}{6} \\
					&\dfrac{(k+1)[k(2k+7)+6(k+3)]}{6} \\
					&\dfrac{(k+1)[2k^2+13k+18]}{6} \\
					&\dfrac{(k+1)((k+2)(2k+9)}{6}
				\end{align*}
				\tab Since $k$ is an arbitrary value its true for any case $k\ge{1}$, proving
				\begin{align*}
					1\cdot{3}+2\cdot{4}+3\cdot{5}+...+n(n+2)=\dfrac{n(n+1)(2n+7)}{6}
				\end{align*}
				
		\end{enumerate}
	\end{homeworkProblem}
	%\newpage

	%%%%%%%%%%%%%%%%%%%%%%%%%%%%%%%%%%%%%%%%%%%%%%%%%%%%%%%%%%%%%%%%%%%%%%%%%%%%%%%%%%
	%                                                                                %
	%                           Section 4.1 Problem 2                                %
	%                                                                                %
	%%%%%%%%%%%%%%%%%%%%%%%%%%%%%%%%%%%%%%%%%%%%%%%%%%%%%%%%%%%%%%%%%%%%%%%%%%%%%%%%%%
	
	\begin{homeworkProblem}[Problem 2]
		\tab Prove each of the following for all $n \ge 1$ by the Principle of Mathematical Induction.
		\begin{enumerate}[label=(\alph*)]
			\setcounter{enumi}{1}
			\item 
			$$\sum_{i=1}^{n} i(2^i)=2+(n-1)2^{n+1}$$
			\item 
			$$\sum_{i=1}^{n} (i)(i!)=(n+1)!-1$$
		\end{enumerate} 
	
		\textbf{Solution:}
		
		\begin{enumerate}[label=(\alph*)]
			\setcounter{enumi}{1}
			\item The first step to induction is the basis case. let\textquotesingle s use 1 in this case. 
				\begin{align*}
					(1)(2^1)=2+(0)2^{1+1}
				\end{align*}
				\tab This is true, so now we can move on. Now let\textquotesingle s make an assumption, 
				let\textquotesingle s assume:
				\begin{align*}
					S(k):\sum_{i=1}^{k} i(2^i)=2+(k-1)2^{k+1} \text{ for some } k\ge{1}
				\end{align*}
				\tab From here lets do the inductive step and trying to prove this for $k+1$. We are trying to prove:
				\begin{align*}
					S(k+1):\sum_{i=1}^{k+1} i(2^i)&=2+((k+1)-1)2^{(k+1)+1} \\
				                            &=2+(k)2^{k+2}
				\end{align*}
				\tab So from our assumption:
				\begin{align*}
					S(k+1):\qquad & \sum_{i=1}^{k+1} i(2^i) \\
					&\sum_{i=1}^{k} i(2^i) + (k+1)2^{k+1} \\
					&[2+(k-1)2^{k+1}] + (k+1)2^{k+1} \\
					&2+(2^{k+1})[(k-1)+(k+1)] \\
					&2+(2^{k+1})(2k) \\
					&2+(k)(2^{k+2}) 
				\end{align*}
				\tab Since $k$ is an arbitrary value its true for any case $k\ge{1}$, proving
				\begin{align*}
					\sum_{i=1}^{n} i(2^i)=2+(n-1)2^{n+1}
				\end{align*}
			\item The first step to induction is the basis case. let\textquotesingle s use 1 in this case. 
				\begin{align*}
					(1)(1!)=(1+1)!-1
				\end{align*}
				\tab This is true, so now we can move on. Now let\textquotesingle s make an assumption, let\textquotesingle s assume:
				\begin{align*}
					S(k): \sum_{i=1}^{k} (i)(i!)=(k+1)!-1 \text{ for some } k\ge 1
				\end{align*}
				\tab From here lets do the inductive step and trying to prove this for $k+1$. We are trying to prove:
				\begin{align*}
					S(k+1):\sum_{i=1}^{k+1} (i)(i!)&=((k+1)+1)!-1 \\
					                               &=(k+2)!-1
				\end{align*}
				\tab So from our assumption:
				\begin{align*}
					S(k+1):\qquad & \sum_{i=1}^{k+1} (i)(i!) \\
					              &= \sum_{i=1}^{k} (i)(i!) + (k+1)((k+1)!) \\  
					              &= (k+1)!-1 + (k+1)((k+1)!) \\
					              &= (k+1)!(1+(k+1))-1 \\
					              &= (k+1)!(k+2)-1 \\
					              &= (k+2)!-1
				\end{align*}
				\tab Since $k$ is an arbitrary value its true for any case $k\ge{1}$, proving
				\begin{align*}
					\sum_{i=1}^{n} i(2^i)=2+(n-1)2^{n+1}
				\end{align*}
		\end{enumerate}
	\end{homeworkProblem}
	%\newpage
	
	%%%%%%%%%%%%%%%%%%%%%%%%%%%%%%%%%%%%%%%%%%%%%%%%%%%%%%%%%%%%%%%%%%%%%%%%%%%%%%%%%%
	%                                                                                %
	%                           Section 4.1 Problem 4                                %
	%                                                                                %
	%%%%%%%%%%%%%%%%%%%%%%%%%%%%%%%%%%%%%%%%%%%%%%%%%%%%%%%%%%%%%%%%%%%%%%%%%%%%%%%%%%
	
	\begin{homeworkProblem}[Problem 4]
		\tab A wheel of fortune has the integers from 1 to 25 placed on it in a random manner. Show that regardless of how the numbers are 
		positioned on the wheel, there are three adjacent numbers whose sum is at least 39. \\
		
		\textbf{Solution:} 
		
		\tab Let\textquotesingle s assume that: $x_1+x_2+x_3<39,x_2+x_3+x_4,...,x_{24}+x_{25}+x_1<39$ for all possible
		combinations of possible numbers on the wheel. For this to be true that would mean:
		\begin{align*}
			\sum_{i=1}^{25}3x_i < 25(39)
		\end{align*}
		\tab But the problem is if you do the math:
		\begin{align*}
			\sum_{i=1}^{25}3x_i &= 975 \\
			25\cdot 39 &= 975
		\end{align*}
		\tab So because of this, there is at least one combination of three adjacent numbers that has a value that adds up to at least 39.
	\end{homeworkProblem}
	%\newpage
	
	%%%%%%%%%%%%%%%%%%%%%%%%%%%%%%%%%%%%%%%%%%%%%%%%%%%%%%%%%%%%%%%%%%%%%%%%%%%%%%%%%%
	%                                                                                %
	%                           Section 4.1 Problem 7                                %
	%                                                                                %
	%%%%%%%%%%%%%%%%%%%%%%%%%%%%%%%%%%%%%%%%%%%%%%%%%%%%%%%%%%%%%%%%%%%%%%%%%%%%%%%%%%
	
	\begin{homeworkProblem}[Problem 7]
		\tab A lumberjack has $4n + 110$ logs in a pile consisting of $n$ layers. Each layer has two more logs than the layer directly above
		it. If the top layer has six logs, how many layers are there? \\
		
		\textbf{Solution:} 
		
		\tab If the top layer has 6 logs, and each layer below has exactly 2 more then we can create an equation for this.
		We essentially have 2 equations, and if we set them equal to each other we will be able to find out how many $n$ 
		layers there are. This can be described by:
		\begin{align*}
			4n+110&=6+8+10+...+(6 + 2(n-1)) \\
			&=6n + 2\left((n-1)\left(\dfrac{n}{2}\right)\right)\\
			&=6n+n^2-n\\
			&=n^2+5n\\
		\end{align*}
		\tab From here we are able to actually simply solve for n. 
		\begin{align*}
			n^2+5n&=4n+110 \\
			n^2+n&=110 \\
			n&=\bm{10}
		\end{align*}
		
		
		
		
	\end{homeworkProblem}
	%\newpage

	%%%%%%%%%%%%%%%%%%%%%%%%%%%%%%%%%%%%%%%%%%%%%%%%%%%%%%%%%%%%%%%%%%%%%%%%%%%%%%%%%%
	%                                                                                %
	%                          Section 4.1 Problem 14                                %
	%                                                                                %
	%%%%%%%%%%%%%%%%%%%%%%%%%%%%%%%%%%%%%%%%%%%%%%%%%%%%%%%%%%%%%%%%%%%%%%%%%%%%%%%%%%
	
	\begin{homeworkProblem}[Problem 14]
		\tab Prove that for all $n \in \mathbb{Z}^+, n > 3 \implies 2^n < n!$ \\
		
		\textbf{Solution:}
		
		\tab The first step to induction is the basis case. let\textquotesingle s use 4 in
		this case. 
		\begin{align*}
			2^4 < 4!
		\end{align*}
		\tab This is true, so now we can move on. Now let\textquotesingle s make an
		assumption, let\textquotesingle s assume:
		\begin{align*}
			S(k): 2^k < k! \text{ for } k > 3
		\end{align*}
		\tab From here lets do the inductive step and trying to prove this for 
		$k+1$. We are trying to prove:
		\begin{align*}
			S(k+1): 2^{k+1} <(k+1)! 
		\end{align*}
		\tab So from our assumption:
		\begin{align*}
			S(k+1)&= 2^{k+1} < (k+1)!\\
			&=2^k \cdot 2 < k! \cdot (k+1)\\
			&=k! \cdot 2 < k! \cdot (k+1)\\
			&= 2 < k+1
		\end{align*}
		\tab Since $k$ is an arbitrary value its true for any case $k>{3}$,
		proving
		\begin{align*}
			n \in \mathbb{Z}^+, n > 3 \implies 2^n < n!
		\end{align*}
	\end{homeworkProblem} 
	%\newpage
	
	%%%%%%%%%%%%%%%%%%%%%%%%%%%%%%%%%%%%%%%%%%%%%%%%%%%%%%%%%%%%%%%%%%%%%%%%%%%%%%%%%%
	%                                                                                %
	%                          Section 4.1 Problem 15                                %
	%                                                                                %
	%%%%%%%%%%%%%%%%%%%%%%%%%%%%%%%%%%%%%%%%%%%%%%%%%%%%%%%%%%%%%%%%%%%%%%%%%%%%%%%%%%
	
	\begin{homeworkProblem}[Problem 15]
		\tab Prove that for all $n \in \mathbb{Z}^+, n > 4 \implies n^2 < 2^n$ \\
		
		\textbf{Solution:}
		
		\tab The first step to induction is the basis case. let\textquotesingle s use 4 in
		this case. 
		\begin{align*}
			2^4 < 4!
		\end{align*}
		\tab This is true, so now we can move on. Now let\textquotesingle s make an
		assumption, let\textquotesingle s assume:
		\begin{align*}
			S(k): 2^k < k! \text{ for } k > 3
		\end{align*}
		\tab From here lets do the inductive step and trying to prove this for 
		$k+1$. We are trying to prove:
		\begin{align*}
			S(k+1): 2^{k+1} <(k+1)! 
		\end{align*}
		\tab So from our assumption:
		\begin{align*}
			S(k+1)&= 2^{k+1} < (k+1)!\\
			&=2^k \cdot 2 < k! \cdot (k+1)\\
			&=k! \cdot 2 < k! \cdot (k+1) \\
			&= 2 < k+1
		\end{align*}
		\tab Since $k$ is an arbitrary value its true for any case $k>{3}$,
		proving
		\begin{align*}
			n \in \mathbb{Z}^+, n > 3 \implies 2^n < n!
		\end{align*}
	\end{homeworkProblem} 
	%\newpage
	
	%%%%%%%%%%%%%%%%%%%%%%%%%%%%%%%%%%%%%%%%%%%%%%%%%%%%%%%%%%%%%%%%%%%%%%%%%%%%%%%%%%
	%                                                                                %
	%                          Section 4.1 Problem 19                                %
	%                                                                                %
	%%%%%%%%%%%%%%%%%%%%%%%%%%%%%%%%%%%%%%%%%%%%%%%%%%%%%%%%%%%%%%%%%%%%%%%%%%%%%%%%%%
	
	\begin{homeworkProblem}[Problem 19]
		\tab For $n \in \mathbb{Z}^+$ let $S(n)$ be the open statement
		\begin{align*}
			\sum_{i=1}^{n}i=\dfrac{\left(n+\left(\frac{1}{2}\right)\right)^2}{2}
		\end{align*}
		\tab Show the truth of $S(k)$ implies the truth of $S(k+1)$ for all $k \in \mathbb{Z}^+$.
		Is $S(n)$ true for all $n \in \mathbb{Z}^+$? \\
		
		\textbf{Solution:}
		
		\tab The first step to induction is the basis case. let\textquotesingle s use 1
		\begin{align*}
			&n=0:\quad 1=\frac{\left(1+\frac{1}{2}\right)^2}{2}
		\end{align*}
		\tab This is false, which proves the statement false for any $k\ge{1}$, however I would
		like to see if I can show that it is true inductively with $k$ and $k+1$
		\begin{align*}
			S(k): \sum_{i=1}^{k}i=\frac{\left(k+\left(\frac{1}{2}\right)\right)^2}{2} 
				\text{ for all } k\in \mathbb{Z}^+
		\end{align*}
		\tab From here lets do the inductive step and trying to prove this for 
			$k+1$. We are trying to prove:
		\begin{align*}
			S(k+1):  \sum_{i=1}^{k+1}i=\frac{\left((k+1)+\left(\frac{1}{2}\right)\right)^2}{2} 
		\end{align*}
		\tab So from our assumption:
		\begin{align*}
			S(k+1)&=\sum_{i=1}^{k+1}i \\
			      &=\sum_{i=1}^{k}i + (k+1) \\
			      &=\frac{\left(k+\left(\frac{1}{2}\right)\right)^2}{2}  + (k+1) \\
			      &=\frac{\left(k+\left(\frac{1}{2}\right)\right)^2+2k+2}{2} \\
			      &=\frac{k^2+k+\frac{1}{4}+2k+2}{2} \\
			      &=\frac{k^2+k+\frac{1}{4}+2k+2}{2} \\
			      &=\frac{(k+1)^2+(k+1)+\frac{1}{4}}{2} \\
			      &=\frac{(k+1)^2+(k+1)+\frac{1}{4}}{2} \\
			      &=\frac{\left((k+1)+\left(\frac{1}{2}\right)\right)}{2}
		\end{align*}
		\tab Since $k$ is an arbitrary value its true for any case $k\in \mathbb{Z}^+$ this
		passes the inductive part of the proof. 
	\end{homeworkProblem} 
	%\newpage
	
	%%%%%%%%%%%%%%%%%%%      
	%	NEW SECTION   %
	%%%%%%%%%%%%%%%%%%%
	\section*{Chapter 4.2}
	
	%%%%%%%%%%%%%%%%%%%%%%%%%%%%%%%%%%%%%%%%%%%%%%%%%%%%%%%%%%%%%%%%%%%%%%%%%%%%%%%%%%
	%                                                                                %
	%                          Section 4.2 Problem 1                                 %
	%                                                                                %
	%%%%%%%%%%%%%%%%%%%%%%%%%%%%%%%%%%%%%%%%%%%%%%%%%%%%%%%%%%%%%%%%%%%%%%%%%%%%%%%%%%
	
	\begin{homeworkProblem}[Problem 1]
		\tab The integer sequence $a_1, a_2, a_3,...,$ defined explicitly by the formula $a_n=5n$ 
		for $n\in \mathbb{Z}^+$, can also be defined
		recursively by
		\begin{enumerate}[label=\arabic*)]
			\item $a_1=5$; and
			\item $a_{n+1} = a_n + 5$ for $n\ge 1$.
		\end{enumerate}
	
		\tab For an integer sequence $b_1,b_2,b_3,...,$ where $b=n(n+2)$ for all $n \in \mathbb{Z}^+$, we can also provide the recursive 
		definition:
		\begin{enumerate}[label=\arabic*)\textquotesingle]
			\item $b_1=3$; and
			\item $b_{n+1}=b_n+2n+3,$ for $n\ge{1}$.
		\end{enumerate}
		\tab Give a recursive definition for each of the following integer sequences $c_1,c_2,c_3,...,$ where $n\in{\mathbb{Z}^+}$ we have
		\begin{enumerate}[label=(\alph*)]
			\item $c_n=7n$
			\item $c_n=7^n$
			\item $c_n=3n+7$
			\item $c_n=7$
			\item $c_n=n^2$
			\item $c_n=2-(-1)^n$
		\end{enumerate}
		
		\textbf{Solution:}
		
		\begin{enumerate}[label=(\alph*)]
			\item $\bm{c_1=7;}$ and $\bm{c_{n+1}=c_n+7}$, for $n\ge{1}$
			\item $\bm{c_1=7;}$ and $\bm{c_{n+1}\cdot 7}$, for $n\ge{1}$
			\item $\bm{c_1=10;}$ and $\bm{c_{n+1}\cdot 3 + 7}$, for $n\ge{1}$
			\item $\bm{c_1=7;}$ and $\bm{c_{n+1}=c_n}$, for $n\ge{1}$
			\item $\bm{c_1=1;}$ and $\bm{c_{n+1}=c_n+2n+1}$, for $n\ge{1}$
			\item $\bm{c_1=3,c_2=1;}$  and $\bm{c_{n+2}=c_n}$, for $n\ge{1}$
		\end{enumerate}
	\end{homeworkProblem} 
	%\newpage
	
	%%%%%%%%%%%%%%%%%%%%%%%%%%%%%%%%%%%%%%%%%%%%%%%%%%%%%%%%%%%%%%%%%%%%%%%%%%%%%%%%%%
	%                                                                                %
	%                          Section 4.2 Problem 11                                %
	%                                                                                %
	%%%%%%%%%%%%%%%%%%%%%%%%%%%%%%%%%%%%%%%%%%%%%%%%%%%%%%%%%%%%%%%%%%%%%%%%%%%%%%%%%%
	
	\begin{homeworkProblem}[Problem 11]
		Define the integer sequence $a_0, a_1, a_2,a_3,...,$ recursively by
		\begin{enumerate}[label=\arabic*)]
			\item $a_0=1, a_1=1, a_2=1$; and
			\item For $n\ge 3, a_n=a_{n-1}+a_{n-3}$
		\end{enumerate}
		Prove that $a_{n+2}\ge (\sqrt{2})^n$ for all $n\ge 0$ \\

		
		\textbf{Solution:}
		
		\tab The first step to induction is the basis case. let\textquotesingle s use 0, 1,
		and 2 this case. 
		\begin{align*}
			&n=0:\quad 1\ge\sqrt{2}^0 \\
			&n=1:\quad 2\ge\sqrt{2}^1 \\
			&n=2:\quad 3\ge\sqrt{2}^2 
		\end{align*}
		\tab These are true so now we can move on. Now let\textquotesingle s make an
		assumption, let\textquotesingle s assume:
		\begin{align*}
			S(k): a_{k+2}\ge (\sqrt{2})^k \text{ for all } k>2
		\end{align*}
		\tab From here lets do the inductive step and trying to prove this for 
			$k+1$. We are trying to prove:
		\begin{align*}
			S(k+1): a_{k+3}\ge (\sqrt{2})^{k+1} 
		\end{align*}
		\tab So from our assumption:
		\begin{align*}
			S(k+1)&=a_{k+3} \\
				  &= a_{k}+a_{k+2} \ge \sqrt{2}^{k-2} \sqrt{2}^{k} \\
				  &= a_{k}+a_{k+2} \ge \left(\sqrt{2}^{2}+1 \right) \sqrt{2}^{k-2}  \\
				  &= a_{k}+a_{k+2} \ge  \left(3\sqrt{2}^{-2}\right) \left(\sqrt{2}^{k}\right)  \\
				  &= a_{k}+a_{k+2} \ge \frac{3\sqrt{2}^{k}}{2}  \\
				  &= a_{k}+a_{k+2} \ge \frac{3\sqrt{2}^{k}}{2}  \\
				  &= a_{k}+a_{k+2} \ge \frac{3}{2} \sqrt{2}^{k} \ge \sqrt{2} \sqrt{2}^{k}\\
				  &= a_{k}+a_{k+2} \ge \frac{3}{2} \sqrt{2}^{k} \ge \sqrt{2}^{k+1}\\
		\end{align*}
		\tab Since $k$ is an arbitrary value its true for any case $k>{2}$ and the
		first three basis cases, proving
		\begin{align*}
			a_{n+2}\ge (\sqrt{2})^n \text{ for all } n\ge 0
		\end{align*}		
			
	\end{homeworkProblem} 
	%\newpage
	
	%%%%%%%%%%%%%%%%%%%%%%%%%%%%%%%%%%%%%%%%%%%%%%%%%%%%%%%%%%%%%%%%%%%%%%%%%%%%%%%%%%
	%                                                                                %
	%                          Section 4.2 Problem 13                                %
	%                                                                                %
	%%%%%%%%%%%%%%%%%%%%%%%%%%%%%%%%%%%%%%%%%%%%%%%%%%%%%%%%%%%%%%%%%%%%%%%%%%%%%%%%%%
	
	\begin{homeworkProblem}[Problem 13]
		Prove that for any positive integer $n$,
		\begin{align*}
			\sum_{i=1}^{n}\frac{F_{i-1}}{2^i}=1-\frac{F_{n+2}}{2^n}.
		\end{align*}
		
		
		\textbf{Solution:}

		\tab The first step to induction is the basis case. let\textquotesingle s use 1 in
		this case. 
		\begin{align*}
			\frac{F_{0}}{2^i}=1-\frac{F_{2}}{2^n}
		\end{align*}
		\tab This is true, so now we can move on. Now let\textquotesingle s make an
		assumption, let\textquotesingle s assume:
		\begin{align*}
			S(k):& \sum_{i=1}^{k}\frac{F_{i-1}}{2^i}=1-\frac{F_{k+2}}{2^k}
			\text{ for all } k\in{\mathbb{Z}^+}
		\end{align*}
		\tab From here lets do the inductive step and trying to prove this for 
		$k+1$. We are trying to prove:
		\begin{align*}
			S(k+1):& \sum_{i=1}^{k+1}\frac{F_{i-1}}{2^i}=1-\frac{F_{k+3}}{2^{k+1}}. 
		\end{align*}
		\tab So from our assumption:
		\begin{align*}
			S(k+1)&= \sum_{i=1}^{k+1}\frac{F_{i-1}}{2^i} \\
			      &= \sum_{i=1}^{k}\frac{F_{i-1}}{2^i} + \frac{F_k}{2^{k+1}} \\
			      &= 1-\frac{F_{k+2}}{2^k}+\frac{F_k}{2^{k+1}} \\
			      &= 1+\left(\frac{1}{2^{k+1}}\right)(F_k-2F_{k+2}) \\
			      &= 1+\left(\frac{1}{2^{k+1}}\right)((F_k-F_{k+2})-F_{k+2}) \\
			      &= 1+\left(\frac{1}{2^{k+1}}\right)(-F_{k+1}-F_{k+2}) \\
			      &= 1+\left(\frac{1}{2^{k+1}}\right)(-F_{k+1}-F_{k+2}) \\
			      &= 1-\left(\frac{1}{2^{k+1}}\right)(F_{k+1}+F_{k+2}) \\
			      &= 1-\left(\frac{F_{k+3}}{2^{k+1}}\right) 
		\end{align*}
		\tab Since $k$ is an arbitrary value its true for any case $k\in \mathbb{Z}^+$,
		proving
		\begin{align*}
			\sum_{i=1}^{n}\frac{F_{i-1}}{2^i}=1-\frac{F_{n+2}}{2^n}.
		\end{align*}		
	\end{homeworkProblem} 
	%\newpage
\end{document}