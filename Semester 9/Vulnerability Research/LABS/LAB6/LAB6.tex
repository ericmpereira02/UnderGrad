\documentclass[10.9pt]{article}
\newcommand\tab[1][0.5cm]{\hspace*{#1}}
\usepackage[utf8]{inputenc}
\usepackage{listings}
\usepackage{hyperref}
\usepackage{color}
\pagenumbering{gobble}
\usepackage{changepage}
\usepackage{titletoc}
\usepackage{titlesec}
\usepackage{multicol}
\usepackage{graphicx}

\usepackage{makecell}

\definecolor{codegreen}{rgb}{0,0.6,0}
\definecolor{codegray}{rgb}{0.5,0.5,0.5}
\definecolor{codepurple}{rgb}{0.58,0,0.82}
\definecolor{backcolour}{rgb}{0.95,0.95,0.92}

\hypersetup{
	colorlinks,
	citecolor=black,
	filecolor=black,
	linkcolor=black,
	urlcolor=black
}

\lstdefinestyle{mystyle}{
	backgroundcolor=\color{backcolour},   
	commentstyle=\color{codegreen},
	keywordstyle=\color{magenta},
	numberstyle=\tiny\color{codegray},
	stringstyle=\color{codepurple},
	basicstyle=\footnotesize\ttfamily,
	breakatwhitespace=false,         
	breaklines=true,                 
	captionpos=b,                    
	keepspaces=true,                 
	numbers=left,                    
	numbersep=5pt,                  
	showspaces=false,                
	showstringspaces=false,
	showtabs=false,                  
	tabsize=2
}

\lstset{basicstyle=\tiny,style=mystyle}

\newcommand{\titledate}[2][2.5in]{%
	\noindent%
	\begin{tabular}{@{}p{#1}@{}}
		\\ \hline \\[-.75\normalbaselineskip]
		#2
	\end{tabular} \hspace{1in}
	\begin{tabular}{@{}p{#1}@{}}
		\\ \hline \\[-.75\normalbaselineskip]
		Date
	\end{tabular}
}

\titleformat{\section}{\normalfont\Large\bfseries}{}{0pt}{}

% for forcing tables to fit
\usepackage{changepage}

\begin{document}
	

\begin{titlepage}
	
\author{Eric Pereira}
\date{Septemer 24\textsuperscript{th}, 2019}
\title{CSE4501 -- Vulnerability Research:\\Lab 6}

\maketitle

\end{titlepage}

\titlecontents{section}[0em]
{\vskip 0.5ex}%
{\scshape}% numbered sections formattin
{\itshape}% unnumbered sections formatting
{}%

\tableofcontents

\newpage \pagenumbering{arabic}

Use the necessary tools to perform analysis of the compiled source code generated. Since we are working in 32-bit use ``-m32" for your gcc compilation options. For compilation I would suggest using compiler option``\texttt{-fno-stack-protector}" and ``\texttt{-z execstack}". Please submit write-ups of your analysis. Your points will be based on completeness of analysis, your understanding of the problems, and if you were able to achieve the goals. Each problem is worth 5 pts for a total of 25pts.


%%%%%%%%%%%%%%%%%%%%%%%%%%%%%%%%%%%%%%%%%%%%%%%%%%%%%%%%%%%%%%%%%%%%%%%%%%%%%%%%%%
%                                                                                %
%                                   Problem 1                                    %
%                                                                                %
%%%%%%%%%%%%%%%%%%%%%%%%%%%%%%%%%%%%%%%%%%%%%%%%%%%%%%%%%%%%%%%%%%%%%%%%%%%%%%%%%%

\section{Problem 1}
\tab Cause the CFD.
\begin{lstlisting}[language=C]
#include<stdlib.h>
#include<unistd.h>
#include<stdio.h>
#include<string.h>

int target;

void vuln(char *string){
	printf(string);
	
	if(target){
		printf("you have modified the target :)\n");
	}
}

int main(int argc, char **argv){
	vuln(argv[1]);
}
\end{lstlisting}
\textbf{\\Solution:\\} 
\tab 

%%%%%%%%%%%%%%%%%%%%%%%%%%%%%%%%%%%%%%%%%%%%%%%%%%%%%%%%%%%%%%%%%%%%%%%%%%%%%%%%%%
%                                                                                %
%                                   Problem 2                                    %
%                                                                                %
%%%%%%%%%%%%%%%%%%%%%%%%%%%%%%%%%%%%%%%%%%%%%%%%%%%%%%%%%%%%%%%%%%%%%%%%%%%%%%%%%%
\section{Problem 2}
\tab Set target to the specific value to win.


\begin{lstlisting}[language=C]
#include<stdlib.h>
#include<unistd.h>
#include<stdio.h>
#include<string.h>

int target;

void vuln() {
	char buffer[512];
	
	fgets(buffer, sizeof(buffer), stdin);
	printf(buffer);
	
	if (target == 64){
		printf("You have modified the target :)\n");
	} else {
		printf("target is %d :(\n", %target);
	}

}

int main(){
	vuln();
}
\end{lstlisting}
\textbf{\\Solution:\\}


%%%%%%%%%%%%%%%%%%%%%%%%%%%%%%%%%%%%%%%%%%%%%%%%%%%%%%%%%%%%%%%%%%%%%%%%%%%%%%%%%%
%                                                                                %
%                                   Problem 3                                    %
%                                                                                %
%%%%%%%%%%%%%%%%%%%%%%%%%%%%%%%%%%%%%%%%%%%%%%%%%%%%%%%%%%%%%%%%%%%%%%%%%%%%%%%%%%
\section{Problem 3}
\tab Set target to specific value to win.
\begin{lstlisting}[language=C]
#include <stdlib.h>
#include <unistd.h>
#include<stdio.h>
#include<string.h>

int target;

void printbuffer(char *string){
	printf(string);
}

void vuln(){
	char buffer[512];
	
	fgets(buffer,sizeof(buffer), stdin);
	
	printbuffer(buffer);
	
	if(target==0x01025544){
		printf("you have modified the target :)\n");
	} else {
		printf("target is %08x :(\n", target);
	}
}

int main(int argc, char **argv){
	vuln();
}

void main() {
	char *name[2];

	name[0] = "/bin/sh";
	name[1] = NULL;
	execve(name[0], name, NULL);
}
\end{lstlisting}
\textbf{\\Solution:\\}


%%%%%%%%%%%%%%%%%%%%%%%%%%%%%%%%%%%%%%%%%%%%%%%%%%%%%%%%%%%%%%%%%%%%%%%%%%%%%%%%%%
%                                                                                %
%                                   Problem 4                                    %
%                                                                                %
%%%%%%%%%%%%%%%%%%%%%%%%%%%%%%%%%%%%%%%%%%%%%%%%%%%%%%%%%%%%%%%%%%%%%%%%%%%%%%%%%%
\section{Problem 4}
\tab Compile the below program; open in debugger. Examine the different shellcodes; What are
each of them doing?
\begin{lstlisting}[language=C]
#include <stdlib.h>
#include <unistd.h>
#include <stdio.h>
#include <string.h>

int target;

void win(){
	printf("Code execution redirected! you win\n");
}

void vuln(){
	char buffer[512];
	
	fgets(buffer, sizeof(buffer), stdin);
	
	printf(buffer);

	exit(1);
}

int main(int argc, char **argv){
	vuln();
}
\end{lstlisting}
\textbf{\\Solution:\\}


%%%%%%%%%%%%%%%%%%%%%%%%%%%%%%%%%%%%%%%%%%%%%%%%%%%%%%%%%%%%%%%%%%%%%%%%%%%%%%%%%%
%                                                                                %
%                                   Problem 5                                    %
%                                                                                %
%%%%%%%%%%%%%%%%%%%%%%%%%%%%%%%%%%%%%%%%%%%%%%%%%%%%%%%%%%%%%%%%%%%%%%%%%%%%%%%%%%
\section{Problem 5}
\tab Compile the below program; Cause a control flow deviation to “win”
\begin{lstlisting}[language=C]
#include <stdlib.h>
#include <unistd.h>
#include <stdio.h>
#include <string.h>

int target;

void GetAddr(char* Name){Z
	printf("%s is at %p\n", Name, getenv(Name));
}

void vuln()
{
	char buffer[512];
	
	fgets(buffer, sizeof(buffer), stdin);
	
	printf(buffer);
	
	exit(1);	
}

int main(int argc, char** argv){
	GetAddr(argv[1]);
	vuln();
}
\end{lstlisting}
\textbf{\\Solution:\\}

\end{document}
