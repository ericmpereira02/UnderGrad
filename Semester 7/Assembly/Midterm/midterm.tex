\documentclass[12pt]{article}
\newcommand\tab[1][1cm]{\hspace*{#1}}
\usepackage[utf8]{inputenc}
\usepackage{listings}
\usepackage{amsmath}
\usepackage{enumerate}
\usepackage{array}
\usepackage{hyperref}
\usepackage{color}
\newcolumntype{C}{>$c<$}
\pagenumbering{gobble}

\definecolor{codegreen}{rgb}{0,0.6,0}
\definecolor{codegray}{rgb}{0.5,0.5,0.5}
\definecolor{codepurple}{rgb}{0.58,0,0.82}
\definecolor{backcolour}{rgb}{0.95,0.95,0.92}

\hypersetup{
	colorlinks,
	citecolor=black,
	filecolor=black,
	linkcolor=black,
	urlcolor=black
}

\lstdefinestyle{mystyle}{
	backgroundcolor=\color{backcolour},   
	commentstyle=\color{codegreen},
	keywordstyle=\color{magenta},
	numberstyle=\tiny\color{codegray},
	stringstyle=\color{codepurple},
	basicstyle=\footnotesize,
	breakatwhitespace=false,         
	breaklines=true,                 
	captionpos=b,                    
	keepspaces=true,                 
	numbers=left,                    
	numbersep=5pt,                  
	showspaces=false,                
	showstringspaces=false,
	showtabs=false,                  
	tabsize=2
}

\lstset{style=mystyle}

\begin{document}
\begin{titlepage}
	

\author{Eric Pereira\\
Student ID: 902215588\\
CSE3020: Section 01}
\date{October 3\textsuperscript{rd}, 2018}
\title{Introduction to Architecture and Assembly Midterm}
\maketitle
\end{titlepage}
\tableofcontents
\newpage
\pagenumbering{arabic}
\section{5.8.2-1}
\subsubsection*{Write a sequence of \texttt{PUSH} and \texttt{POP} instructions to exchange the values of EAX and EBX.}
Answer: \\
\texttt{PUSH EAX \\
PUSH EBX \\
POP EAX \\
POP EBX \\}
\section{5.8.2-2}
\subsubsection*{Suppose you wanted a subroutine to return to an address that was 3 bytes higher in memory than the return address currently on the stack. Write a sequence of instructions that would be inserted just before the subroutine's RET instruction that accomplish this task.}
Answer: \\
\texttt{subProc PROC \\
	POP EAX \\
	ADD EAX, 3 \\
	PUSH EAX \\ 
	RET \\
	subProc ENDP \\}
\section{4.9.2-10}
\subsubsection*{Write a sequence of instructions that set both the Carry and the Overflow flags at the same time.}
Answer: \\
\texttt{mov al, 80h \\
	add al, 80h }
\section{4.9.2-4}
\subsubsection*{Write a code using byte operands that adds two negative integers and causes the overflow flag to be set}
Answer: \\
\texttt{mov al, -100 \\
	add al, -50 }
\section{3.9.2-13}
\subsubsection*{Declare a string variable containing the word \texttt{"TEST"} repeated 500 times}
Answer: \\
\texttt{TESTARRAY BYTE 500 DUP("TEST")}
\section{1.7.1-25}
\subsubsection*{Create a truth table to show all possible inputs and outputs for the Boolean function described by \texttt{NOT (A OR B)}.}
Answer: \newline \newline
$
\begin{array}{c|c|c|c}
	A & B & A \lor B & NOT (A \lor B)
	\\\hline
	T & T & T & F\\
	T & F & T & F\\
	F & T & T & F\\
	F & F & F & T
	
\end{array}$

\section{1.7.1-15}
\subsubsection*{What is the decimal representation of each of the following signed binary numbers?}
\begin{enumerate}[(a)]
	\item 10110111
	\item 00111010
	\item 11111000
\end{enumerate}
Answer: \\
a.) Starts with 1, this means it is signed. Use 2's complement\\
\tab \(10110111\rightarrow01001001\) \\
\tab \(2^0+2^3+2^6=1+8+64=73\) \\
\tab \(-73\) \\
b.) Starts with 0, therefore the number is positive:\\
\tab \(2^5+2^4+2^3+2^1=2+8+16+32=58\) \\
\tab \(58\) \\
c.) Starts with 1, this means it is signed. Use 2's complement\\
\tab \(11111000\rightarrow00001000\) \\
\tab \(2^3=8\) \\
\tab \(-8\) 
\section{4.10-5}
\subsubsection*{Write a program that uses a loop to calculate the first seven values of the Fibonacci number sequence described by the following formula: Fib(1) = 1, Fib(2) = 1, Fib(n)=Fib(n-1)+Fib(n-2)}
Answer: \\
\texttt{mov eax, 1 ;value of Fib(n) \\
	mov ebx, 0 ;value of Fib(n-1) \\
	mov ecx, 7 ;loop decrementer \\
	mov edx, 0 ;value of Fib(n-2) \\
	FIBONACCI: \\
	\tab add eax, edx \\
	\tab mov edx, ebx \\
	\tab mov ebx, eax \\
	\tab loop FIBONACCI}
\end{document}

