\documentclass[12pt]{article}
\newcommand\tab[1][1cm]{\hspace{#1}}
\usepackage[utf8]{inputenc}
\usepackage{listings}
\usepackage{color}
\pagenumbering{gobble}

\definecolor{codegreen}{rgb}{0,0.6,0}
\definecolor{codegray}{rgb}{0.5,0.5,0.5}
\definecolor{codepurple}{rgb}{0.58,0,0.82}
\definecolor{backcolour}{rgb}{0.95,0.95,0.92}

\lstdefinestyle{mystyle}{
	backgroundcolor=\color{backcolour},   
	commentstyle=\color{codegreen},
	keywordstyle=\color{magenta},
	numberstyle=\tiny\color{codegray},
	stringstyle=\color{codepurple},
	basicstyle=\footnotesize,
	breakatwhitespace=false,         
	breaklines=true,                 
	captionpos=b,                    
	keepspaces=true,                 
	numbers=left,                    
	numbersep=5pt,                  
	showspaces=false,                
	showstringspaces=false,
	showtabs=false,                  
	tabsize=2
}

\lstset{style=mystyle}


\begin{document}
\author{Eric Pereira\\
	CSE3120: Section 02}
\date{September 5\textsuperscript{th}, 2018}
\title{Chapter 2: Architecture}
\maketitle

\section*{2.8 Review Questions}

\subsection*{1: In 32-bit mode, aside from the stack pointer (ESP), what other register points to variables on the stack?}
\subsubsection*{Answer:}
The base pointer register. 
\subsection*{3: Which flag is set when the result of an unsigned arithmetic operation is too large to fit into the destination?}
\subsubsection*{Answer:}
The carry flag
\subsection*{4: Which flag is set when the result of a signed arithmetic operation is either too large or too small to fit into the destination?}
\subsubsection*{Answer:}
The overflow flag
\subsection*{6: Which flag is set when an arithmetic or logical operation generates a negative result?}
\subsubsection*{Answer:}
The sign flag
\subsection*{8: On a 32-bit processor, how many bits are contained in each floating-point data register?}
\subsubsection*{Answer:}
80 bits
\end{document}