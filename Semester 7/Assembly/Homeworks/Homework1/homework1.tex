\documentclass[12pt]{article}
\newcommand\tab[1][1cm]{\hspace{#1}}
\usepackage[utf8]{inputenc}
\usepackage{listings}
\usepackage{color}
%\documentclass[12pt]{article}
%\newcommand\tab[1][1cm]{\hspace{#1}}
\usepackage[utf8]{inputenc}
\usepackage{listings}
\usepackage{hyperref}
\usepackage{color}
\pagenumbering{gobble}

\definecolor{codegreen}{rgb}{0,0.6,0}
\definecolor{codegray}{rgb}{0.5,0.5,0.5}
\definecolor{codepurple}{rgb}{0.58,0,0.82}
\definecolor{backcolour}{rgb}{0.95,0.95,0.92}

\hypersetup{
	colorlinks,
	citecolor=black,
	filecolor=black,
	linkcolor=black,
	urlcolor=black
}

\lstdefinestyle{mystyle}{
	backgroundcolor=\color{backcolour},   
	commentstyle=\color{codegreen},
	keywordstyle=\color{magenta},
	numberstyle=\tiny\color{codegray},
	stringstyle=\color{codepurple},
	basicstyle=\footnotesize,
	breakatwhitespace=false,         
	breaklines=true,                 
	captionpos=b,                    
	keepspaces=true,                 
	numbers=left,                    
	numbersep=5pt,                  
	showspaces=false,                
	showstringspaces=false,
	showtabs=false,                  
	tabsize=2
}

\lstset{style=mystyle}
\renewcommand{\thesubsection}{\thesection.\alph{subsection}}


\begin{document}
\author{Eric Pereira\\
	CSE3120: Section 02}
\date{August 27, 2018}
\title{Chapter 1: Data Representation}
\maketitle
\section*{Section 1.7.1}
\subsection*{2a: What is the Decimal representation of the following unsigned binary integer?: 00110101}
\begin{center}
$00110101 = 2^0 + 2^2 + 2^4 + 2^5 = 1 + 4 + 16 + 32 = 53$
\end{center}
\paragraph*{Answer:} 53

\subsection*{3a: What is the sum of the pair of binary integers?: 10101111 + 11011011}
\subsubsection*{Answer:}
\[
\begin{array} {@{}cr}
	& 10101111 \\
   +& 11011011 \\
   \cline{2-2} 
    & 110001010   
  \end{array}	
  \]
  
\subsection*{4: Calculate binary 00001101 minus 00000111}
\subsubsection*{Answer:}
\[
\begin{array} {@{}cr}
& 00001101 \\
-& 00000111 \\
\cline{2-2} 
& 00000110
\end{array}	
\]

\subsection*{5: How many bits are used by each of the following data types?}
\subsubsection*{a. word}
\paragraph{Answer: } 16 bits, 2 bytes
\subsubsection*{b. doubleword}
\paragraph{Answer: } 32 bits, 4 bytes
\subsubsection*{c. quadword}
\paragraph{Answer: } 64 bits, 8 bytes
\subsubsection*{d. double quadword}
\paragraph{Answer: } 128 bits, 16 bytes

\subsection*{7a: What is the hexadecimal representation of the following binary number: 0011 0101 1101 1010?}
$0011 = 2^0 + 2^1 = 1 + 2 = 3$ \\ 
$0101 = 2^0 + 2^2 = 1 + 4 = 5$ \\
$1101 = 2^0 + 2^2 + 2^3 = 1 + 4 + 8 = 13$ OR D in hexadecimal \\ 
$1010 = 2^1 + 2^3 = 2 + 8 = 10$ OR A in hexadecimal \\
When Placed in order you get 35DA
\paragraph*{Answer: } 35DA

\subsection*{15a: What is the decimal representation of the following signed binary number: 10110101?}
\tab A one is the leftmost number of this byte, which indicates that this is a negatively signed number. In order to find the answer we have to use two's complement. First we start by inverting the number. \\
\begin{center}
	01001010
\end{center}
After this we add 1 to the new number.
\begin{center}
	01001011
\end{center}
Finally, we convert the number to decimal and add a negative symbol in front of it. \\
\begin{center}
	$01001011 = 2^0 + 2^1 + 2^3 + 2^6 = 1 + 2 + 8 + 64 = 75$
\end{center}
\paragraph*{Answer:} -75 

\section*{Section 1.7.2}
\subsection*{8: Write a Java program that contains the calculation shown below. Then, use the \textit{javap -c} command to disassemble your code. Add comments to each line that provide your best guess as to its purpose.\\ \\ 
 \tt{int Y;\\
 int X = (Y + 4) * 3;}\\}
\paragraph*{Answer: \\}
\begin{lstlisting}[language=Java, caption=Java Code Written]
public class homework1 {
   public static void main(String[] args){
      int Y = 5;
      int X = (Y + 4) * 3;
   }
}
\end{lstlisting}

\begin{lstlisting}[caption = Return of \textit{javap -c homework1.java} with notes]
public class homework1 {
public homework1();
Code:
0: aload_0
1: invokespecial #1                  
4: return

public static void main(java.lang.String[]);
Code:
0: iconst_5	 //I think this is where I have constant 5
1: istore_1  //This is where it was stored, in value Y
2: iload_1  //This loads Y, starting in int X equation
3: iconst_4  //This gets the constant 4 in the equation
4: iadd  //This adds Y and 4
5: iconst_3 //This gets the constant 3 in the equation
6: imul //This multiplies 3 and result of line 4:
7: istore_2 //This stores value in X
8: return //returns out of function
}
\end{lstlisting}
\end{document}