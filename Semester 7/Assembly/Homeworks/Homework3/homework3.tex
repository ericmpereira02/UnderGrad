\documentclass[12pt]{article}
\newcommand\tab[1][1cm]{\hspace{#1}}
\usepackage[utf8]{inputenc}
\usepackage{listings}
\usepackage{color}
\usepackage{textcomp}
\pagenumbering{gobble}

\definecolor{codegreen}{rgb}{0,0.6,0}
\definecolor{codegray}{rgb}{0.5,0.5,0.5}
\definecolor{codepurple}{rgb}{0.58,0,0.82}
\definecolor{backcolour}{rgb}{0.95,0.95,0.92}

\lstdefinestyle{mystyle}{
	backgroundcolor=\color{backcolour},   
	commentstyle=\color{codegreen},
	keywordstyle=\color{magenta},
	numberstyle=\tiny\color{codegray},
	stringstyle=\color{codepurple},
	basicstyle=\footnotesize,
	breakatwhitespace=false,         
	breaklines=true,                 
	captionpos=b,                    
	keepspaces=true,                 
	numbers=left,                    
	numbersep=5pt,                  
	showspaces=false,                
	showstringspaces=false,
	showtabs=false,                  
	tabsize=2
}

\lstset{style=mystyle}


\begin{document}
\author{Eric Pereira\\
	CSE3120: Section 02}
\date{September 8\textsuperscript{th}, 2018}
\title{Chapter 3: Assembly Introduction}
\maketitle


\section*{Section 3.5.5}
\subsection*{5. Use a TEXTEQU expression to redefine “proc” as “procedure.”}
\begin{lstlisting}
procedure TEXTEQU <proc> 
\end{lstlisting}

\section*{3.9.1}
\subsection*{16. Show an example of a block comment.}
\begin{lstlisting}
	COMMENT ! 
	this
	is
	a
	block
	comment
	!
\end{lstlisting}
\section*{3.9.2}
\subsection*{4. Find out if you can declare a variable of type DWORD and assign it a negative value. What does this tell you about the assembler’s type checking?}
\tab DWORD is an unsigned variable, however you can assign it a negative value. It will not store it as a negative value though, instead it will store it as $2^3^2-x$ where x is the absolute value of the negative number it was assigned.
\subsection*{6. Given the number 456789ABh, list out its byte values in little-endian order.}
\tab the order would be: 0000:AB, 0001:89, 0002:67, 0003:45
\subsection*{12. Declare an uninitialized array of 50 signed doublewords named dArray.}
\begin{lstlisting}
	dArray DWORD 50 DUP(?)
\end{lstlisting}

\section*{3.10}
\subsection*{3. Data Definitions: Write a program that contains a definition of each data type listed in Table 3-2 in Section 3.4. Initialize each variable to a value that is consistent with its data type.}

\end{document}