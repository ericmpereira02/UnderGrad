\documentclass[12pt]{article}
\newcommand\tab[1][1cm]{\hspace{#1}}
\usepackage[utf8]{inputenc}
\usepackage{listings}
\usepackage{color}
\usepackage{textcomp}
\usepackage{sectsty}
\usepackage{graphicx}
\pagenumbering{gobble}

\definecolor{codegreen}{rgb}{0,0.6,0}
\definecolor{codegray}{rgb}{0.5,0.5,0.5}
\definecolor{codepurple}{rgb}{0.58,0,0.82}
\definecolor{backcolour}{rgb}{0.95,0.95,0.92}
\lstdefinestyle{mystyle}{
	backgroundcolor=\color{backcolour},   
	commentstyle=\color{codegreen},
	keywordstyle=\color{magenta},
	numberstyle=\tiny\color{codegray},
	stringstyle=\color{codepurple},
	basicstyle=\footnotesize,
	breakatwhitespace=false,         
	breaklines=true,                 
	captionpos=b,                    
	keepspaces=true,                 
	numbers=left,                    
	numbersep=5pt,                  
	showspaces=false,                
	showstringspaces=false,
	showtabs=false,                  
	tabsize=2
}

\lstset{style=mystyle}


\begin{document}
\author{Eric Pereira\\
	CSE3120: Section 01}
\date{September 22\textsuperscript{nd}, 2018}
\title{Chapter 4 Part 2}
\maketitle
\section*{1.} 
\subsection*{\texttt{message db "smartest"} \\Submit a program that computes the sum of all the byte values in the original message, and stores the result as a number on 4 bytes stored in the last four bytes of the message.}
\texttt{.386 \\
	.model flat,stdcall \\
	.stack 4096 \\
	ExitProcess PROTO, dwExitCode:DWORD \\
	.data \\
	message db "smartest" \\
	target BYTE SIZEOF message DUP(0) \\
	count EQU LENGTHOF message \\
	.code \\
	main PROC \\
	mov esi,0 \\
	mov ecx,SIZEOF message \\
	mov eax, 0 \\
	mov ebx, 0 \\
	lea edx, message \\
	add edx, 4 \\
	L1: \\
	mov al,message[esi] \\
	add ebx, eax \\
	inc esi \\
	loop L1 \\
	mov [edx], ebx \\
	invoke ExitProcess,0 \\
	main ENDP \\
	END main} 
\section*{4.4.5}
\subsection*{1. (True/False): Any 32-bit general-purpose register can be used as an indirect operand.}
True
\section*{5.8}
\subsection*{1. Which instruction pushes all of the 32-bit general-purpose registers on the stack?}
PUSHAD
\subsection*{12. (True/False): The USES operator only generates PUSH instructions, so you must code POP instructions yourself}
False
\section*{5.82}
\subsection*{1. Write a sequence of statements that use only PUSH and POP instructions to exchange the values in the EAX and EBX registers (or RAX and RBX in 64-bit mode). }
\texttt{.386 \\
	.model flat,stdcall \\
	.stack 4096 \\
	ExitProcess PROTO, dwExitCode:DWORD \\
	.code \\
	main PROC \\
\tab	MOV EAX, 2 \\
\tab	MOV EBX, 4 \\
\tab	PUSH EAX \\
\tab	PUSH EBX \\
\tab	POP EAX \\
\tab	POP EBX \\
\tab	INVOKE ExitProcess,0 \\
\tab	main ENDP \\
	END main}
\subsection*{2. Suppose you wanted a subroutine to return to an address that was 3 bytes higher in memory than the return address currently on the stack. Write a sequence of instructions that would be inserted just before the subroutine’s RET instruction that accomplish this task.}
\texttt{.386 \\
	.model flat,stdcall \\
	.stack 4096 \\
	ExitProcess PROTO, dwExitCode:DWORD \\
	.code \\
	main PROC \\
	CALL subRoutine \\
	INVOKE ExitProcess,0 \\
	main ENDP \\
	subRoutine PROC \\
	ADD EBP, 7 \\
	RET \\
	subRoutine ENDP \\
	END main}
\end{document}