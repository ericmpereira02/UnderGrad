\documentclass[12pt]{article}
\newcommand\tab[1][1cm]{\hspace*{#1}}
\usepackage[utf8]{inputenc}
\usepackage{listings}
\usepackage{hyperref}
\usepackage{color}
\usepackage{titlesec}
\pagenumbering{gobble}
\usepackage{changepage}

\usepackage{makecell}

\definecolor{codegreen}{rgb}{0,0.6,0}
\definecolor{codegray}{rgb}{0.5,0.5,0.5}
\definecolor{codepurple}{rgb}{0.58,0,0.82}
\definecolor{backcolour}{rgb}{0.95,0.95,0.92}

\hypersetup{
	colorlinks,
	citecolor=black,
	filecolor=black,
	linkcolor=black,
	urlcolor=black
}

\lstdefinestyle{mystyle}{
	backgroundcolor=\color{backcolour},   
	commentstyle=\color{codegreen},
	keywordstyle=\color{magenta},
	numberstyle=\tiny\color{codegray},
	stringstyle=\color{codepurple},
	basicstyle=\footnotesize,
	breakatwhitespace=false,         
	breaklines=true,                 
	captionpos=b,                    
	keepspaces=true,                 
	numbers=left,                    
	numbersep=5pt,                  
	showspaces=false,                
	showstringspaces=false,
	showtabs=false,                  
	tabsize=2
}

\lstset{style=mystyle}

\titleformat{\section}
{\normalfont\normalsize\bfseries}{\thesection}{1em}{}
\titleformat{\subsection}
{\normalfont\normalsize\bfseries}{\thesection}{1em}{}

% for forcing tables to fit
\usepackage{changepage}

\begin{document}
	

\begin{titlepage}
	
\author{Eric Pereira}
\date{December 7\textsuperscript{th}, 2018}
\title{Assignment 3}

\maketitle

\end{titlepage}

\tableofcontents

\newpage \pagenumbering{arabic}

\section*{Q1. Given the relation r(A,B,C) and the functional dependencies A$\rightarrow$B and B$\rightarrow$C, give a lossless join dependency preserving decomposition of R into BCNF}
\addcontentsline{toc}{section}{\protect\numberline{}Q1}%
Answer: \\
R\textsubscript{1}(A,B) \\
R\textsubscript{2}(B,C)
\section*{Q2. Consider the following functional dependencies for relation schema: \\
$R=(A,B,C,D,E):A\rightarrow BC, CD\rightarrow E, B \rightarrow D, E \rightarrow A.$ \\ Compute $A^+$}
\addcontentsline{toc}{section}{\protect\numberline{}Q2}%
Answer: A\textsuperscript{+}=\{A,B,C,D,E\}

\section*{Q3. Consider the following set F of functional dependencies on the relation schema: \\
$r(A,B,C,D,E,F)$ \\ 
$A\rightarrow BCD, BC\rightarrow DE, B\rightarrow D, D\rightarrow A$)}
\addcontentsline{toc}{section}{\protect\numberline{}Q3}%

\subsection*{1. Compute B\textsuperscript{+}.}
\addcontentsline{toc}{subsection}{\protect\numberline{}1.}%
Answer: B\textsuperscript{+}=\{ABCDE\}
\subsection*{2. Prove (using Armstrong's axioms) that \textit{AF} is a superkey.}
\addcontentsline{toc}{subsection}{\protect\numberline{}2.}%
1. A $\rightarrow$ BCD (Given) \\
2. BC $\rightarrow$ DE (Given) \\
3. BCD $\rightarrow$ DE (Augmentation 2 \& D) \\
4. BCD $\rightarrow$ CDE (Augmentation 3 \& C) \\
5. BCD $\rightarrow$ BCDE (Augmentation 4 \& B) \\
6. A $\rightarrow$ BCDE (Transitivity 1 \& 5) \\
7. AF $\rightarrow$ BCDEF (Augmentation 6 \& F) \\
0. AF\textsuperscript{+} $\rightarrow$ ABCDEF (Augmentation 7 \& A) 



\subsection*{3. Compute a canonical cover for the above set fo functional dependencies F; give each step of your derivation with an explanation.}
\addcontentsline{toc}{subsection}{\protect\numberline{}3.}%
\tab B$\rightarrow$D is extraneous, so in the case A$\rightarrow$BCD, so you could change it to A$\rightarrow$BC. \\
\tab D is also extraneous in BC$\rightarrow$DE because B$\rightarrow$D, so that would change to BC$\rightarrow$E
\subsection*{4. Give a 3NF decomposition of \textit{r} based on the canonical cover.}
R\textsubscript{0}(A,B,C) \{A$\rightarrow$ BC\}\\
R\textsubscript{1}(B,D,E) \{B $\rightarrow$ DE\}\\
R\textsubscript{2}(A,D) \{D $\rightarrow$ A\}\\
R\textsubscript{3}(A,F) \{\}
\addcontentsline{toc}{subsection}{\protect\numberline{}4.}%
\subsection*{5. Give a BCNF decomposition of \textit{r} using the original set of functional dependencies.}
\addcontentsline{toc}{subsection}{\protect\numberline{}5.}%
R\textsubscript{0}(A,B,C) \{A$\rightarrow$ BC\}\\
R\textsubscript{1}(B,D,E) \{B $\rightarrow$ DE\}\\
R\textsubscript{2}(A,D) \{D $\rightarrow$ A\}\\
R\textsubscript{3}(A,F) \{\}
\addcontentsline{toc}{subsection}{\protect\numberline{}4.}%
\section*{Q4. Given the following functional dependencies: \\
$A\rightarrow BCD, CD\rightarrow E, B\rightarrow D, E\rightarrow A,AD\rightarrow E$}
\addcontentsline{toc}{section}{\protect\numberline{}Q4}%

\subsection*{1. Find a canonical cover of the above set of dependencies (you must explain how you arrived at the answer).}
\addcontentsline{toc}{subsection}{\protect\numberline{}1.}%
\tab AD$\rightarrow$E is not needed as this way is proven through transitivity with A$\rightarrow$BCD and CD$\rightarrow$E. \\
\tab D is extraneous in A$\rightarrow$BCD as B$\rightarrow$D. Therefore, you will get A$\rightarrow$BC instead. 
\subsection*{2. Normalize the relation to 3nF (again, you must explain how you arrived at the answer).}
\addcontentsline{toc}{subsection}{\protect\numberline{}2.}%
R\textsubscript{0}(A,B,C) \{A$\rightarrow$BC\} \\
R\textsubscript{1}(B,D) \{B$\rightarrow$ D\} \\
R\textsubscript{2}(A,E) \{E$\rightarrow$A\}\\
R\textsubscript{3}(C,D,E) \{C,D$\rightarrow$E\}



\end{document}
