\documentclass[12pt]{article}
\newcommand\tab[1][1cm]{\hspace*{#1}}
\usepackage[utf8]{inputenc}
\usepackage{listings}
\usepackage{hyperref}
\usepackage{color}
\pagenumbering{gobble}
\usepackage{changepage}

\usepackage{makecell}

\definecolor{codegreen}{rgb}{0,0.6,0}
\definecolor{codegray}{rgb}{0.5,0.5,0.5}
\definecolor{codepurple}{rgb}{0.58,0,0.82}
\definecolor{backcolour}{rgb}{0.95,0.95,0.92}

\hypersetup{
	colorlinks,
	citecolor=black,
	filecolor=black,
	linkcolor=black,
	urlcolor=black
}

\lstdefinestyle{mystyle}{
	backgroundcolor=\color{backcolour},   
	commentstyle=\color{codegreen},
	keywordstyle=\color{magenta},
	numberstyle=\tiny\color{codegray},
	stringstyle=\color{codepurple},
	basicstyle=\footnotesize,
	breakatwhitespace=false,         
	breaklines=true,                 
	captionpos=b,                    
	keepspaces=true,                 
	numbers=left,                    
	numbersep=5pt,                  
	showspaces=false,                
	showstringspaces=false,
	showtabs=false,                  
	tabsize=2
}

\lstset{style=mystyle}

\begin{document}
	

\begin{titlepage}
	
\author{Eric Pereira}
\date{November 13\textsuperscript{th}, 2018}
\title{Assignment 2}

\maketitle

\end{titlepage}

\tableofcontents

\newpage \pagenumbering{arabic}

\section*{Q1: Given the schema\\ 
\tab \textit{item(\underline{itemid}, name, category, price)\\
\tab itemsale(\underline{transid}, itemid, quantity)\\ \tab transaction(\underline{transid}, custid, date) \\
\tab customer(\underline{custid}, name, street-addr, city)}}
\addcontentsline{toc}{section}{\protect\numberline{}Q1}%

\subsection*{1. Find the name and price of the most expensive item (if more than one item is the most expensive, print them all).}
\addcontentsline{toc}{subsection}{\protect\numberline{}1.}%

\begin{lstlisting}
SELECT price, name
FROM item
WHERE price=(
	SELECT MAX(price) 
	FROM item);
\end{lstlisting}

\subsection*{2. Print the total sales (in terms of units and total price) of every item category in every customer-city.}
\addcontentsline{toc}{subsection}{\protect\numberline{}2.}%

\begin{lstlisting}
	SELECT i.name, cust.city, i.category,
	SUM(i.price * sale.quantity) AS totalprice
	FROM item i 
		JOIN itemsale sale 
		JOIN transaction trans
		JOIN customer cust
	GROUP BY i.category, cust.city 
\end{lstlisting}


\subsection*{3. Find items with no sales at all to customers in Mumbai.}
\addcontentsline{toc}{subsection}{\protect\numberline{}3.}%


\begin{lstlisting}
SELECT i.name
FROM item i JOIN itemsale sale JOIN transaction trans JOIN customer cust
WHERE cust.city='Mumbai'
	AND i.itemid NOT IN (
		SELECT itemid 
		FROM itemsale);
	AND trans.transid=sale.transid 
	AND trans.custid=cust.custid
\end{lstlisting}

\subsection*{4. Find customers who bought the same quantity of the same item on subsequent dates.}
\addcontentsline{toc}{subsection}{\protect\numberline{}4.}%

\subsection*{5. Find all the customers who did not buy any item in category "Electronics".}
\addcontentsline{toc}{subsection}{\protect\numberline{}5.}%

\begin{lstlisting}
SELECT customer.name
FROM customer
WHERE (
	customer.custid NOT IN (
		SELECT customer.custid
		FROM customer
		INNER JOIN transaction ON (customer.custid=transaction.custid)
		INNER JOIN itemsale ON (transaction.transid=itemsale.transid)
		INNER JOIN item ON (itemsale.itemid=item.itemid)
		WHERE item.category!='Electronics'
		)
	);
\end{lstlisting}

\section*{Q2: Given the schema\\ 
\tab \textit{member(\underline{memb\_no}, name, age) \\
\tab book(\underline{isbn}, title, authors, publisher) \\ 
\tab borrowed(\underline{memb\_no}, \underline{isbn}, date)}}
\addcontentsline{toc}{section}{\protect\numberline{}Q2}%

\subsection*{1. Print the names of members who have borrowed any book published by “McGraw-Hill”. }
\addcontentsline{toc}{subsection}{\protect\numberline{}1.}%

\begin{lstlisting}
SELECT name
FROM member memb JOIN borrowed borr JOIN book bk 
WHERE memb.memb_no=borr.memb_no 
	AND borr.isbn=bk.isbn 
	AND bk.publisher='McGraw-Hill'
\end{lstlisting}

\subsection*{2. Print the names of members who have borrowed all books published by “McGraw-Hill”.}
\addcontentsline{toc}{subsection}{\protect\numberline{}2.}%

\begin{lstlisting}
SELECT member.name
FROM member
	INNER JOIN borrowed ON (member.memb_no=borrowed.memb_no)
	INNER JOIN book ON (borrowed.isbn=book.isbn)
WHERE book.publisher='McGraw-Hill'
GROUP BY member.name
HAVING (
	COUNT(book.isbn)=(
		SELECT COUNT(borrowed.isbn)
		FROM borrowed
			INNER JOIN member ON (borrowed.memb_no=member.memb_no)
		INNER JOIN book ON (book.isbn=borrowed.isbn)
	WHERE (book.publisher='McGraw-Hill')
	)
);
\end{lstlisting}

\subsection*{3. For each publisher, print the names of members who have borrowed more than five books of that publisher. }
\addcontentsline{toc}{subsection}{\protect\numberline{}3.}%\\

\begin{lstlisting}
	SELECT customer.custid as custid1
	FROM customer
	INNER JOIN transaction on (transaction.transid=customer.custid)
	INNER JOIN itemsale on (itemsale.transid=transaction.transid)
	where transaction.date=(
	SELECT date
	FROM transaction
	INNER JOIN itemsale ON (itemsale.transid=transaction.transid)
	INNER JOIN item ON (itemsale.itemid=item.itemid)
	INNER JOIN customer ON (customer.custid=transaction.custid)
	WHERE customer.custid!=custid1 AND 
	
\end{lstlisting}
\begin{lstlisting}
SELECT member.name, book.publisher
FROM member
	INNER JOIN borrowed ON (member.memb_no=borrowed.memb_no)
	INNER JOIN book ON (book.isbn=borrowed.isbn)
GROUP BY member.name, book.publisher
HAVING (COUNT(borrowed.memb_no)>5)
\end{lstlisting}

\subsection*{4. Print the total number of books borrowed per member. Take into account that if a member does not borrow any books, then that member does not appear in the borrowed relation at all. }
\addcontentsline{toc}{subsection}{\protect\numberline{}4.}%

\begin{lstlisting}
SELECT member.name, COUNT(borrowed.isbn)
FROM member
	INNER JOIN borrowed ON (member.memb_no=borrowed.memb_no)
	INNER JOIN book ON (borrowed.isbn=book.isbn)
GROUP BY member.name	
\end{lstlisting}

\end{document}
