\documentclass{article}

\input{defaultPream}

\usepackage{outlines}

\begin{document}

%----------BEGIN TITLEPAGE----------

\begin{titlepage}

  \title{[name]}
  \date{\today}
  \author{Joseph Bostik\\
    Thomas van Haastrecht\\
    Eric Pereira\\
    Ryan Wojtyla\\}

  \maketitle

\end{titlepage}

%-----------END TITLEPAGE-----------

%----------BEGIN NOTES----------

\section{Notes}

\subsection{What is the premise of our game?}

\begin{outline}[enumerate]
  \1 The player is in charge of preparing a city for nuclear war; the goal is to
  ensure the survival of as much of the population as possible.
  \1 Resources must be managed and relocated, infrastructure must be prepared,
  and the population must be prepared and controlled.
  \1 The player does not know exactly when the bombs will drop, but news
  bulletins will be delivered that indicate the likelihood of imminent war.
  \1 The player must balance preparation with civil unrest.
    \2 If the player institutes provocative measures without suitably
    dark news, the population may resist or become frightened.
    \2 Civil unrest significantly hinders the players ability to adequately prepare.
\end{outline}

\subsection{What kind of game are we making?}

\begin{outline}[enumerate]
  \1 turn-based
  \1 grid-based
    \2 2D
  \1 rogue-like
    \2 the game is short and once it's done, it's done
  \1 procedurally generated 
    \2 random game-end time
    \2 random city layout
      \3 random resource placement
      \3 random buildings
\end{outline}

\subsection{What is the gameplay?}

\begin{outline}[enumerate]
  \1 moving resources
    \2 infrastructure health determines the speed and efficiency of shipping
    \2 resources can be moved from less secure locations to more secure locations
    \2 resources can be more strategically located across the city
    \2 hastily moving resources can cause civil unrest
      \3 the population may take exception to emptying the grocery stores
  \1 modify infrastructure
    \2 roads can be blocked or designated for limited use
    \2 buildings can be fortified
    \2 manage water and power
      \3 being too hasty can cause civil unrest
      \3 may impact ability to prepare
  \1 interact with population
    \2 broadcast PSAs
    \2 institute directives
      \3 rationing
      \3 limited movement
    \2 censorship
      \3 protect population from troubling information
        \4 maintain current level of happiness
      \3 civilian discovery of censorship dramatically decreases happiness
    \2 manage happiness
      \3 maintaining normalcy stabilizes happiness
      \3 directives decrease happiness
      
\end{outline}

\subsection{How is score calculated?}

\begin{outline}[enumerate]
  \1 The player's score is the number of survivors after several time
  intervals. The later intervals will have a score multiplier to incentivize
  survival longevity. 
    \2 how many survived immediately after the blast? 
    \2 a week after? 
    \2 a month after? 
    \2 a year after?
  \1 per tile calculations
    \2 To determine how many people survive the initial blast, each tile will
    undergo a survivability check to determine what percentage of the people
    within it survive.
      \3 A building's structural integrity will determine how likely it is to be
      destroyed.
      \3 If a building is destroyed, all its resources and population are lost.
    \2 The remaining population is used as the starting point for the subsequent
    calculations.
  \1 city-wide calculations
    \2 NOTE: all calculations will be compared to the surviving population; a
    larger survivor rate means more resources will be needed to keep those
    people alive
    \2 the sum of the city's food stores
      \3 How long can the population be fed?
      \3 More people require more food. 
    \2 the state of farms
      \3 Is the land undamaged enough to be usable?
      \3 Is there sufficient tooling and fuel to farm more food?
    \2 the state of medical buildings
      \3 Is there medical equipment available?
        \4 Medical equipment can slightly increase survival rate.
      \3 Medical buildings provide an increase in survival rate; that bonus is
      lost if the buildings are destroyed.
    \2 the number of livable buildings
      \3 People crammed into shelters during the war need someplace to live
      afterward.
      \3 If there is not enough housing, the survival rate decreases.
\end{outline}

%-----------END NOTES-----------


\end{document}
