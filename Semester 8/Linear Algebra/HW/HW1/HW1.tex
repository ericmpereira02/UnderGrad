\documentclass{article}

\usepackage{fancyhdr}
\usepackage{extramarks}
\usepackage{amsmath}
\usepackage{amsthm}
\usepackage{amsfonts}
\usepackage{tikz}
\usepackage[plain]{algorithm}
\usepackage{algpseudocode}
\usepackage{amssymb}

\usetikzlibrary{automata,positioning}

%
% Basic Document Settings
%

\topmargin=-0.45in
\evensidemargin=0in
\oddsidemargin=0in
\textwidth=6.5in
\textheight=9.0in
\headsep=0.25in

\linespread{1.1}

\pagestyle{fancy}
\lhead{\hmwkAuthorName}
\chead{\hmwkClass\ (\hmwkClassInstructor\ \hmwkClassTime): \hmwkTitle}
\rhead{\firstxmark}
\lfoot{\lastxmark}
\cfoot{\thepage}

\newcommand\tab[1][1cm]{\hspace*{#1}}
\renewcommand\headrulewidth{0.4pt}
\renewcommand\footrulewidth{0.4pt}

\setlength\parindent{0pt}

%
% Create Problem Sections
%

\newcommand{\enterProblemHeader}[1]{
    \nobreak\extramarks{}{Problem \arabic{#1} continued on next page\ldots}\nobreak{}
    \nobreak\extramarks{Problem \arabic{#1} (continued)}{Problem \arabic{#1} continued on next page\ldots}\nobreak{}
}

\newcommand{\exitProblemHeader}[1]{
    \nobreak\extramarks{Problem \arabic{#1} (continued)}{Problem \arabic{#1} continued on next page\ldots}\nobreak{}
    \stepcounter{#1}
    \nobreak\extramarks{Problem \arabic{#1}}{}\nobreak{}
}

\setcounter{secnumdepth}{0}
\newcounter{partCounter}
\newcounter{homeworkProblemCounter}
\setcounter{homeworkProblemCounter}{1}
\nobreak\extramarks{Problem \arabic{homeworkProblemCounter}}{}\nobreak{}

%
% Homework Problem Environment
%
% This environment takes an optional argument. When given, it will adjust the
% problem counter. This is useful for when the problems given for your
% assignment aren't sequential. See the last 3 problems of this template for an
% example.
%
\newenvironment{homeworkProblem}[1][-1]{
    \ifnum#1>0
        \setcounter{homeworkProblemCounter}{#1}
    \fi
    \section{Problem \arabic{homeworkProblemCounter}}
    \setcounter{partCounter}{1}
    \enterProblemHeader{homeworkProblemCounter}
}{
    \exitProblemHeader{homeworkProblemCounter}
}

%
% Homework Details
%   - Title
%   - Due date
%   - Class
%   - Section/Time
%   - Instructor
%   - Author
%

\newcommand{\hmwkTitle}{Homework\ \#1}
\newcommand{\hmwkDueDate}{January 25, 2019}
\newcommand{\hmwkClass}{Linear Algebra}
\newcommand{\hmwkClassTime}{Section 01}
\newcommand{\hmwkClassInstructor}{Dr. Munevver Subasi}
\newcommand{\hmwkAuthorName}{\textbf{Eric Pereira}}

%
% Title Page
%
\pagenumbering{gobble}
\title{
    \vspace{2in}
    \textmd{\textbf{\hmwkClass:\ \hmwkTitle}}\\
    \normalsize\vspace{0.1in}\small{Due\ on\ \hmwkDueDate\ at 11:59pm}\\
    \vspace{0.1in}\large{\textit{\hmwkClassInstructor\ \hmwkClassTime}}
    \vspace{3in}
}

\author{\hmwkAuthorName}
\date{}

\renewcommand{\part}[1]{\textbf{\large Part \Alph{partCounter}}\stepcounter{partCounter}\\}

%
% Various Helper Commands
%

% Useful for algorithms
\newcommand{\alg}[1]{\textsc{\bfseries \footnotesize #1}}

% For derivatives
\newcommand{\deriv}[1]{\frac{\mathrm{d}}{\mathrm{d}x} (#1)}

% For partial derivatives
\newcommand{\pderiv}[2]{\frac{\partial}{\partial #1} (#2)}

% Integral dx
\newcommand{\dx}{\mathrm{d}x}

% Alias for the Solution section header
\newcommand{\solution}{\textbf{\large Solution}}

% Probability commands: Expectation, Variance, Covariance, Bias
\newcommand{\E}{\mathrm{E}}
\newcommand{\Var}{\mathrm{Var}}
\newcommand{\Cov}{\mathrm{Cov}}
\newcommand{\Bias}{\mathrm{Bias}}

\begin{document}

\maketitle

\pagebreak
\pagenumbering{arabic}

\begin{homeworkProblem}
	\tab Let \textbf{u}, \textbf{v} $\in$ $\mathbb{R} ^2$ and $\theta$ be the angle between them. Prove that 
	\begin{equation*}
		\cos \theta = \frac{\textbf{u} \cdot \textbf{v}}{||\textbf{u}||||		\textbf{v}||}
	\end{equation*}
	\tab Where $||$\textbf{u}$||$=$\sqrt{u_1^2+...+u_n^2} $ and $||$\textbf{v}$||$=$\sqrt{v_1^2+...+v_n^2}$ \\

    \textbf{Solution} \\
	\tab To start let us recognize the law of cosines: 
	\begin{equation*}
		c^2 = a^2 + b^2 - 2ab\cos\theta
	\end{equation*}
	\tab Now lets put this in terms of $||\textbf{u}||$ and $||\textbf{v}||$ where:
	\begin{equation*}
		a=||\textbf{u}|| $$  $$  
		b=||\textbf{v}|| $$  $$
		c=||\textbf{v}-\textbf{u}||
	\end{equation*}
	\tab So now the law of cosines equation becomes:
	\begin{equation*}
		||\textbf{v-u}||^2=||\textbf{u}||^2 + ||\textbf{v}||^2 - 2||\textbf{u}||||\textbf{v}||\cos\theta $$ 
		\tab Furthermore, this is the definition of dot product: $$
		\textbf{u} \cdot \textbf{v} = ||\textbf{u}||||\textbf{v}||\cos\theta $$
		\tab Using the definition of dot product we can put it in terms of cos $\theta$ and use it in the law of cosines equation: $$
		\textbf{u} \cdot \textbf{v} = ||\textbf{u}||||\textbf{v}||\cos\theta $$ $$
		\frac{\textbf{u} \cdot \textbf{v}}{||\textbf{u}||||\textbf{v}||}=\cos\theta $$
		\tab Now to utilize this in the dot product equation: $$
		||\textbf{v} - \textbf{u} ||^2 = ||\textbf{u}||^2+ ||\textbf{v}||^2-2\textbf{u} \cdot \textbf{v} $$
		\tab We can replace $||\textbf{v}-\textbf{u}||^2$ with $||\textbf{u}||^2+||\textbf{v}||^2-2||\textbf{u}||||\textbf{v}||\cos\theta$ and compare. $$
		||\textbf{u}||^2+||\textbf{v}||^2-2||\textbf{u}||||\textbf{v}||\cos\theta = ||\textbf{u}||^2+ ||\textbf{v}||^2-2\textbf{u} \cdot \textbf{v} $$ $$
		-2||\textbf{u}||||\textbf{v}||\cos\theta=-2\textbf{u}\cdot\textbf{v} $$ $$
		||\textbf{u}||||\textbf{v}||\cos\theta=\textbf{u}\cdot\textbf{v} $$ $$
		\cos\theta=\frac{\textbf{u}\cdot\textbf{v}}{||\textbf{u}||||\textbf{v}||}
	\end{equation*}
\end{homeworkProblem}
\newpage 

\begin{homeworkProblem}
	\tab Let \textbf{u}, \textbf{v} $\in$ $\mathbb{R} ^n$ . Prove the Cauchy-Schwarz Inequality in $\mathbb{R} ^n$
	
	\begin{equation*}
	(\textbf{u} \cdot \textbf{v})^2 \leq ||\textbf{u}||^2||\textbf{v}||^2
	\end{equation*}
	\tab or equivalently,
	\begin{equation*}
		(\textbf{u} \cdot \textbf{v})\leq ||\textbf{u}||||\textbf{v}||,
	\end{equation*}
	\tab where $||\textbf{u}||=\sqrt{u_1^2+...+u_n^2}$ and $||\textbf{v}||=\sqrt{v_1^2+...+v_n^2}$ \\
\textbf{Solution} \\
	\tab To start the solution the definiton of the dot product can be used.
	\begin{equation*}
		\textbf{u} \cdot \textbf{v} = ||\textbf{u}||||\textbf{v}||\cos\theta $$
		\tab Use this definition to replace dot product with the Cauchy-Schwarz Inequality: $$
		(||\textbf{u}||||\textbf{v}||\cos\theta)^2\leq||u||^2||v||^2 $$ $$
		||\textbf{u}||^2||\textbf{v}||^2\cos^2\theta\leq||u||^2||v||^2 $$ $$
		\cos^2\theta\leq1
	\end{equation*}
	\tab We know that for any $\cos^2\theta$ will hold the maximum value of 1, which makes this statement true.
\end{homeworkProblem}
\newpage

\begin{homeworkProblem}
	\tab Let \textbf{u}, \textbf{v} $\in$ $\mathbb{R} ^n$ . Prove the Parallelogram Equation for Vectors:
	\begin{equation*}
		||\textbf{u}+\textbf{v}||^2  + ||\textbf{u}-\textbf{v}||^2=2(||\textbf{u}||^2+||\textbf{v}||^2),
	\end{equation*}
	\tab where $||\textbf{u}||=\sqrt{u_1^2+...+u_n^2}$ and $||\textbf{v}||=\sqrt{v_1^2+...+v_n^2}$ \\
	\textbf{Solution} \\
	\tab This problem can be solved using basic algebra. Lets start with the left side:
	\begin{equation*}
		||\textbf{u}||+||\textbf{v}||^2  + ||\textbf{u}-\textbf{v}||^2 = 2(||\textbf{u}||^2+||\textbf{v}||^2)$$ $$
		(||\textbf{u}||+||\textbf{v}||)(||\textbf{u}||+||\textbf{v}||) + (||\textbf{u}-\textbf{v}||)(||\textbf{u}-\textbf{v}||) = 2(||\textbf{u}||^2+||\textbf{v}||^2)$$ $$
		(||\textbf{u}||^2+||\textbf{v}||^2+2||\textbf{u}||||\textbf{v}||)+(||\textbf{u}||^2+||\textbf{v}||^2-2||\textbf{u}||||\textbf{v}||)= 2(||\textbf{u}||^2+||\textbf{v}||^2) $$ $$
		2||\textbf{u}||^2+2||\textbf{v}||^2 = 2(||\textbf{u}||^2+||\textbf{v}||^2) $$ $$
		2(||\textbf{u}||^2+||\textbf{v}||^2)=2(||\textbf{u}||^2+||\textbf{v}||^2)
	\end{equation*}
	\tab As a result of doing this algebra it is easy to see that both sides are the same, and prove the Parallelogram Equation for Vectors.
\end{homeworkProblem}
\end{document}