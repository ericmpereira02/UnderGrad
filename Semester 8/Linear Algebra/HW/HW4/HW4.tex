\documentclass{article}

\usepackage{fancyhdr}
\usepackage{extramarks}
\usepackage{amsmath}
\usepackage{amsthm}
\usepackage{amsfonts}
\usepackage{tikz}
\usepackage[plain]{algorithm}
\usepackage{algpseudocode}
\usepackage{amssymb}
\usepackage{enumitem}
\usepackage{relsize}
\usepackage{textcomp}

\usetikzlibrary{automata,positioning}

%
% Basic Document Settings
%

\topmargin=-0.45in
\evensidemargin=0in
\oddsidemargin=0in
\textwidth=6.5in
\textheight=9.0in
\headsep=0.25in

\linespread{1.1}

\pagestyle{fancy}
\lhead{\hmwkAuthorName}
\chead{\hmwkClass\ (\hmwkClassInstructor\ \hmwkClassTime): \hmwkTitle}
\rhead{\firstxmark}
\lfoot{\lastxmark}
\cfoot{\thepage}

\newcommand{\minus}{\scalebox{0.5}[1.0]{$-$}}
\newcommand\tab[1][1cm]{\hspace*{#1}}
\renewcommand\headrulewidth{0.4pt}
\renewcommand\footrulewidth{0.4pt}
\renewcommand{\theenumi}{\Alph{enumi}}

\setlength\parindent{0pt}

%
% Create Problem Sections
%

\newcommand{\enterProblemHeader}[1]{
    \nobreak\extramarks{}{Problem \arabic{#1} continued on next page\ldots}\nobreak{}
    \nobreak\extramarks{Problem \arabic{#1} (continued)}{Problem \arabic{#1} continued on next page\ldots}\nobreak{}
}

\newcommand{\exitProblemHeader}[1]{
    \nobreak\extramarks{Problem \arabic{#1} (continued)}{Problem \arabic{#1} continued on next page\ldots}\nobreak{}
    \stepcounter{#1}
    \nobreak\extramarks{Problem \arabic{#1}}{}\nobreak{}
}

\setcounter{secnumdepth}{0}
\newcounter{partCounter}
\newcounter{homeworkProblemCounter}
\setcounter{homeworkProblemCounter}{1}
\nobreak\extramarks{Problem \arabic{homeworkProblemCounter}}{}\nobreak{}

%
% Homework Problem Environment
%
% This environment takes an optional argument. When given, it will adjust the
% problem counter. This is useful for when the problems given for your
% assignment aren't sequential. See the last 3 problems of this template for an
% example.
%
\newenvironment{homeworkProblem}[1][-1]{
    \ifnum#1>0
        \setcounter{homeworkProblemCounter}{#1}
    \fi
    \section{Problem \arabic{homeworkProblemCounter}}
    \setcounter{partCounter}{1}
    \enterProblemHeader{homeworkProblemCounter}
}{
    \exitProblemHeader{homeworkProblemCounter}
}

%
% Homework Details
%   - Title
%   - Due date
%   - Class
%   - Section/Time
%   - Instructor
%   - Author
%

\newcommand{\hmwkTitle}{Homework\ \#4}
\newcommand{\hmwkDueDate}{February 24, 2019}
\newcommand{\hmwkClass}{Linear Algebra}
\newcommand{\hmwkClassTime}{Section 01}
\newcommand{\hmwkClassInstructor}{Dr. Subasi}
\newcommand{\hmwkAuthorName}{\textbf{Eric Pereira}}

%
% Title Page
%
\pagenumbering{gobble}
\title{
    \vspace{2in}
    \textmd{\textbf{\hmwkClass:\ \hmwkTitle}}\\
    \normalsize\vspace{0.1in}\small{Due\ on\ \hmwkDueDate\ at 11:59pm}\\
    \vspace{0.1in}\large{\textit{\hmwkClassInstructor\ \hmwkClassTime}}
    \vspace{3in}
}

\author{\hmwkAuthorName}
\date{}

\renewcommand{\part}[1]{\textbf{\large Part \Alph{partCounter}}\stepcounter{partCounter}\\}

%
% Various Helper Commands
%

% Useful for algorithms
\newcommand{\alg}[1]{\textsc{\bfseries \footnotesize #1}}

% For derivatives
\newcommand{\deriv}[1]{\frac{\mathrm{d}}{\mathrm{d}x} (#1)}

% For partial derivatives
\newcommand{\pderiv}[2]{\frac{\partial}{\partial #1} (#2)}

% Integral dx
\newcommand{\dx}{\mathrm{d}x}

% Alias for the Solution section header
\newcommand{\solution}{\textbf{\large Solution}}

% Probability commands: Expectation, Variance, Covariance, Bias
\newcommand{\E}{\mathrm{E}}
\newcommand{\Var}{\mathrm{Var}}
\newcommand{\Cov}{\mathrm{Cov}}
\newcommand{\Bias}{\mathrm{Bias}}

\begin{document}

\maketitle

\pagebreak
\pagenumbering{arabic}

%%%%%%%%%%%%%%%%%%%%%%%%%%%%%%%%%%%%%%%%%%%%%%%%%%%%%%%%%%%%%%%%%%%%%%%%%%%%%%%%%%
%                                                                                %
%                              Problem 1                                         %
%                                                                                %
%%%%%%%%%%%%%%%%%%%%%%%%%%%%%%%%%%%%%%%%%%%%%%%%%%%%%%%%%%%%%%%%%%%%%%%%%%%%%%%%%%

\begin{homeworkProblem}
	Consider the $4 \times 4$ matrix $A=\left({\begin{array}{cccc}
	1&1&0&1 \\ \minus 1&\minus 1&2&1 \\ 2&2&\minus 1&1 \\ \minus 1&\minus 1&\minus 1&0
	 \end{array}}\right)$.
	\begin{enumerate}[label=(\alph*)]
		\item (10pts) Find the reduced row echelon form of $A$. \\
		\textbf{Solution:} \\
		\begin{align*}
			R2+R1\rightarrow R2 \left({\begin{array}{cccc}
				1&1&0&1 \\ 0&0&2&2 \\ 2&2&\minus 1&1 \\ \minus 1&\minus 1&\minus 1&0
				\end{array}}\right) \\
			R3-2R1\rightarrow R3 \left({\begin{array}{cccc}
				1&1&0&1 \\ 0&0&2&2 \\ 0&0&\minus 1&\minus 1 \\ \minus 1&\minus 1&\minus 1&0
				\end{array}}\right) \\
			R4+R1\rightarrow R4 \left({\begin{array}{cccc}
				1&1&0&1 \\ 0&0&2&2 \\ 0&0&\minus 1&\minus 1 \\ 0&0&\minus 1& 1 \end{array}}\right) \\
			\frac{1}{2}R2\rightarrow R2 \left({\begin{array}{cccc}
				1&1&0&1 \\ 0&0&1&1 \\ 0&0&\minus 1&\minus 1 \\ 0&0&\minus 1& 1 \end{array}}\right) \\
			R3+R2\rightarrow R3 \left({\begin{array}{cccc}
				1&1&0&1 \\ 0&0&1&1 \\ 0&0&0&0 \\ 0&0&\minus 1& 1 \end{array}}\right) \\
			R4+R2\rightarrow R4 \left({\begin{array}{cccc}
				1&1&0&1 \\ 0&0&1&1 \\ 0&0&0&0 \\ 0&0&0&2 \end{array}}\right) \\ 
			R3\leftrightarrow R4\left({\begin{array}{cccc}
				1&1&0&1 \\ 0&0&1&1 \\ 0&0&0&2 \\  0&0&0&0 \end{array}}\right) \\
			\frac{1}{2}R3\rightarrow R3\left({\begin{array}{cccc}
				1&1&0&1 \\ 0&0&1&1 \\ 0&0&0&1 \\  0&0&0&0 \end{array}}\right) \\
		\end{align*}
		\begin{align*} 
			R2-R3\rightarrow R2\left({\begin{array}{cccc}
				1&1&0&1 \\ 0&0&1&0 \\ 0&0&0&1 \\  0&0&0&0 \end{array}}\right) \\ 
			R1-R3\rightarrow R1\left({\begin{array}{cccc}
				1&1&0&0 \\ 0&0&1&0 \\ 0&0&0&1 \\  0&0&0&0 \end{array}}\right) \\
		\end{align*}
	
		\item (5pts) Is the system $Ax=b$ consistent for every $b\in\mathbb{R}^n$.
		Justify your answer. \\
		\textbf{Solution:} \\
		No, this can be proven in:
		\begin{align*}
			b=\left({\begin{array}{c} b_1 \\ b_2 \\ b_3 \\ b_4 \end{array}}\right)
		\end{align*}
		\begin{align*}
			\left({\begin{array}{cccc|c}
				1&1&0&0&b_1 \\ 0&0&1&0&b_2 \\ 0&0&0&1&b_3 \\  0&0&0&0&b_4 \end{array}}\right)
		\end{align*}
		As we can see, in the last row if $b_4$ is a value that is non-zero then it does not 
		work; therefore we can see that $Ax=b$ is inconsistent.
	
		\item (10pts) Find a basis for null-space $Null\text{ }A$. What is the
		dimension of $Null\text{ }A$?\\
		\textbf{Solution:} \\
		\begin{align*}
			\left({\begin{array}{cccc|c}
				1&1&0&0&0 \\ 0&0&1&0&0 \\ 0&0&0&1&0 \\  0&0&0&0&0 \end{array}}\right) \\
			x_3=0,\; x_4=0,\; x_2=s,\; x_1=-s \\
			Null\;A=\left({\begin{array}{c} \minus 1 \\ 1 \\ 0 \\ 0 \end{array}}\right)
		\end{align*}
	
		\item (5pts) Determine whether the set $S=\left\{
		\left({\begin{array}{c} \minus 1 \\ 1 \\ 1 \\ 0 \end{array}}\right)_\mathlarger{,}
		\left({\begin{array}{c} \minus 6 \\ 4 \\ 2 \\ 2 \end{array}}\right)_\mathlarger{,}
		\left({\begin{array}{c} \minus 2 \\ 1 \\ 0 \\ 1 \end{array}}\right)
		\right\}$ is a basis for $Null\;A$. \\
		\textbf{Solution:} \\
		No, it is impossible for the set $S$ to be a basis for $Null\;A$ because all that
		can be done on $Null\;A$ is scalar multiplication, and using scalar multiplication it
		is impossible to get any values in the the third and fourth rows as they are 0, making
		this basis impossible.
		
		\item (10pts) Find a basis for column-space $Col\;A$. What is the
		dimension of $Col\;A$? \\
		\textbf{Solution:} \\
		\begin{align*}
			\left\{
			\left({\begin{array}{c} 1 \\ 0 \\ 0 \\ 0 \end{array}}\right)_\mathlarger{,}
			\left({\begin{array}{c} 0 \\ 1 \\ 0 \\ 0 \end{array}}\right)_\mathlarger{,}
			\left({\begin{array}{c} 1 \\ 0 \\ 1 \\ 0 \end{array}}\right)
			\right\}
		\end{align*}
		
		\item (5pts) Determine whether the set $S=\left\{
		\left({\begin{array}{c} \minus 1 \\ 1 \\ 1 \\ 0 \end{array}}\right)_\mathlarger{,}
		\left({\begin{array}{c} \minus 6 \\ 4 \\ 2 \\ 2 \end{array}}\right)_\mathlarger{,}
		\left({\begin{array}{c} \minus 2 \\ 1 \\ 0 \\ 1 \end{array}}\right)
		\right\}$ is a basis for $Col\;A$. \\
		\textbf{Solution:} \\
		No, it is impossible for $S$ to be a basis in $Col\;A$ because $Col\;A$ has no values
		in the fourth row, therefore it is impossible for $S$ to exist as a basis because it does
		have values in the fourth row. 
		
		
		\item (5pts) Determine whether the vector $u=
		\left({\begin{array}{c} \minus 1 \\ 0 \\ 1 \\ \minus 2 \end{array}}\right)$ belongs
		to $Null\;A$. \\
		\textbf{Solution:} \\
		This vector does not belong in $Null\;A$ because using only scalar multiplication and
		vector addition it is impossible to get values in the third and fourth rows as the 
		third and fourth row in $Null\;A$ are 0. This makes it impossible to hold the values
		that $u$ has. 
		
		\item (5pts)Determine whether the vector $u=
		\left({\begin{array}{c} \minus 1 \\ 0 \\ 1 \\ \minus 2 \end{array}}\right)$ belongs
		to $Col\text{ }A$. \\
		\textbf{Solution:} \\ 
		It does not exist in $Col\;A$ because none of the vectors in $Col\;A$ have a value in
		the fourth row, meaning that it is impossible for $u$ to exist at all in $Col\;A$
	
	
	\end{enumerate}
\end{homeworkProblem}
\newpage

%%%%%%%%%%%%%%%%%%%%%%%%%%%%%%%%%%%%%%%%%%%%%%%%%%%%%%%%%%%%%%%%%%%%%%%%%%%%%%%%%%
%                                                                                %
%                              Problem 2                                         %
%                                                                                %
%%%%%%%%%%%%%%%%%%%%%%%%%%%%%%%%%%%%%%%%%%%%%%%%%%%%%%%%%%%%%%%%%%%%%%%%%%%%%%%%%%

\begin{homeworkProblem}
	(20pts) Let $A$ be an $m\times n$ matrix. Prove that the null-space of 
	$A^TA$ is a subspace of $\mathbb{R}^n$ \\
	\textbf{Solution:} \\
	If $A$ is an $m\times n$ matrix, that means that $A^T$ is an $n\times m$ matrix. This
	means that $A^TA$ will be an $n\times n$ matrix. Now, the $null(A^TA)$ is a set of vectors, these
	vectors show that $A^TAx=0$. let's say that there is some $y$ where $A^TAy=0$. then, using vector
	addition we can say:
	\begin{align*}
		A^TAx=A^TAy=0 
	\end{align*}
	\begin{align*}
		A^TA(x+y)=A^TAx+A^TAy=0+0=0
	\end{align*}
	Therefore we can prove it under vector addition, next is to prove it under scalar multiplication.
	This can be shown by:
	\begin{align*}
		A^TA(cx)=c(A^TAx)=c(0)=0
	\end{align*} 
	
\end{homeworkProblem}
\newpage

%%%%%%%%%%%%%%%%%%%%%%%%%%%%%%%%%%%%%%%%%%%%%%%%%%%%%%%%%%%%%%%%%%%%%%%%%%%%%%%%%%
%                                                                                %
%                              Problem 3                                         %
%                                                                                %
%%%%%%%%%%%%%%%%%%%%%%%%%%%%%%%%%%%%%%%%%%%%%%%%%%%%%%%%%%%%%%%%%%%%%%%%%%%%%%%%%%

\begin{homeworkProblem}
	(25pts)Let $A$ be an $m\times n$ matrix. Prove that the null-space of $A$ and
	null-space of $A^TA$ are equal, i.e.,
	\begin{align*}
		Null\text{ }A=Null\text{ }A^TA
	\end{align*}
	\textbf{Hint:} You must prove a vector $x\in$ \textit{Null} $A$ also
	belongs to \textit{Null} $A^TA$ and any vector $x\in$ \textit{Null} $A^TA$ 
	also belongs to \textit{Null} $A$. \\
	\textbf{Solution:} \\
	Let say $x$ is a vector in $null(A)$. We know that $Ax=0$. Let's multiply $x$ with $A^TA$, so that
	we get $A^TAx$. We know $Ax=0$ so we can alter the equation such that:
	\begin{align*}
		A^TAx \\
		A^T(Ax)\\
		A^T(0)=0
	\end{align*}
	This means $x$ also exists in $null(A^TA)$. At this point we know any $x\in null(A)$ exists in 
	$null(A^TA)$, now we have to prove that $null(A^TA)$ exists in $null(A)$. This is a bit more
	complicated, but can be shown by:
	\begin{align*}
		A^TAx=0 \\
		(A^T)^{\minus 1}(A^TAx)=(A^T)^{\minus 1}(0) \\
		Ax=0
	\end{align*}
	Therefore proving anything in $null(A^TA)$ exists within $null(A)$.
\end{homeworkProblem}
\newpage



\end{document}