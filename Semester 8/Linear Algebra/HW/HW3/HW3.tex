\documentclass{article}

\usepackage{fancyhdr}
\usepackage{extramarks}
\usepackage{amsmath}
\usepackage{amsthm}
\usepackage{amsfonts}
\usepackage{tikz}
\usepackage[plain]{algorithm}
\usepackage{algpseudocode}
\usepackage{amssymb}
\usepackage{enumitem}
\usepackage{relsize}
\usepackage{textcomp}

\usetikzlibrary{automata,positioning}

%
% Basic Document Settings
%

\newcommand{\minus}{\scalebox{0.5}[1.0]{$-$}}
\topmargin=-0.45in
\evensidemargin=0in
\oddsidemargin=0in
\textwidth=6.5in
\textheight=9.0in
\headsep=0.25in

\linespread{1.1}

\pagestyle{fancy}
\lhead{\hmwkAuthorName}
\chead{\hmwkClass\ (\hmwkClassInstructor\ \hmwkClassTime): \hmwkTitle}
\rhead{\firstxmark}
\lfoot{\lastxmark}
\cfoot{\thepage}

\newcommand\tab[1][1cm]{\hspace*{#1}}
\renewcommand\headrulewidth{0.4pt}
\renewcommand\footrulewidth{0.4pt}
\renewcommand{\theenumi}{\Alph{enumi}}

\setlength\parindent{0pt}

%
% Create Problem Sections
%

\newcommand{\enterProblemHeader}[1]{
    \nobreak\extramarks{}{Problem \arabic{#1} continued on next page\ldots}\nobreak{}
    \nobreak\extramarks{Problem \arabic{#1} (continued)}{Problem \arabic{#1} continued on next page\ldots}\nobreak{}
}

\newcommand{\exitProblemHeader}[1]{
    \nobreak\extramarks{Problem \arabic{#1} (continued)}{Problem \arabic{#1} continued on next page\ldots}\nobreak{}
    \stepcounter{#1}
    \nobreak\extramarks{Problem \arabic{#1}}{}\nobreak{}
}

\setcounter{secnumdepth}{0}
\newcounter{partCounter}
\newcounter{homeworkProblemCounter}
\setcounter{homeworkProblemCounter}{1}
\nobreak\extramarks{Problem \arabic{homeworkProblemCounter}}{}\nobreak{}

%
% Homework Problem Environment
%
% This environment takes an optional argument. When given, it will adjust the
% problem counter. This is useful for when the problems given for your
% assignment aren't sequential.
%
\newenvironment{homeworkProblem}[1][-1]{
    \ifnum#1>0
        \setcounter{homeworkProblemCounter}{#1}
    \fi
    \section{Problem \arabic{homeworkProblemCounter}}
    \setcounter{partCounter}{1}
    \enterProblemHeader{homeworkProblemCounter}
}{
    \exitProblemHeader{homeworkProblemCounter}
}

%
% Homework Details
%   - Title
%   - Due date
%   - Class
%   - Section/Time
%   - Instructor
%   - Author
%

\newcommand{\hmwkTitle}{Homework\ \#3}
\newcommand{\hmwkDueDate}{February 18, 2019}
\newcommand{\hmwkClass}{Linear Algebra}
\newcommand{\hmwkClassTime}{Section 01}
\newcommand{\hmwkClassInstructor}{Dr. Subasi}
\newcommand{\hmwkAuthorName}{\textbf{Eric Pereira}}

%
% Title Page
%
\pagenumbering{gobble}
\title{
    \vspace{2in}
    \textmd{\textbf{\hmwkClass:\ \hmwkTitle}}\\
    \normalsize\vspace{0.1in}\small{Due\ on\ \hmwkDueDate\ at 11:59pm}\\
    \vspace{0.1in}\large{\textit{\hmwkClassInstructor\ \hmwkClassTime}}
    \vspace{3in}
}

\author{\hmwkAuthorName}
\date{}

\renewcommand{\part}[1]{\textbf{\large Part \Alph{partCounter}}\stepcounter{partCounter}\\}

%
% Various Helper Commands
%

% Useful for algorithms
\newcommand{\alg}[1]{\textsc{\bfseries \footnotesize #1}}

% For derivatives
\newcommand{\deriv}[1]{\frac{\mathrm{d}}{\mathrm{d}x} (#1)}

% For partial derivatives
\newcommand{\pderiv}[2]{\frac{\partial}{\partial #1} (#2)}

% Integral dx
\newcommand{\dx}{\mathrm{d}x}

% Alias for the Solution section header
\newcommand{\solution}{\textbf{\large Solution}}

% Probability commands: Expectation, Variance, Covariance, Bias
\newcommand{\E}{\mathrm{E}}
\newcommand{\Var}{\mathrm{Var}}
\newcommand{\Cov}{\mathrm{Cov}}
\newcommand{\Bias}{\mathrm{Bias}}

\begin{document}

\maketitle

\pagebreak
\pagenumbering{arabic}

%%%%%%%%%%%%%%%%%%%%%%%%%%%%%%%%%%%%%%%%%%%%%%%%%%%%%%%%%%%%%%%%%%%%%%%%%%%%%%%%%
%																				%
%																			    %
%							  PROBLEM 1                                         %
%                                                                               %
%																				%
%%%%%%%%%%%%%%%%%%%%%%%%%%%%%%%%%%%%%%%%%%%%%%%%%%%%%%%%%%%%%%%%%%%%%%%%%%%%%%%%%

\begin{homeworkProblem}
Let $S=\left\{
\left({\begin{array}{c} 1 \\ 0 \end{array}}\right)_\mathlarger{,}
\left({\begin{array}{c} \minus3 \\ 3 \end{array}}\right)_\mathlarger{,}
\left({\begin{array}{c} 0 \\ 1 \end{array}}\right)
\right\}$ be a set of vectors in $\mathbb{R}^2$.
%\left({\begin{array}{c} 1 \\ 0 \end{array}}\right)$
	\begin{enumerate}[label=(\alph*)]
		\item (10 points) find $span$ $S$ and determine whether the vector $
		\left({\begin{array}{c} 4 \\ \minus5 \end{array}}\right)$ is in $span$ $S$. 
		\\
		\textbf{Solution:} \\
		The span $S$ is equal to:
		\begin{align*}
			S=\left\{
			\left({\begin{array}{c} 1 \\ 0 \end{array}}\right)_\mathlarger{,}
			\left({\begin{array}{c} \minus3 \\ 3 \end{array}}\right)
			\right\}
		\end{align*}
		It is possible to get rid of the $\left({\begin{array}{c} 0 \\ 1
		\end{array}}\right)$ vector because you can get that vector by:
		\begin{align*}
			\left({\begin{array}{c} 1 \\ 0 \end{array}}\right)+
			\frac{1}{3} \left({\begin{array}{c} \minus3 \\ 3 \end{array}}\right)=
			\left({\begin{array}{c} 0 \\ 1 \end{array}}\right)
		\end{align*}
		and the vector $\left({\begin{array}{c} 4 \\ \minus5 \end{array}}\right)$ is
		a part of the span, this can be proven by:
		\begin{align*}
			-\left({\begin{array}{c} 1 \\ 0 \end{array}}\right)+
			\frac{\minus5}{3} \left({\begin{array}{c} \minus3 \\ 3 \end{array}}\right)=
			\left({\begin{array}{c} 4 \\ \minus5 \end{array}}\right)
		\end{align*}
		\item (5 points) Find a subset of $S$ with the same span as $S$ that is as small as possible. 
		\\
		\textbf{Solution:} \\
		A subset $S$ with the same span as $S$ would be:
		\begin{align*}
			span=\left\{
			\left({\begin{array}{c} 1 \\ 0 \end{array}}\right)_\mathlarger{,}
			\left({\begin{array}{c} 0 \\ 1 \end{array}}\right)
			\right\}
		\end{align*}
		We can get this value by reducing $\left({\begin{array}{c} \minus3 \\ 3 \end{array}}\right)$
		\begin{align*}
			\left({\begin{array}{c} 1 \\ 0 \end{array}}\right)+
			\frac{1}{3} \left({\begin{array}{c} \minus3 \\ 3 \end{array}}\right)=
			\left({\begin{array}{c} 0 \\ 1 \end{array}}\right)
		\end{align*}

	\end{enumerate}
\end{homeworkProblem}
\newpage

%%%%%%%%%%%%%%%%%%%%%%%%%%%%%%%%%%%%%%%%%%%%%%%%%%%%%%%%%%%%%%%%%%%%%%%%%%%%%%%%%
%																				%
%																			    %
%							  PROBLEM 2                                         %
%                                                                               %
%																				%
%%%%%%%%%%%%%%%%%%%%%%%%%%%%%%%%%%%%%%%%%%%%%%%%%%%%%%%%%%%%%%%%%%%%%%%%%%%%%%%%%

\begin{homeworkProblem}
Let $S=\left\{
\left({\begin{array}{c} 1 \\ \minus1 \\ 1 \end{array}}\right)_\mathlarger{,}
\left({\begin{array}{c} 0 \\ 1 \\ 0 \end{array}}\right)_\mathlarger{,}
\left({\begin{array}{c} \minus3 \\ 3 \\ 3 \end{array}}\right)
\right\}$ be a set of vectors in $\mathbb{R}^2$.
	\begin{enumerate}[label=(\alph*)]
	\item (5 points) Determine whether $S$ is linearly independent or linearly dependent. \\
	\textbf{Solution:} \\
	$S$ is linearly independent because no two vectors in the set can create another vector,
	thus making each one linearly independent. 
	\item (5 points) find a subset of $S'$ with the same span as $S$ that is as small as possible. \\
	\textbf{Solution:} \\
	A subset $S'$ with the same span as $S$ that is as small as possible is:
	\begin{align*}
		S=\left\{
		\left({\begin{array}{c} 1 \\ 0 \\ 0 \end{array}}\right)_\mathlarger{,}
		\left({\begin{array}{c} 0 \\ 1 \\ 0 \end{array}}\right)_\mathlarger{,}
		\left({\begin{array}{c} 0 \\ 0 \\ 1 \end{array}}\right)
		\right\}
	\end{align*}
	It is possible to get reduce to this by:
	\begin{align*}
		\left({\begin{array}{c} 1 \\ \minus1 \\ 1 \end{array}}\right)+
		\frac{1}{3}\left({\begin{array}{c} \minus3 \\ 3 \\ 3 \end{array}}\right)=
		\left({\begin{array}{c} 0 \\ 0 \\ 1 \end{array}}\right) \\
		\left({\begin{array}{c} 1 \\ \minus1 \\ 1 \end{array}}\right)+
		\left({\begin{array}{c} 0 \\ 1 \\ 0 \end{array}}\right)+
		-1\left({\begin{array}{c} 0 \\ 0 \\ 1 \end{array}}\right)=
		\left({\begin{array}{c} 1 \\ 0 \\ 0 \end{array}}\right)
	\end{align*}
	
	\item (10 points) Determine whether the vector $\left({\begin{array}{c} 1 \\ 0 \\ 1 \end{array}}\right)$ belongs to $SpanS'$. \\
	\textbf{Solution:} \\
	\tab The vector does belong to $Span S$. This can be calculated by:
	\begin{align*}
		\left({\begin{array}{c} 1 \\ 0 \\ 0 \end{array}}\right) + 
		\left({\begin{array}{c} 0 \\ 0 \\ 1 \end{array}}\right) = 
		\left({\begin{array}{c} 1 \\ 0 \\ 1 \end{array}}\right)
	\end{align*}
	Because we are able to get the value from other values in the span it belongs to Span S. 
	\end{enumerate}
\end{homeworkProblem}
\newpage

%%%%%%%%%%%%%%%%%%%%%%%%%%%%%%%%%%%%%%%%%%%%%%%%%%%%%%%%%%%%%%%%%%%%%%%%%%%%%%%%%
%																				%
%																			    %
%							  PROBLEM 3                                         %
%                                                                               %
%																				%
%%%%%%%%%%%%%%%%%%%%%%%%%%%%%%%%%%%%%%%%%%%%%%%%%%%%%%%%%%%%%%%%%%%%%%%%%%%%%%%%%

\begin{homeworkProblem}
Determine whether the following sets are linearly dependent or linearly independent.
	\begin{enumerate}[label=(\alph*)]
		\item (5 points) $S_1=\left\{\left({\begin{array}{c} \minus2 \\ 3 \\ 0 \\ 7 \end{array}}\right)\right\}$ \\
		\textbf{Solution:} \\
		This is linearly independent because being that it is the only vector
		in the set it cannot express linear dependence. 
		\item (5 points) $S_2=\left\{
		\left({\begin{array}{c} \minus1 \\ 3 \\ \minus8 \end{array}}\right)_\mathlarger{,}
		\left({\begin{array}{c} 0 \\ 0 \\ 0 \end{array}}\right)_\mathlarger{,}
		\left({\begin{array}{c} 4 \\ \minus5 \\ 6 \end{array}}\right)
		\right\}$ \\
		\textbf{Solution:} \\
		This is linearly dependent because I can use scalar multiplication to get the
		0 vector from one of the vectors in the set. 
		\item (10 points) $S_3=\left\{
		\left({\begin{array}{c} \minus3 \\ 7 \\ 2 \end{array}}\right)_\mathlarger{,}
		\left({\begin{array}{c} 0 \\ 0 \\ \minus1 \end{array}}\right)_\mathlarger{,}
		\left({\begin{array}{c} 0 \\ 2 \\ 1 \end{array}}\right)
		\right\}$ \\
		\textbf{Solution:} \\
		This set of vectors is linearly independent as no one vector can be created
		from the linear combination of the other vectors.  
	\end{enumerate}
\end{homeworkProblem}
\newpage

%%%%%%%%%%%%%%%%%%%%%%%%%%%%%%%%%%%%%%%%%%%%%%%%%%%%%%%%%%%%%%%%%%%%%%%%%%%%%%%%%
%																				%
%																			    %
%							  PROBLEM 4                                         %
%                                                                               %
%																				%
%%%%%%%%%%%%%%%%%%%%%%%%%%%%%%%%%%%%%%%%%%%%%%%%%%%%%%%%%%%%%%%%%%%%%%%%%%%%%%%%%

\begin{homeworkProblem}
	Consider the following subset of $\mathbb{R}^4$: 
	\begin{align*}
		V=\left\{
		\left({\begin{array}{c} x_1 \\ x_2 \\ x_3 \\ x_4 \end{array}}\right)
		\in \mathbb{R}^4:x_1+11x_3+4x_4=0\text{ and }x_2+9x_3+3x_4=0
		\right\}.
	\end{align*}
	
	\begin{enumerate}[label=(\alph*)]
		\item (10 pts) Show that $V$ is a subspace of $\mathbb{R}^4$. \\
			\textbf{Solution:} \\
			Take an arbitrary $X$ in $V$, we want to show $cX$ is also in $V$.
			\begin{align*}
				cx_1+11cx_3+4cx_4=0\\
				c(x_1+11x_3+4x_4)=0\\
			\end{align*}
			Because $X$ is in $V$,
			\begin{align*}
				c(0)=0
			\end{align*}
			The same is true for the second equation given:
			\begin{align*}
				cx_2+9cx_3+3cx_4=0 \\
				c(x_2+9x_3+3x_4)=0 \\
				c(0)=0
			\end{align*}
			Therefore our new vector $cX$ satisfies both equations. Next is to prove that the
			set is closed under vector addition. This can be shown by:
			\begin{align*}
				x_1+y_1+11x_3+11y_3+4x_4+4y_4=0 \\
				(x_1+11x_3+4x_4)+(y_1+11x_3+4x_4)=0 \\
			\end{align*}
			Because $x$ and $y$ are in the set:
			\begin{align*}
				(0)+(0)=0
			\end{align*}
			This is also true for the next equation:
			\begin{align*}
				x_2+y_2+9x_3+9y_3+3x_4+3y_4=0 \\
				(x_2+9x_3+3x_4)+(y_2+9y_3+3y_4)=0 \\
				(0)+(0)=0
			\end{align*}
			
		\item (5 pts) Find the dimension of $V$. \\
			\textbf{Solution:} 
			\begin{align*}
				x_3 = s,\text{ }
				x_4 = t \\
				x_1+11s+4t=0\\
				x_1=\minus11s-4t\\
				x_2+9s+3t=0 \\
				x_2=\minus9s-3t \\
				\left({\begin{array}{c} \minus11s-4t \\ \minus9s-3t \\ s \\
				t \end{array}}\right) \\
				= s\left({\begin{array}{c} \minus11 \\ \minus9 \\ 1 \\
				0 \end{array}}\right) + t\left({\begin{array}{c} \minus4 \\ \minus3 \\ 0 \\
				1 \end{array}}\right) \\
				=\left\{
				\left({\begin{array}{c} \minus11 \\ \minus9 \\ 1 \\
				0 \end{array}}\right)_\mathlarger{,}
				\left({\begin{array}{c} \minus4 \\ \minus3 \\ 0 \\
				1 \end{array}}\right)
				\right\}
			\end{align*}
		\item (10 pts) Determine whether the set $B= \left\{
			\left({\begin{array}{c} 1 \\ 0 \\ 1 \\ \minus3 \end{array}}\right)_\mathlarger{,}
			\left({\begin{array}{c} 2 \\ 3 \\ \minus2 \\ 5 \end{array}}\right)_\mathlarger{,}
			\right\}$ is a basis for the subspace $V$. \\
			\textbf{Solution:} \\
			Yes, the set $B$ is a basis for the subspace $V$. This can be proven by:
			\begin{align*}
				\left({\begin{array}{c} \minus 11 \\ \minus 9 \\ 1 \\ 0 \end{array}}\right) -
				3\left({\begin{array}{c} \minus 4 \\ \minus 3 \\ 0 \\ 1 \end{array}}\right) =
				\left({\begin{array}{c} 1 \\ 0 \\ 1 \\ \minus 3 \end{array}}\right) \\
				\minus \left({\begin{array}{c} \minus 4 \\ \minus 3 \\ 0 \\ 1 \end{array}}\right)-
				2\left({\begin{array}{c} 1 \\ 0 \\ 1 \\ \minus 3 \end{array}}\right)=
				\left({\begin{array}{c} 2 \\ 3 \\ \minus2 \\ 5 \end{array}}\right)				
			\end{align*}
			
		\item (5 pts) Can the set $S= \left\{
			\left({\begin{array}{c} 1 \\ 0 \\ 1 \\ \minus3 \end{array}}\right)_\mathlarger{,}
			\left({\begin{array}{c} 2 \\ 3 \\ \minus2 \\ 5 \end{array}}\right)_\mathlarger{,}
			\left({\begin{array}{c} 1 \\ \minus1 \\ 2 \\ 4 \end{array}}\right)
			\right\}$ is a basis for the subspace $V$? \\
			\textbf{Solution:} \\
			No, it is not a basis. It is impossible for the last vector in $S$ to exist within the
			set, it can not be created by the other vectors and describes values that you are not
			supposed to be able to attain from the original description of $V$.
	\end{enumerate}
\end{homeworkProblem}
\newpage

%%%%%%%%%%%%%%%%%%%%%%%%%%%%%%%%%%%%%%%%%%%%%%%%%%%%%%%%%%%%%%%%%%%%%%%%%%%%%%%%%
%																				%
%																			    %
%							  PROBLEM 5                                         %
%                                                                               %
%																				%
%%%%%%%%%%%%%%%%%%%%%%%%%%%%%%%%%%%%%%%%%%%%%%%%%%%%%%%%%%%%%%%%%%%%%%%%%%%%%%%%%

\begin{homeworkProblem}
	Consider the following subset of $\mathbb{R}^4$: 
	\begin{align*}
		W=\left\{
		\left({\begin{array}{c} x_1 \\ x_2 \\ x_3 \\ x_4 \end{array}}\right)
		\in \mathbb{R}^4:x_1+x_3+2x_4=0\text{ and }x_2-x_3-x_4=0
		\right\}
	\end{align*}
	
	\begin{enumerate}[label=(\alph*)]
		\item (10 pts) Show that $W$ is a subspace of $\mathbb{R}^4$. \\
			\textbf{Solution:} \\
			\textbf{Solution:} \\
			Take an arbitrary $X$ in $V$, we want to show $cX$ is also in $V$.
			\begin{align*}
			cx_1+cx_3+2cx_4=0\\
			c(x_1+x_3+2x_4=0)=0\\
			\end{align*}
			Because $X$ is in $V$,
			\begin{align*}
			c(0)=0
			\end{align*}
			The same is true for the second equation given:
			\begin{align*}
			cx_2-cx_3-cx_4=0 \\
			c(x_2-x_3-x_4)=0 \\
			c(0)=0
			\end{align*}
			Therefore our new vector $cX$ satisfies both equations. Next is to prove that the
			set is closed under vector addition. This can be shown by:
			\begin{align*}
			x_1+y_1+x_3+y_3+2x_4+2y_4=0 \\
			(x_1+x_3+2x_4)+(y_1+x_3+2x_4)=0 \\
			\end{align*}
			Because $x$ and $y$ are in the set:
			\begin{align*}
			(0)+(0)=0
			\end{align*}
			This is also true for the next equation:
			\begin{align*}
			x_2-y_2-x_3-y_3-x_4-y_4=0 \\
			(x_2-x_3-x_4)+(y_2-y_3-y_4)=0 \\
			(0)+(0)=0
			\end{align*}
			
		\item (5 pts) Find a basis for $W$. What is the dimension of $W$?\\
			\textbf{Solution:} \\
			\begin{align*}
				x_3 = s,\text{ }
				x_4 = t \\
				x_1+s+2t=0\\
				x_1=\minus s-2t\\
				x_2-s-t=0 \\
				x_2=+s+t \\
				\left({\begin{array}{c} \minus s-2t \\ s+t \\ s \\
				t \end{array}}\right) \\
				= s\left({\begin{array}{c} \minus 1 \\ 1 \\ 1 \\
				0 \end{array}}\right) + t\left({\begin{array}{c} \minus2 \\ 1 \\ 0 \\
				1 \end{array}}\right) \\
				=\left\{
				\left({\begin{array}{c} \minus 1 \\ 1 \\ 1 \\
				0 \end{array}}\right)_\mathlarger{,}
				\left({\begin{array}{c} \minus 2 \\ 1 \\ 0 \\
				1 \end{array}}\right)
				\right\}				
			\end{align*}
		\item (5 pts) Determine whether the set $B= \left\{
			\left({\begin{array}{c} \minus 3 \\ 2 \\ 1 \\ 1 \end{array}}\right)_\mathlarger{,}
			\left({\begin{array}{c} \minus 2 \\ 1 \\ 0 \\ 1 \end{array}}\right)
			\right\}$ is a basis for the subspace $W$. \\
			\textbf{Solution:} \\
			Yes it is. It is a basis because you can get the set $B$ from:
			\begin{align*}
				\left({\begin{array}{c} \minus 2 \\ 1 \\ 0 \\ 1 \end{array}}\right)+
				\left({\begin{array}{c} \minus 1 \\ 1 \\ 1 \\ 0 \end{array}}\right)=
				\left({\begin{array}{c} \minus 3 \\ 2 \\ 1 \\ 1 \end{array}}\right)
			\end{align*}
			Because it is a set of linearly independent vectors that can describe the same set.
			
		\item (5 pts) Determine whether the set $S= \left\{
			\left({\begin{array}{c} \minus 1 \\ 1 \\ 1 \\ 0 \end{array}}\right)_\mathlarger{,}
			\left({\begin{array}{c} \minus 6 \\ 4 \\ 2 \\ 2 \end{array}}\right)_\mathlarger{,}
			\left({\begin{array}{c} \minus 2 \\ 1 \\ 0 \\ 1 \end{array}}\right)
			\right\}$ is a basis for the subspace $W$. \\
			\textbf{Solution:} \\
			No, although the entire set can describe the same set the vectors
			linearly dependent.
			\begin{align*}
				\left({\begin{array}{c} \minus 2 \\ 1 \\ 0 \\ 1 \end{array}}\right)+
				\left({\begin{array}{c} \minus 1 \\ 1 \\ 1 \\ 0 \end{array}}\right)=
				\left({\begin{array}{c} \minus 3 \\ 2 \\ 1 \\ 1 \end{array}}\right) \\
				2\left({\begin{array}{c} \minus 3 \\ 2 \\ 1 \\ 1 \end{array}}\right)=
				\left({\begin{array}{c} \minus 6 \\ 4 \\ 2 \\ 2 \end{array}}\right)
			\end{align*}
	\end{enumerate}
\end{homeworkProblem}
\newpage

\end{document}