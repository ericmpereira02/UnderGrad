\documentclass{article}

\usepackage{fancyhdr}
\usepackage{extramarks}
\usepackage{amsmath}
\usepackage{amsthm}
\usepackage{amsfonts}
\usepackage{tikz}
\usepackage[plain]{algorithm}
\usepackage{algpseudocode}
\usepackage{amssymb}
\usepackage{enumitem}
\usepackage{relsize}
\usepackage{textcomp}
\usepackage{soul}

\usetikzlibrary{automata,positioning}

%
% Basic Document Settings
%

\topmargin=-0.45in
\evensidemargin=0in
\oddsidemargin=0in
\textwidth=6.5in
\textheight=9.0in
\headsep=0.25in

\linespread{1.1}

\pagestyle{fancy}
\lhead{\hmwkAuthorName}
\chead{\hmwkClass\ (\hmwkClassInstructor\ \hmwkClassTime): \hmwkTitle}
\rhead{\firstxmark}
\lfoot{\lastxmark}
\cfoot{\thepage}

\def\doubleunderline#1{\underline{\underline{#1}}}
\newcommand{\minus}{\scalebox{0.5}[1.0]{$-$}}
\newcommand\tab[1][1cm]{\hspace*{#1}}
\renewcommand\headrulewidth{0.4pt}
\renewcommand\footrulewidth{0.4pt}
\renewcommand{\theenumi}{\Alph{enumi}}

\setlength\parindent{0pt}

%
% Create Problem Sections
%

\newcommand{\enterProblemHeader}[1]{
    \nobreak\extramarks{}{Problem \arabic{#1} continued on next page\ldots}\nobreak{}
    \nobreak\extramarks{Problem \arabic{#1} (continued)}{Problem \arabic{#1} continued on next page\ldots}\nobreak{}
}

\newcommand{\exitProblemHeader}[1]{
    \nobreak\extramarks{Problem \arabic{#1} (continued)}{Problem \arabic{#1} continued on next page\ldots}\nobreak{}
    \stepcounter{#1}
    \nobreak\extramarks{Problem \arabic{#1}}{}\nobreak{}
}

\setcounter{secnumdepth}{0}
\newcounter{partCounter}
\newcounter{homeworkProblemCounter}
\setcounter{homeworkProblemCounter}{1}
\nobreak\extramarks{Problem \arabic{homeworkProblemCounter}}{}\nobreak{}

%
% Homework Problem Environment
%
% This environment takes an optional argument. When given, it will adjust the
% problem counter. This is useful for when the problems given for your
% assignment aren't sequential. See the last 3 problems of this template for an
% example.
%
\newenvironment{homeworkProblem}[1][-1]{
    \ifnum#1>0
        \setcounter{homeworkProblemCounter}{#1}
    \fi
    \section{Problem \arabic{homeworkProblemCounter}}
    \setcounter{partCounter}{1}
    \enterProblemHeader{homeworkProblemCounter}
}{
    \exitProblemHeader{homeworkProblemCounter}
}

%
% Homework Details
%   - Title
%   - Due date
%   - Class
%   - Section/Time
%   - Instructor
%   - Author
%

\newcommand{\hmwkTitle}{Homework\ \#5-2}
\newcommand{\hmwkDueDate}{April 14, 2019}
\newcommand{\hmwkClass}{Linear Algebra}
\newcommand{\hmwkClassTime}{Section 01}
\newcommand{\hmwkClassInstructor}{Dr. Subasi}
\newcommand{\hmwkAuthorName}{\textbf{Eric Pereira}}

%
% Title Page
%
\pagenumbering{gobble}
\title{
    \vspace{2in}
    \textmd{\textbf{\hmwkClass:\ \hmwkTitle}}\\
    \normalsize\vspace{0.1in}\small{Due\ on\ \hmwkDueDate\ at 11:59pm}\\
    \vspace{0.1in}\large{\textit{\hmwkClassInstructor\ \hmwkClassTime}}
    \vspace{3in}
}

\author{\hmwkAuthorName}
\date{}

\renewcommand{\part}[1]{\textbf{\large Part \Alph{partCounter}}\stepcounter{partCounter}\\}

%
% Various Helper Commands
%

% Useful for algorithms
\newcommand{\alg}[1]{\textsc{\bfseries \footnotesize #1}}

% For derivatives
\newcommand{\deriv}[1]{\frac{\mathrm{d}}{\mathrm{d}x} (#1)}

% For partial derivatives
\newcommand{\pderiv}[2]{\frac{\partial}{\partial #1} (#2)}

% Integral dx
\newcommand{\dx}{\mathrm{d}x}

% Alias for the Solution section header
\newcommand{\solution}{\textbf{\large Solution}}

% Probability commands: Expectation, Variance, Covariance, Bias
\newcommand{\E}{\mathrm{E}}
\newcommand{\Var}{\mathrm{Var}}
\newcommand{\Cov}{\mathrm{Cov}}
\newcommand{\Bias}{\mathrm{Bias}}

\begin{document}

\maketitle

\pagebreak
\pagenumbering{arabic}

%%%%%%%%%%%%%%%%%%%%%%%%%%%%%%%%%%%%%%%%%%%%%%%%%%%%%%%%%%%%%%%%%%%%%%%%%%%%%%%%%
%																				%
%																			    %
%							  PROBLEM 1                                         %
%                                                                               %
%																				%
%%%%%%%%%%%%%%%%%%%%%%%%%%%%%%%%%%%%%%%%%%%%%%%%%%%%%%%%%%%%%%%%%%%%%%%%%%%%%%%%%

\begin{homeworkProblem}
	Consider the following $3\times 3$ matrix:
	\begin{align*}
		A=\left({\begin{array}{ccc} 1&\minus 2&2 \\ \minus 2&4&\minus 4 \\ 2&\minus 4&4
		\end{array}}\right)
	\end{align*}
	\begin{enumerate}[label=(\alph*)]
		\item (10pts) Find eigenvalues of $A$. \\
		\textbf{Solution:} \\
		\begin{align*}
			det(A-\lambda I)=0 
		\end{align*}
		\begin{align*}
			A-\lambda I=\left({\begin{array}{ccc} 1&\minus 2&2 \\ \minus 2&4&\minus 4 \\ 2&\minus 4&4
			\end{array}}\right)-
			\left({\begin{array}{ccc} \lambda&0&0 \\ 0&\lambda&0 \\ 0&0&\lambda \end{array}}\right)=
			\left({\begin{array}{ccc} 1-\lambda&\minus 2&2 \\ \minus 2&4-\lambda&\minus 4 \\
			 2&\minus 4&4-\lambda \end{array}}\right) \\
			 det(A-\lambda I)=(1-\lambda)
			 \left({\begin{array}{cc} 4-\lambda&\minus 4 \\ \minus 4&4-\lambda \end{array}}\right)-
			 (\minus 2)\left({\begin{array}{cc} \minus 2&\minus 4 \\ 2&4-\lambda \end{array}}\right)+
			 (2)\left({\begin{array}{cc} \minus 2&4-\lambda \\ 2&\minus 4 \end{array}}\right) \\
			 =(1-\lambda)[(4-\lambda)(4-\lambda)-(\minus 4)(\minus4)]+
			 (2)[(\minus 2)(4-\lambda)-(\minus 4)(2)]+
			 (2)[(\minus 2)(\minus 4)-(4-\lambda)(2)] \\
			 =(1-\lambda)(16-8\lambda+\lambda^2-16)+
			 (2)(\minus 8+2\lambda+ 8)+
			 (2)(8-(8-2\lambda))\\
			 =(1-\lambda)(\lambda^2-8\lambda)+
			 (2)(2\lambda)+
			 (2)(2\lambda)\\
			 =(\minus \lambda^3+9\lambda^2-8\lambda)+(4\lambda)+(4\lambda) \\
			 =\minus \lambda^3+9\lambda^2-8\lambda+8\lambda \\
			 =\minus \lambda^3+9\lambda^2 \\
			 =-\lambda^2(\lambda-9)
		\end{align*}	
		When solving for the determinant the solutions are the eigenvalues. This means any case where
		$-\lambda^2(\lambda-9)$ is equal to 0 is the solution. Therefore, the eigenvalues must be 9 and
		0.			
			
		\item (10pts) Find the eigenspaces of $A$ corresponding to the eigenvalues found in
		Part (a). \ul{Justify your answer.} \\
		\textbf{Solution:} \\
	
		\item (5pts) Determine whether $A$ is diagonalizable. \ul{Justify your answer.} \\
		\textbf{Solution:} \\
		
		\item (10pts) Find an invertible matrix $P$ and a diagonal matrix $D$ such that
		$P^{\minus1AP=D}.$ \ul{No need to verify the equation holds true.}\\
		\textbf{Solution:} \\
		
		\item (10pts) Find a matrix $S$ whose columns form an orthonormal basis for $\mathbb{R}^3$
		obtained from the columns of matrix $P$ found in part (d). \\
		\textbf{Solution:} \\
		
		\item (5pts) Is $S$ invertible? If yes, find $S^{\minus1}$.
		\textbf{Solution:} \\
		
		\item (5pts) Show that $S^{\minus1}AS=D$, where $D$ is the diagonal matrix found in Part
		(d), that is, $A$ is orthogonally diagonlizable. \\
		\textbf{Solution:} \\
		
	\end{enumerate}
\end{homeworkProblem}
\newpage

%%%%%%%%%%%%%%%%%%%%%%%%%%%%%%%%%%%%%%%%%%%%%%%%%%%%%%%%%%%%%%%%%%%%%%%%%%%%%%%%%
%																				%
%																			    %
%							  PROBLEM 2                                         %
%                                                                               %
%																				%
%%%%%%%%%%%%%%%%%%%%%%%%%%%%%%%%%%%%%%%%%%%%%%%%%%%%%%%%%%%%%%%%%%%%%%%%%%%%%%%%%

\begin{homeworkProblem}
	Let $A$ and $B$ be two $n \times n$ orthogonal matrices. Answer the following questions:
	\begin{enumerate}[label=(\alph*)]
		\item (5pts) Is $\minus 3A$ orthogonal? \ul{Justify your answer.} \\
		\textbf{Solution:} \\
		Yes, orthogonal matrices remain orthogonal under scalar multiplication
		\item (5pts) Is $\minus B$ orthogonal? \ul{Justify your answer.} \\
		\textbf{Solution:} \\
		Yes, orthogonal matrices remain orthogonal under scalar multiplication
		\item (5pts) Is $B^{\minus 1}AB$ orthogonal? \ul{Justify your answer.} \\
		\textbf{Solution:} \\
		Yes, orthogonal matrices are closed under multiplication. 
		\item (5pts) Is $A^T$ invertible? If yes, find its inverse. \\
		\textbf{Solution:} \\
		Yes, to explain further, the transpose of an orthogonal matrix is the original matrix's
		inverse. This means that $A^T=A^{\minus 1}$, and the inverse of $A^{\minus 1}$ is just $A$. 
		\item (5pts) Is $AB$ orthogonal? \ul{Justify your answer.} \\
		\textbf{Solution:} \\
		Yes, orthogonal matrices are closed under multiplication. 
	\end{enumerate}
	
\end{homeworkProblem}
\newpage

%%%%%%%%%%%%%%%%%%%%%%%%%%%%%%%%%%%%%%%%%%%%%%%%%%%%%%%%%%%%%%%%%%%%%%%%%%%%%%%%%
%																				%
%																			    %
%							  PROBLEM 3                                         %
%                                                                               %
%																				%
%%%%%%%%%%%%%%%%%%%%%%%%%%%%%%%%%%%%%%%%%%%%%%%%%%%%%%%%%%%%%%%%%%%%%%%%%%%%%%%%%

\begin{homeworkProblem}
	Suppose that $A$ is an $n\times n$ orthogonal matrix.
	\begin{enumerate}[label=(\alph*)]
		\item (10pts) Show that the matrix operator $T:\mathbb{R}^n \rightarrow \mathbb{R}^n,T(x)=Ax$
		is an orthogonal operator. \\
		\textbf{Solution:} \\
		
		\item (10pts) Prove that the only eigenvalues of $A$ are 1 and $\minus$1 \\
		\textbf{Solution:} \\
		
	\end{enumerate}
\end{homeworkProblem}
\newpage

%%%%%%%%%%%%%%%%%%%%%%%%%%%%%%%%%%%%%%%%%%%%%%%%%%%%%%%%%%%%%%%%%%%%%%%%%%%%%%%%%
%																				%
%																			    %
%							  PROBLEM 4                                         %
%                                                                               %
%																				%
%%%%%%%%%%%%%%%%%%%%%%%%%%%%%%%%%%%%%%%%%%%%%%%%%%%%%%%%%%%%%%%%%%%%%%%%%%%%%%%%%

\begin{homeworkProblem}
	(15pts) Let $V$ be a subspace of $\mathbb{R}^n$. The orthogonal complement of 
	$V$ in $\mathbb{R}^n$ is defined as
	\begin{align*}
		V^\bot=\{x\in \mathbb{R}^n|x\cdot v=0\;\forall \; v\in V \}
	\end{align*}
	Prove that $V^\bot$ is also a subspace of $\mathbb{R}^n$.
\end{homeworkProblem}
\newpage

%%%%%%%%%%%%%%%%%%%%%%%%%%%%%%%%%%%%%%%%%%%%%%%%%%%%%%%%%%%%%%%%%%%%%%%%%%%%%%%%%
%																				%
%																			    %
%							  PROBLEM 5                                         %
%                                                                               %
%																				%
%%%%%%%%%%%%%%%%%%%%%%%%%%%%%%%%%%%%%%%%%%%%%%%%%%%%%%%%%%%%%%%%%%%%%%%%%%%%%%%%%

\begin{homeworkProblem}
	Let $u_1=\left({\begin{array}{c} 0 \\ 1 \\ 0 \end{array}}\right)$ and
	$u_2=\left({\begin{array}{c} 1 \\ 1 \\ 0 \end{array}}\right)$
	\begin{enumerate}[label=(\alph*)]
		\item (5pts) Determine whether the set $S=\{u_1,u_2\}$ is orthogonal. \\
		\textbf{Solution:} \\
		
		\item (5pts) Transform $S$ into an orthonormal set of vectors $B$. \\
		\textbf{Solution:} \\
		
		\item (5pts) Find the orthogonal projection of $u=\left({\begin{array}{c} 1 \\ 2 \\ 3
		\end{array}}\right)$ on the subspace $W=SpanB$, where $B$ is the orthornormal basis
		obtained in part (b). \\
		
		%gram schmidt process explanation 
		\noindent\rule{16cm}{0.4pt} 
		\doubleunderline{Gram-Schmidt Process:} if $\{u_1,u_2,...,u_k\}$ is a basis for
		a subspace $W$, then $\{v_1,v_2,...,v_k\}$ is an orthogonal basis for $W$, where
		\begin{align*}
		v_1=u_1, \; v_2=u_2-\frac{u_2\cdot v_1}{||v_1||^2}v_1,\;...,\;v_k=u_k-
		\frac{u_k\cdot v_1}{||v_1||^2}v_1-\;...\;-\frac{u_k\cdot v_{k-1}}{||v_{k-1}||^2}v_{k-1}
		\end{align*}
		Also, if $\{w_1,w_2,...w_k \}$ is an orthornormal basis, then the orthogonal projection
		, $w$, of a vector $u$ on a subspace $W$ is given by
		\begin{align*}
			w=(u\cdot w_1)w_1+(u\cdot w_2)w_2+...+(u\cdot w_k)w_k
		\end{align*}
		\noindent\rule{16cm}{0.4pt} \\
		%%%%%%%%%%%%%%%%%%%%
		
		\textbf{Solution:} \\
		
	\end{enumerate}
\end{homeworkProblem}

%%%%%%%%%%%%%%%%%%%%%%%%%%%%%%%%%%%%%%%%%%%%%%%%%%%%%%%%%%%%%%%%%%%%%%%%%%%%%%%%%
%																				%
%																			    %
%							  PROBLEM 6                                         %
%                                                                               %
%																				%
%%%%%%%%%%%%%%%%%%%%%%%%%%%%%%%%%%%%%%%%%%%%%%%%%%%%%%%%%%%%%%%%%%%%%%%%%%%%%%%%%

\begin{homeworkProblem}
	(20pts) Classify each of the following statements as True or False. \ul{Justify your answers}.
	\begin{enumerate}[label=(\alph*)]
		\item The determinant of an orthogonal matrix $A$ is equal to 1. \\
		\textbf{Solution:} \\
		
		\item The projection matrix $P$ defined as $P=A(A^TA)^{\minus1}A^T$ is
		symmetric for all A. \\
		\textbf{Solution:} \\
		
		\item Every non-zero subspace $W$ of $\mathbb{R}^n$ has an orthonormal basis. \\
		\textbf{Solution:} \\
		
		\item Any two bases in a fininte-dimensional vector space $V$ have the same number of
		elements. \\
		\textbf{Solution:} \\
		
	\end{enumerate}
\end{homeworkProblem}
\newpage
\end{document}0