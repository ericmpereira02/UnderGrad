\documentclass{article}


\usepackage{fancyhdr}
\usepackage{extramarks}
\usepackage{amsmath}
\usepackage{amsthm}
\usepackage{amsfonts}
\usepackage{tikz}
\usepackage[plain]{algorithm}
\usepackage{algpseudocode}
\usepackage{amssymb}
\usepackage{enumitem}

\usetikzlibrary{automata,positioning}

%
% Basic Document Settings
%

\topmargin=-0.45in
\evensidemargin=0in
\oddsidemargin=0in
\textwidth=6.5in
\textheight=9.0in
\headsep=0.25in

\linespread{1.1}

\pagestyle{fancy}
\lhead{\hmwkAuthorName}
\chead{\hmwkClass\ (\hmwkClassInstructor\ \hmwkClassTime): \hmwkTitle}
\rhead{\firstxmark}
\lfoot{\lastxmark}
\cfoot{\thepage}

\newcommand\tab[1][1cm]{\hspace*{#1}}
\renewcommand\headrulewidth{0.4pt}
\renewcommand\footrulewidth{0.4pt}
\renewcommand{\theenumi}{\Alph{enumi}}

\setlength\parindent{0pt}

%
% Create Problem Sections
%

\newcommand{\enterProblemHeader}[1]{
    \nobreak\extramarks{}{Problem \arabic{#1} continued on next page\ldots}\nobreak{}
    \nobreak\extramarks{Problem \arabic{#1} (continued)}{Problem \arabic{#1} continued on next page\ldots}\nobreak{}
}

\newcommand{\exitProblemHeader}[1]{
    \nobreak\extramarks{Problem \arabic{#1} (continued)}{Problem \arabic{#1} continued on next page\ldots}\nobreak{}
    \stepcounter{#1}
    \nobreak\extramarks{Problem \arabic{#1}}{}\nobreak{}
}

\setcounter{secnumdepth}{0}
\newcounter{partCounter}
\newcounter{homeworkProblemCounter}
\setcounter{homeworkProblemCounter}{1}
\nobreak\extramarks{Problem \arabic{homeworkProblemCounter}}{}\nobreak{}

%
% Homework Problem Environment
%
% This environment takes an optional argument. When given, it will adjust the
% problem counter. This is useful for when the problems given for your
% assignment aren't sequential. See the last 3 problems of this template for an
% example.
%
\newenvironment{homeworkProblem}[1][-1]{
    \ifnum#1>0
        \setcounter{homeworkProblemCounter}{#1}
    \fi
    \section{Problem \arabic{homeworkProblemCounter}}
    \setcounter{partCounter}{1}
    \enterProblemHeader{homeworkProblemCounter}
}{
    \exitProblemHeader{homeworkProblemCounter}
}

%
% Homework Details
%   - Title
%   - Due date
%   - Class
%   - Section/Time
%   - Instructor
%   - Author
%

\newcommand{\hmwkTitle}{Homework\ \#2}
\newcommand{\hmwkDueDate}{March 24, 2019}
\newcommand{\hmwkClass}{Linear Algebra}
\newcommand{\hmwkClassTime}{Section 01}
\newcommand{\hmwkClassInstructor}{Dr. Subasi}
\newcommand{\hmwkAuthorName}{\textbf{Eric Pereira}}

%
% Title Page
%
\pagenumbering{gobble}
\title{
    \vspace{2in}
    \textmd{\textbf{\hmwkClass:\ \hmwkTitle}}\\
    \normalsize\vspace{0.1in}\small{Due\ on\ \hmwkDueDate\ at 11:59pm}\\
    \vspace{0.1in}\large{\textit{\hmwkClassInstructor\ \hmwkClassTime}}
    \vspace{3in}
}

\author{\hmwkAuthorName}
\date{}

\renewcommand{\part}[1]{\textbf{\large Part \Alph{partCounter}}\stepcounter{partCounter}\\}

%
% Various Helper Commands
%

% Useful for algorithms
\newcommand{\alg}[1]{\textsc{\bfseries \footnotesize #1}}

% For derivatives
\newcommand{\deriv}[1]{\frac{\mathrm{d}}{\mathrm{d}x} (#1)}

% For partial derivatives
\newcommand{\pderiv}[2]{\frac{\partial}{\partial #1} (#2)}

% Integral dx
\newcommand{\dx}{\mathrm{d}x}

% Alias for the Solution section header
\newcommand{\solution}{\textbf{\large Solution}}

% Probability commands: Expectation, Variance, Covariance, Bias
\newcommand{\E}{\mathrm{E}}
\newcommand{\Var}{\mathrm{Var}}
\newcommand{\Cov}{\mathrm{Cov}}
\newcommand{\Bias}{\mathrm{Bias}}

\begin{document}

\maketitle

\pagebreak
\pagenumbering{arabic}

%
%	Problem 1
%
\begin{homeworkProblem}
	\tab Determine whether the given system of linear equations is consistent, and if so find its general solution.
	\begin{enumerate}[label=\alph*]
		\item System 1:
		\begin{align*}
			x_1+3x_2+x_3+x_4=-1 \\
			-2x_1-6x_2-x_3=5 \\
		x_1+3x_2+2x_3+3x_4=2
		\end{align*}
		\textbf{Solution} \\
		\begin{align*}
			\left({\begin{array}{cccc|c} 1&3&1&1&-1 \\ -2&-6&-1&0&5
			\\ 1&3&2&3&2 \end{array}}\right)\\
			R_2-(-2R_1)\rightarrow R_2
			\left({\begin{array}{cccc|c} 1&3&1&1&-1 \\ 0&0&1&2&3
			\\ 1&3&2&3&2 \end{array}}\right)\\
			(R_3)-(R_1)\rightarrow R_3	
			\left({\begin{array}{cccc|c} 1&3&1&1&-1 \\ 0&0&1&2&3
			\\ 0&0&1&2&3 \end{array}}\right)\\
			(R_3)-(R_2)\rightarrow R_3
			\left({\begin{array}{cccc|c} 1&3&1&1&-1 \\ 0&0&1&2&3
			\\ 0&0&0&0&0 \end{array}}\right)\\
			(R_1)-(R_2)\rightarrow R_1
			\left({\begin{array}{cccc|c} 1&3&0&-1&-4 \\ 0&0&1&2&3
			\\ 0&0&0&0&0 \end{array}}\right)\\ \\
			x_1+3x_2-x_4=-4\\
			x_3+2x_4=3\\
			x_3=-2x_4+3\\
			x_1=-3x_2+x_4-4\\ \\
			X=\left({\begin{array}{c} -3x_2+x_4-4 \\ x_2
			\\ -2x_4+3 \\ x_4 \end{array}}\right)
		\end{align*}
		
		
		
		\item System 2:
		\begin{align*}
		x_1-2x_2-5x_3=4 \\
		x_2+3x_3=-2 \\
		-x_2-3x_3=3		
		\end{align*}
		\textbf{Solution} \\
		\begin{align*}
			\left({\begin{array}{ccc|c} 1&-2&-5&4 \\ 0&1&3&-2 \\
			0&-1&-3&3 \end{array}}\right) \\
			R_2+R_3\rightarrow R_2 \left({\begin{array}{ccc|c}
			1&-2&-5&4 \\ 0&0&0&1 \\ 0&-1&-3&3 
			\end{array}}\right) \\
		\end{align*}
		\tab This is not consistent after this operation, as 
		$0x_1+0x_2+0x_3=-1$ is not possible. 
	\end{enumerate}
	
\end{homeworkProblem}
\newpage

%
%	Problem 2
%
\begin{homeworkProblem}
\tab Consider the following system of linear equations:
	\begin{align*}
		x_1+3x_2=1+s \\
		x_1+rx_2=5
	\end{align*}
	\begin{enumerate}[label=\alph*]
		\item (5 points) For what values of $r$ and $s$ is the system
			inconsistent? \\
			\textbf{Solution} \\
			\begin{align*}
			\left({\begin{array}{cc|c} 1&3&1+s \\ 1&r&5 \end{array}}\right) \\
			R2-R1\rightarrow R2
			\left({\begin{array}{cc|c} 1&3&1+s \\ 0&r-3&4+s \end{array}}\right)
			\end{align*} \\ 
			\tab From here we can start to find use the equation $(r-3)x_2=4+s$
			to find some answers. \\
			\begin{align*}
				(r-3)x_2=4+s \\
				x_2=\frac{4+s}{r-3}
			\end{align*}
			\tab Now from here we can see that the system is inconsistent if
			$r=3$ and $s\neq-4$.

			
		\item 
			(5 points) For what values of $r$ and $s$ does the system have
			infinitely many solutions? \\
			\textbf{Solution} \\
			\tab Using the same equation in 2a we know that $x_2=\frac{4+s}{r-3}
			$. Utilizing this we can determine at what point the system has
			infinitely many solutions. The case in which it has infinitely many
			solutions is when $r=3$  and $s=-4$.
		\item 
			(5 points) For what values of $r$ and $s$ does the system have
			unique solution? \\
			\textbf{Solution} \\
			\tab There is a unique solution when $r\neq3$. 
	\end{enumerate}
\end{homeworkProblem}
\newpage

%
%	Problem 3
%
\begin{homeworkProblem}
	\tab Given the system of linear equations
	\begin{align*}
		x_1+rx_2=5 \\
		-3x_1+6x_2=s
	\end{align*}
	\begin{enumerate}[label=\alph*]
		\item (5 points) Determine the values of $r$ and $s$ so that the system
			is consistent. \\
			\textbf{Solution} \\
			\tab To start lets put the equations in a matrix: \\
			\begin{align*}
			\left({\begin{array}{cc|c} 1&r&5 \\ -3&6&s \end{array}}\right) \\
			R2+3R1\rightarrow R2
			\left({\begin{array}{cc|c} 1&r&5 \\ 0&6+3r&s+15 \end{array}}\right) 
			\end{align*}
			\tab In this case the equation produced will come out to be:
			\begin{align*}
				(6+3r)x_2=s+15 \\
				x_2=\frac{s+15}{6+3r}
			\end{align*}
			\tab The system is consistent for any instance where $r\neq -2$
		\item (5 points) Determine the values of $r$ and $s$ so that the system
			is inconsistent. \\
			\textbf{Solution} \\
			\tab The solution is nearly the opposite of that in 3a. The system 
			is inconsistent for any instance $r=-2$ and $s\neq -15$
		\item (5 points) Determine the values of $r$ and $s$ so that the system
			has infinitely many solutions. \\
			\textbf{Solution} \\
			\tab The system has infinitely many solutions in any case where 
			$r=-2$ and $s=-15$
	\end{enumerate}
\end{homeworkProblem}
\newpage

%
%	Problem 4
%
\begin{homeworkProblem}
	\tab Let A= 
	$\left({\begin{array}{ccc} 1&2&3 \\ 0&1&0 \\ 0&0&-3 \end{array}}\right)$
	\begin{enumerate}[label=\alph*]
		\item (5 points) Find $A^{-1}$, if it exists.\\
			\textbf{Solution}\\
			\begin{align*}
				 \left({\begin{array}{ccc|ccc} 1&2&3&1&0&0 \\ 0&1&0&0&1&0 \\ 
				 0&0&-3&0&0&1 \end{array}}\right) \\
				 R_1+R_3\rightarrow R_1\left({\begin{array}{ccc|ccc} 1&2&0&1&0&1
				 \\ 0&1&0&0&1&0 \\ 0&0&-3&0&0&1 \end{array}}\right)  \\
				 R_1-2R_2\rightarrow R_1 \left({\begin{array}{ccc|ccc} 
				 1&0&0&1&-2&1 \\ 0&1&0&0&1&0 \\ 0&0&-3&0&0&1 \end{array}}\right) 
				 \\
				 -\frac{1}{3}R_3\rightarrow R_3\left({\begin{array}{ccc|ccc} 
				 1&0&0&1&-2&1 \\ 0&1&0&0&1&0 \\ 0&0&1&0&0&-\frac{1}{3} 
				 \end{array}}\right) \\
				 A^{-1}=\left({\begin{array}{ccc} 1&-2&1 \\ 0&1&0 \\ 0&0&
				 \frac{1}{3} \end{array}}\right)
			\end{align*}
		\item (5 points) Determine whether the equation $Ax=b$ is consistent 
			for every $b$ in $\mathbb{R}^3$. If so, what is the solution of the
			system?\\
			\textbf{Solution}\\
			\begin{align*}
				b = \left({\begin{array}{c} b_1 \\ b_2 \\ b_3 \end{array}}
				\right)\\
				\left({\begin{array}{ccc|c} 1&2&3&b_1 \\ 0&1&0&b_2 \\ 
				0&0&-3&b_3 \end{array}}\right) \\
			\end{align*}
			\tab Now, following the steps of 4a we can see that the Reduced
			row echelon form of the equation is the identity matrix. When 
			looking with my matrix for $b$ I put in free variable, where b
			can be anything. Due to it reducing to identity it allows it any
			$x$ because any value will satisfy the b matrix. There are no
			inconsistencies. 
	\end{enumerate}
\end{homeworkProblem}
\newpage

%
%	Problem 5
%
\begin{homeworkProblem}
	\tab Let $A$ be a $3\times4$ matrix whose reduce row echelon form is
	\begin{equation*}
		\hat{A} = 
		\left({\begin{array}{cccc} 1&-2&0&3 \\ 0&0&1&-5 \\ 0&0&0&0
		\end{array}}\right)
	\end{equation*}
	\begin{enumerate}[label=\alph*]
		\item (5 points) Find the general solution $x\in\mathbb{R}^4$ to the 
			system $Ax=0$ \\
			\textbf{Solution}\\
			\begin{align*}
				\left({\begin{array}{cccc|c} 1&-2&0&3&0 \\ 0&0&1&-5&0 \\ 
				0&0&0&0&0 \end{array}}\right) \\
				x_1 - 2x_2 + 3x_4 = 0 \\
				x_3-5x_4=0 \\
				x_3=5x_4 \\
				x_1=2x_2 \\
				x=\left({\begin{array}{c} 2x_2-3x_4 \\ x_2 \\ 5x_4 \\ 
				x_4\end{array}}\right)
			\end{align*}
		\item (5 points) Find the set of all the solutions to $Ax=0$ and write
			it as the span of the linearly independent vectors.\\
			\textbf{Solution}\\
			\begin{align*}
				span\left\{
				\left({\begin{array}{c} 2 \\ 1 \\ 0 \\ 0 \end{array}}\right),
				\left({\begin{array}{c} -3 \\ 0 \\ 5 \\ 1 \end{array}}\right)
				\right\} \\
			\end{align*}
		\item (5 points) True or False: The equation $Ax=b$ has a solution $x\in 
			\mathbb{R}^4$ for every vector $b\in\mathbb{R}^3$. Justify your
			answer.\\
			\textbf{Solution}\\
			\tab False. The reason is because the fourth row is entirely 0's, 
			so in this case if the last element of b is non-zero then it is
			inconsistent.  
		\item (5 points) True or False: If the equation $Ax=b$ has a solution $
			x^*\in\mathbb{R}^4$ for a particular vector $b\in\mathbb{R}^3$, then
			$x^*$ is the unique. Justify your answer.\\
			\textbf{Solution}\\
			\tab False, because the solution for b has a free variable. 
	\end{enumerate}
\end{homeworkProblem}
\newpage

%
%	Problem 6
%

\begin{homeworkProblem}
	\begin{enumerate}[label=\alph*]
		\item (5 points) True or False: If $A$ and $B$ are $2\times2$ matrices
			such that $AB=0$ and $A\neq0$, then $B=0$. Give an argument valid
			for every such A and B if the statement is true, or give a 
			counterexample if false.\\
			\textbf{Solution}\\
			False. A counterexample is:
			\begin{align*}
				A=\left({\begin{array}{cc} 0&1 \\ 0&0  \end{array}}\right) \\
				B=\left({\begin{array}{cc} 1&0 \\ 0&0  \end{array}}\right) \\
				AB=\left({\begin{array}{cc} (0\times1)+(1\times0)&
				(0\times0)+(1\times0) \\ 
				(0\times1)+(0\times0)&(0\times0)+(0\times0) 
				\end{array}}\right) =
				\left({\begin{array}{cc} 0&0 \\ 0&0 \end{array}}\right)
			\end{align*}
			\tab This is a 0 matrix, but both $A$ and $B$ are not 0 matrices.
			Therefore I can conclude this is false. 
		\item (5 points) True or False: There exists a $4\times3$ matrix $A$ so
			that the equation $Ax=b$ is consistent for every vector 
			$b\in\mathbb{R}^4$. Justify your answer.\\
			\textbf{Solution}\\
			\tab False. The only way that this can be true if you can have some
			sort of identity matrix, with an identity matrix you can have all
			free variables in b, however that is impossible to have in a 
			$4\times3$ matrix, making this impossible. 
		\item (5 points) True or False: if $A$ is a $4\times3$ matrix and the
			equation $Ax=b$ is consistent for a particular vector 
			$b\in\mathbb{R}^4$, then the equation $Ax=cb$ is consistent for
			every scalar c. Justify your answer.\\
			\textbf{Solution}\\
			\tab True, if you manipulate a vector with a scalar you can use that
			scalar on the other side of the equation and manipulate it in 
			exactly the same way. 
		\item (5 points) True or False: If $A$ and $B$ are invertible
			$3\times3$ matrices and $C=A^3B^2$, then $C$ is invertible and 
			$C^{-1}=(B^{-1})^2(A^{-1})^3$. Justify your answer.\\
			\textbf{Solution}\\
			\tab True. The solution seems to look a bit strange but right.Let's
			say that $A^3$ = $X$ and $B^2=Y$ so that $C=XY$. In this case the
			inverse would look like:
			\begin{align*}
				C^{-1}=(Y)^{-1}(X)^{-1})
			\end{align*}
			\tab Which, when swapping X and Y with their counterpart values
			would look like:
			\begin{align*}
				C^{-1}=(B^2)^{-1}(A^3)^{-1} \\
				C^{-1}=(B^{-1})^2(A^{-1})^3
			\end{align*}
			\tab which can be shown to be exactly identical. 
	\end{enumerate}
\end{homeworkProblem}
\newpage

\end{document}