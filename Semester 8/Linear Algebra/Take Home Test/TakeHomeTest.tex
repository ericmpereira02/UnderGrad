\documentclass{article}

\usepackage{fancyhdr}
\usepackage{extramarks}
\usepackage{amsmath}
\usepackage{amsthm}
\usepackage{amsfonts}
\usepackage{tikz}
\usepackage[plain]{algorithm}
\usepackage{algpseudocode}
\usepackage{amssymb}
\usepackage{enumitem}
\usepackage{relsize}
\usepackage{textcomp}

\usetikzlibrary{automata,positioning}

%
% Basic Document Settings
%

\topmargin=-0.45in
\evensidemargin=0in
\oddsidemargin=0in
\textwidth=6.5in
\textheight=9.0in
\headsep=0.25in

\linespread{1.1}

\pagestyle{fancy}
\lhead{\hmwkAuthorName}
\chead{\hmwkClass\ (\hmwkClassInstructor\ \hmwkClassTime): \hmwkTitle}
\rhead{\firstxmark}
\lfoot{\lastxmark}
\cfoot{\thepage}

\newcommand\tab[1][1cm]{\hspace*{#1}}
\renewcommand\headrulewidth{0.4pt}


\renewcommand\footrulewidth{0.4pt}
\renewcommand{\theenumi}{\Alph{enumi}}

\setlength\parindent{0pt}

%
% Create Problem Sections
%

\newcommand{\enterProblemHeader}[1]{
    \nobreak\extramarks{}{Problem \arabic{#1} continued on next page\ldots}\nobreak{}
    \nobreak\extramarks{Problem \arabic{#1} (continued)}{Problem \arabic{#1} continued on next page\ldots}\nobreak{}
}

\newcommand{\exitProblemHeader}[1]{
    \nobreak\extramarks{Problem \arabic{#1} (continued)}{Problem \arabic{#1} continued on next page\ldots}\nobreak{}
    \stepcounter{#1}
    \nobreak\extramarks{Problem \arabic{#1}}{}\nobreak{}
}

\setcounter{secnumdepth}{0}
\newcounter{partCounter}
\newcounter{homeworkProblemCounter}
\setcounter{homeworkProblemCounter}{1}
\nobreak\extramarks{Problem \arabic{homeworkProblemCounter}}{}\nobreak{}

%
% Homework Problem Environment
%
% This environment takes an optional argument. When given, it will adjust the
% problem counter. This is useful for when the problems given for your
% assignment aren't sequential. See the last 3 problems of this template for an
% example.
%
\newenvironment{homeworkProblem}[1][-1]{
    \ifnum#1>0
        \setcounter{homeworkProblemCounter}{#1}
    \fi
    \section{Problem \arabic{homeworkProblemCounter}}
    \setcounter{partCounter}{1}
    \enterProblemHeader{homeworkProblemCounter}
}{
    \exitProblemHeader{homeworkProblemCounter}
}

%
% Homework Details
%   - Title
%   - Due date
%   - Class
%   - Section/Time
%   - Instructor
%   - Author
%

\newcommand{\hmwkTitle}{Take Home Test}
\newcommand{\hmwkDueDate}{April 17, 2019}
\newcommand{\hmwkClass}{Linear Algebra}
\newcommand{\hmwkClassTime}{Section 01}
\newcommand{\hmwkClassInstructor}{Dr. Munevver Subasi}
\newcommand{\hmwkAuthorName}{\textbf{Eric Pereira}}

%
% Title Page
%
\pagenumbering{gobble}
\title{
    \vspace{2in}
    \textmd{\textbf{\hmwkClass:\ \hmwkTitle}}\\
    \normalsize\vspace{0.1in}\small{Due\ on\ \hmwkDueDate\ at 11:59pm}\\
    \vspace{0.1in}\large{\textit{\hmwkClassInstructor\ \hmwkClassTime}}
    \vspace{3in}
}

\author{\hmwkAuthorName}
\date{}

\renewcommand{\part}[1]{\textbf{\large Part \Alph{partCounter}}\stepcounter{partCounter}\\}

%
% Various Helper Commands
%

% Useful for algorithms
\newcommand{\alg}[1]{\textsc{\bfseries \footnotesize #1}}

% For derivatives
\newcommand{\deriv}[1]{\frac{\mathrm{d}}{\mathrm{d}x} (#1)}

% For partial derivatives
\newcommand{\pderiv}[2]{\frac{\partial}{\partial #1} (#2)}

% Integral dx
\newcommand{\dx}{\mathrm{d}x}

% Alias for the Solution section header
\newcommand{\solution}{\textbf{\large Solution}}

% Probability commands: Expectation, Variance, Covariance, Bias
\newcommand{\E}{\mathrm{E}}
\newcommand{\Var}{\mathrm{Var}}
\newcommand{\Cov}{\mathrm{Cov}}
\newcommand{\Bias}{\mathrm{Bias}}

\begin{document}

\maketitle

\pagebreak
\pagenumbering{arabic}

%%%%%%%%%%%%%%%%%%%%%%%%%%%%%%%%%%%%%%%%%%%%%%%%%%%%%%%%%%%%%%%%%%%%%%%%%%%%%%%%%
%																				%
%																			    %
%							  PROBLEM 1                                         %
%                                                                               %
%																				%
%%%%%%%%%%%%%%%%%%%%%%%%%%%%%%%%%%%%%%%%%%%%%%%%%%%%%%%%%%%%%%%%%%%%%%%%%%%%%%%%%

\begin{homeworkProblem}
	Let $A$ be an $n\times n$ matrix. 
	\begin{enumerate}[label=(\alph*)]
		\item (5 points) if $\lambda$ is an eigenvalue of $A$, find an eigenvalue $A^k$
		where $k$ is a positive integer. \\
		\textbf{Solution:} \\
		\begin{align*}
			Ax=\lambda x \\
			AAX=A\lambda x\\	
			A^2x=\lambda Ax \\
			A^2x=\lambda (\lambda x) = \lambda^2 x
		\end{align*}
		Now to introduce $A^k$...
		\begin{align*}
			A^{k-1}Ax=A^{k-1}\lambda x \\
			A^kx=\lambda(\lambda^{k-1}x) \\
			A^kx=\lambda^kx
		\end{align*}
		The eigenvalue of $A^k$ is $\lambda^k$
		\item (5 points) if $v$ is an eigenvector of $A$ corresponding to an eigenvalue $\lambda$,
		find an eigenvector of $A^k$ where $k$ is a positive integer. \\
		\textbf{Solution:} \\
		As shown in problem 1(a), where $v$ and $x$ are eigenvectors you can see that when finding the
		eigenvalues of $A^k$ the vectors are never changed, therefore the eigenvector of $A^k$ where $k$
		is a positive integer is $v$.
		
	\end{enumerate}
	\vspace*{\fill}
	\textbf{Used Class notes to answer 1(a) and 1(b)}
\end{homeworkProblem}
\newpage

%%%%%%%%%%%%%%%%%%%%%%%%%%%%%%%%%%%%%%%%%%%%%%%%%%%%%%%%%%%%%%%%%%%%%%%%%%%%%%%%%
%																				%
%																			    %
%							  PROBLEM 2                                         %
%                                                                               %
%																				%
%%%%%%%%%%%%%%%%%%%%%%%%%%%%%%%%%%%%%%%%%%%%%%%%%%%%%%%%%%%%%%%%%%%%%%%%%%%%%%%%%

\begin{homeworkProblem}
	Let $A$ and $B$ be two $n \times n$ matrices. Assume that $B$ is similar to $A$, i.e, there
	exists an $n \times n$ nonsingular matrix $S$ such that $B=S^{-1}AS$. Let $x\neq 0$ be an 
	eigenvector of $B$ corresponding to eigenvalue $\lambda$.
	\begin{enumerate}[label=(\alph*)]
		\item (5 points) Prove that $A$ and $B$ have some eigenvalues, i.e, $\lambda$ is also an
		eigenvalue of A. \\
		\textbf{Solution:} \\
		\begin{align*}
			det|\lambda I-A|=det|\lambda I-S^{-1}BS| \\	
			det|\lambda I-A|=det|\lambda S^{-1}IS-S^{-1}BS| \\
			det|\lambda I-A|=det|S^{-1}(\lambda I-B)S| \\
			det|\lambda I-A|=det|\frac{1}{S}(\lambda I-B)S| \\
			det|\lambda I-A|=det|\lambda I-B| \\
		\end{align*}
		If this is true then A and B must have the same eigenvalues.\\
		\item (5 points) Find the eigenvector of A corresponding to eigenvalue $\lambda$. \\
		\textbf{Solution:} \\
		\begin{align*}
			B=S^{-1}AS \\
			Bx=\lambda x \\
			S^{-1}ASx=\lambda x \\
			ASx=\lambda Sx \\
		\end{align*}
		We are able to move lambda because it is an integer and commutative.
		
		\vspace*{\fill}
		\textbf{Used Class notes to answer question 2(a) and 2(b)}
	\end{enumerate}
\end{homeworkProblem}
\newpage

%%%%%%%%%%%%%%%%%%%%%%%%%%%%%%%%%%%%%%%%%%%%%%%%%%%%%%%%%%%%%%%%%%%%%%%%%%%%%%%%%
%																				%
%																			    %
%							  PROBLEM 3                                         %
%                                                                               %
%																				%
%%%%%%%%%%%%%%%%%%%%%%%%%%%%%%%%%%%%%%%%%%%%%%%%%%%%%%%%%%%%%%%%%%%%%%%%%%%%%%%%%

\begin{homeworkProblem}
	Let $V$ be a subspace of $\mathbb{R}^n$. The orthogonal complement of $V$ in $\mathbb{R}^n$ is
	defined as
	\begin{align*}
		V^\bot = \{x\in \mathbb{R}^n | x \cdot v = 0 \; \forall \; v\in V\}
	\end{align*}
	Prove that $V^\bot$ is also a subspace of $\mathbb{R}^n$. \\
	\textbf{Solution:} \\
	Expanding the vectors out we get:
	\begin{align*}
		V^\bot=\{(x_1,x_2,...,x_n)\in \mathbb{R}^n | (x_1,x_2,...,x_n) \cdot (v_1,v_2,...,v_n) = 0
		\; \forall \; (v_1,v_2,...,v_n)\in V\} 
	\end{align*}
	Doing the dot product out it will look like:
	\begin{align*}
		V^\bot=\{(x_1,x_2,...,x_n)\in \mathbb{R}^n | (x_1v_1+x_2v_2+...+x_nv_n) = 0
		\; \forall \; (v_1,v_2,...,v_n)\in V\} \\
	\end{align*}
	$v$ is 0 vector, because $x\cdot v=0$, which means that either the $x$ or $v$ is the zero vector
	and because $x$ is all reals that means $v$ is the 0 vector. 
	\begin{align*}
		V^\bot=\{(x_1,x_2,...,x_n)\in \mathbb{R}^n | (x_1+x_2+...+x_n)\cdot(0+0+...+0) = 0\} \\
		V^\bot \in \mathbb{R}^n
	\end{align*}
	
	\vspace*{\fill}
	\textbf{Worked with Jacquelyne Miksanek, and Karly Lorenzini on this problem}
\end{homeworkProblem}
\newpage

%%%%%%%%%%%%%%%%%%%%%%%%%%%%%%%%%%%%%%%%%%%%%%%%%%%%%%%%%%%%%%%%%%%%%%%%%%%%%%%%%
%																				%
%																			    %
%							  PROBLEM 4                                         %
%                                                                               %
%																				%
%%%%%%%%%%%%%%%%%%%%%%%%%%%%%%%%%%%%%%%%%%%%%%%%%%%%%%%%%%%%%%%%%%%%%%%%%%%%%%%%%

\begin{homeworkProblem}
	Let $A$ and $B$ be two $n\times n$ orthogonal matrices. Answer the following questions. \\
	\begin{enumerate}[label=(\alph*)]
		\item (2 points) Is $-3A$ orthogonal? \underline{Justify your answer.} \\
		\textbf{Solution:} \\
		Yes, the orthogonality of a set is closed under scalar multiplication
		\item (2 points) Is $-B$ orthogonal? \underline{Justify your answer.} \\
		\textbf{Solution:} \\
		Yes, for the same reason as A, the orthogonality of a set is closed under
		scalar multiplication.
		\item (2 points) Is $A+B$ orthogonal? \underline{Justify your answer.} \\
		\textbf{Solution} \\
		Yes, because the orthogonality of a set is closed under addition. 
		\item (2 points) Is $B^{-1}AB$ orthogonal? \underline{Justify your answer.} \\
		\textbf{Solution:} \\
		Yes, because the set is closed under multiplication.
		\item (2 points) Is $A^T$ invertible? if yes, find its inverse. \\
		\textbf{Solution:} \\
		Yes, because the transpose of an orthogonal matrix is its inverse (i.e
		$A^T=A^{-1}$). This means that the inverse of the transpose the original
		matrix itself. 
		\item (2 points) Is $AB$ orthogonal? \underline{Justify your answer.} \\
		\textbf{Solution:} \\
		Yes, because the set is closed under multiplication. 
		\item (2 points) Is $A^2B^2$ orthogonal? \underline{Justify your answer.} \\
		\textbf{Solution:} \\
		Yes, because the set is closed under multiplication.
	\end{enumerate}
	\vspace*{\fill}
	\textbf{Worked with Jacquelyne Miksanek on this problem}
\end{homeworkProblem}
\newpage

%%%%%%%%%%%%%%%%%%%%%%%%%%%%%%%%%%%%%%%%%%%%%%%%%%%%%%%%%%%%%%%%%%%%%%%%%%%%%%%%%
%																				%
%																			    %
%							  PROBLEM 5                                         %
%                                                                               %
%																				%
%%%%%%%%%%%%%%%%%%%%%%%%%%%%%%%%%%%%%%%%%%%%%%%%%%%%%%%%%%%%%%%%%%%%%%%%%%%%%%%%%

\begin{homeworkProblem}
	
	\begin{enumerate}[label=(\alph*)]
		\item (5 points) Let $A$ be an $n \times n$ orthogonal matrix. Prove that $det(A)=
		\pm 1$. \\
		\textbf{Solution:} \\
		\begin{align*}
			A^{-1}=A^T \\
			det(AA^T) = det(AA^{-1}) \\
			[det(A)]^2=1 \\
			\sqrt{[det(A)]^2}=\sqrt{1} \\
			det(A)=\pm1
		\end{align*}
		\item (5 points) Let $A$ be an $n \times n$ orthogonal matrix. Prove that only eigenvalues
		of A are 1 and -1. \\
		\textbf{Solution:} \\
		\begin{align*}
			det(A)=\pm1 \\
			det(A-\lambda I)=0 \\
			det(A)-\lambda det(I)=0 \\
			det(A)=\lambda det(I) \\
			det(I)=1 \\
			det(A)= \lambda\\
			\pm1=\lambda
		\end{align*}
	\end{enumerate}
	\vspace*{\fill}
	\textbf{Used class notes to answer question 5(a) and 5(b)}
\end{homeworkProblem}
\newpage
\end{document}