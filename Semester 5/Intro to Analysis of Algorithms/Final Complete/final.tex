%% HEADER
%%%%%%%%%%%%%%%%%%%%%%%%%%%%%%%%%%%%%%%%%%%%%%%%%%%%%%%%%%%%%
\documentclass[letterpaper,oneside,10pt]{article}
% Alternative Options:
%	Paper Size: a4paper / a5paper / b5paper / letterpaper / legalpaper / executivepaper
% Duplex: oneside / twoside
% Base Font Size: 10pt / 11pt / 12pt


%% Language %%%%%%%%%%%%%%%%%%%%%%%%%%%%%%%%%%%%%%%%%%%%%%%%%
\usepackage[USenglish]{babel} %francais, polish, spanish, ...
\usepackage[T1]{fontenc}
\usepackage[ansinew]{inputenc}

\usepackage{lmodern} %Type1-font for non-english texts and characters


%% Packages for Graphics & Figures %%%%%%%%%%%%%%%%%%%%%%%%%%
\usepackage{graphicx} %%For loading graphic files


%% Math Packages %%%%%%%%%%%%%%%%%%%%%%%%%%%%%%%%%%%%%%%%%%%%
\usepackage[leqno]{amsmath}

\usepackage{hyperref}
\usepackage{color}
\hypersetup{
	colorlinks = true,
	linkcolor=black,
	urlcolor=red,
	linktoc=all
}

%%%%%%%%%%%%%%%%%%%%%%%%%%%%%%%%%%%%%%%%%%%%%%%%%%%%%%%%%%%%%
%% DOCUMENT
%%%%%%%%%%%%%%%%%%%%%%%%%%%%%%%%%%%%%%%%%%%%%%%%%%%%%%%%%%%%%
\begin{document}

\pagestyle{empty} %No headings for the first pages.


%% Title Page %%%%%%%%%%%%%%%%%%%%%%%%%%%%%%%%%%%%%%%%%%%%%%%
%% ==> Write your text here or include other files.

%% The simple version:
\title{Introduction to Analysis of Algorithms Final\\CSE4081}
\author{Eric Pereira}
\date{December 15, 2017} %%If commented, the current date is used.
\maketitle
\newpage

%% The nice version:
%\input{titlepage} %%You need a file 'titlepage.tex' for this.
%% ==> TeXnicCenter supplies a possible titlepage file
%% ==> with its templates (File | New from Template...).


%% Inhaltsverzeichnis %%%%%%%%%%%%%%%%%%%%%%%%%%%%%%%%%%%%%%%
\tableofcontents %Table of contents
\cleardoublepage %The first chapter should start on an odd page.

\pagestyle{plain} %Now display headings: headings / fancy / ...


%% Section I : Problem %%%%%%%%%%%%%%%%%%%%%%%%%%%%%%%%%%%%%
\section{Problem 1}
\label{sec:Problem 1}

\begin{equation*}
\sum_{i=0}^{n} \sum_{j=0}^{i} \sum_{k=0}^{j} 1 \\
\end{equation*}
\begin{equation*}
\sum_{i=0}^{n} \sum_{j=0}^{i} j
\end{equation*}
\begin{equation*}
\sum_{i=0}^{n} \frac{i(i+1)}{2}
\end{equation*}
\begin{equation*}
\frac{1}{2} \sum_{i=0}^{n} i^{2}+i
\end{equation*}
\begin{equation*}
\frac{1}{2} (\sum_{i=0}^{n} i^{2} + \sum_{i=0} i)
\end{equation*}
\begin{equation*}
\frac{1}{2} (\frac{1}{6}n(n+1)(2n+1) + \frac{n(n+1)}{2})
\end{equation*}
\begin{equation*}
\frac{n^{3} + 3n^{2} + 2n}{6}
\end{equation*}

\section{Problem 2}
\label{sec:Problem 2}

\qquad What is being stated here is that \(O(g(n))\) = \(f(n)\) so that there exists constants, lets say c and \(n_{0}\) so that 0 \(\leq\) \(f(n)\) \(\leq\) \(cg(n)\) for all \(n\) \(\geq\)\(n_{0}\) and vice versa with \(f(n) = O(g(n))\).  \(f(n) = O(g(n))\) means \(f(n) \leq O(g(n))\).

\section{Problem 3}
\label{sec:Problem 3}
\qquad As stated in problem 2, we will test out what was proven but square all the numbers. \(O(g(n)^{2})\) = \(f(n)^{2}\) so that there exists constants, lets say c and \(n_{0}\) so that 0 \(\leq\) \(f(n)^{2}\) \(\leq\) \(cg(n)^2\) for all \(n\) \(\geq\)\(n_{0}\) and vice versa with \(f(n)^{2} = O(g(n)^{2})\). \(f(n)^{2} = O(g(n)^{2})\) means \(f(n)^{2} \leq O(g(n)^{2})\).

\section{Problem 4}
\paragraph{}
	For both problems an example with n being double will show how difficult it is to keep track when both numbers are doubled. The factor that \(2^{n}\) increases when doubled is \(2^{n}\), compared to \(2^{3}\) when seen in the \(n^{3}\) example. This shows that it is not tractable. 
\label{sec:Problem 4}
\subsection{4a}
\label{sec:4a}
\begin{center}
Doubling input increases output by a factor of 2^{3}
\end{center}
\begin{gather*}
f(n) = n^3 \\
f(2n) = (2n)^3 \\
f(2n) = 2^3f(n) 
\end{gather*}
\begin{center}
lets say n = 10
\end{center}
\begin{gather*}
f(10) = 10^3 = 1,000 \\
f(20) = 20^3 = 8,000
\end{gather*}
\subsection{4b}
\label{sec:4b}
\begin{center}
Doubling input increases output by a factor of 2^{n}
\end{center}
\begin{gather*}
f(n) = 2^n \\
f(2n) = 2^{2n} \\
f(2n) = 2^nf(n)
\end{gather*}
\begin{center}
lets say n = 10
\end{center}
\begin{gather*}
f(10) = 2^{10} = 1,024 \\
f(20) = 2^{20} = 1,048,576
\end{gather*}


\section{Problem 5}
\label{sec:Problem 5}
\begin{equation*}

\end{equation*}
\begin{equation*}
T(n)=3T(n/3)+n
\end{equation*}
\begin{equation*}
T(n)\leq 3(c\frac{n}{3}\log_{3}\frac{n}{3}+n
\end{equation*}
\begin{equation*}
T(n) = cn\log_{3}\frac{n}{3}+n
\end{equation*}
\begin{equation*}
T(n) = cn\log_{3}n-cn\log{3}3+n
\end{equation*}
\begin{equation*}
T(n) = cn\log_{3}n-cn+n
\end{equation*}

\section{Problem 6}
\label{sec:Problem 6}

\begin{equation*}
T(n) = 3T(n/3) + n
\end{equation*}
\begin{equation*}
T(n) = 3(3T(n/9)+n/3))+n
\end{equation*}
\begin{equation*}
T(n) = 9(3T(n/27) + 3n
\end{equation*}
\begin{equation*}
T(n) = 27T(n/27) + 3n
\end{equation*}
\begin{equation*}
T(n) = 3^{i}T(n/3^{i})+i*n
\end{equation*}
\begin{gather*}
\frac{n}{3^{i}} = 1 = i = \log_{3}n
\end{gather*}
\begin{equation*}
T(n) = 3^{log_{3}n}T(1)+\log_{3}n
\end{equation*}
\begin{equation*}
T(n) = n+\log_{3}n*n
\end{equation*}
\begin{equation*}
T(n) = O(n\logn)
\end{equation*}

\section{Problem 7}
\label{sec:Problem 7}

\begin{equation*}
G(z)=\sum_{n=0}^{\infty} t_{n}z^{n}
\end{equation*}
\begin{equation*}
G(z)=\sum_{n=1}^{\infty}[t_{n-1}+2^{n-1}]z^n
\end{equation*}
\begin{equation*}
G(z) = \sum_{n=1}^{\infty}t_{n-1}z^n + \sum_{n=1}^{\infty}2^{n-1}z^{n}
\end{equation*}
\begin{equation*}
G(z) = z\sum_{n=0}^{\infty}t_{n}z^{n}+z\sum_{n=0}^{\infty}2^{z}z^{n}
\end{equation*}
\begin{equation*}
G(z) = zG(z) + \frac{z}{1-2z}
\end{equation*}
\begin{equation*}
G(z) - zG(z) = \frac{z}{1-2z}
\end{equation*}
\begin{equation*}
G(z)[1-z]=\frac{z}{1-2z}
\end{equation*}
\begin{equation*}
G(z) = \frac{z}{(1-z)(1-2z)}
\end{equation*}
\begin{equation*}
G(z) = \frac{1}{1-2z} - \frac{1}{1-z}
\end{equation*}
\begin{equation*}
G(z) = \sum_{n=0}^{\infty}2^{n}z^{n}-\sum_{n=0}^{\infty}z^{n}
\end{equation*}
\begin{equation*}
G(z) = \sum_{n=0}^{\infty}(2^{n}-1)z^{n}
\end{equation*}

\section{Problem 8}
\label{sec:Problem 8}
\subsection{8a}
\label{sec:8a}
\begin{equation*}
O(n + M)
\end{equation*}
\subsection{8b}
\label{sec:8b}
\begin{equation*}
O(nM)
\end{equation*}
\subsection{8c}
\label{sec:8c}
\qquad I think there could be two greedy algorithms used for this problem.
The first greedy algorithm could be to do the quickest jobs first, maximizing
amount of chores done. The next greedy algorithm is to try and do chores that pay the 
best. The time complexity for each would be O(n), and these are terrible was to do the
problem, they will not always produce an optimal solution. 
\section{Problem 9}
\label{sec:Problem 9}
\subsection{9a}
\label{sec:9a}
\begin{equation*}
n \choose k
\end{equation*}
\subsection{9b}
\label{sec:9b}
\begin{equation*}
k \choose 2
\end{equation*}
\subsection{9c}
\label{sec:9c}
\begin{equation*}
O({n \choose k} {k \choose 2})
\end{equation*}
\subsection{9d}
\label{sec:9d}
\begin{equation*}
n*n!
\end{equation*}
\subsection{9e}
\label{sec:9e}
\qquad It is possible to use a nested for loop in order to find the 
independent amount of vertices. You could have one loop go through each k element subset and another loop go through sets within graph G. This would allow for a polynomial time algorithm, and still get the correct answer. 


\end{document}

